\documentclass[cn,10pt,math=newtx,citestyle=gb7714-2015,bibstyle=gb7714-2015]{elegantbook}
%,math=newtx,math=mtpro2

\usepackage{ntheorem}
\usepackage{siunitx}
\usepackage{stmaryrd}
\usepackage{braket}

\def\bm{\boldsymbol}
\def\mbf{\mathbf}
\def\bf{\textbf}
\def\mf{\mathsf}
\def\ms{\mathscr}
\def\mc{\mathcal}
\def\mr{\mathrm}
\def\d{\mathrm d}
\def\e{\mathrm e}
\def\i{\mathrm i}
\def\C{\mathbb C}
\def\R{\mathbb R}
\def\N{\mathbb N}
\def\Z{\mathbb Z}
\def\p{\partial}
\def\T{\mathrm T}
\def\vphi{\varphi}
\def\ve{\varepsilon}
\def\dg{\dagger}
\def\ra{\rightarrow}
\def\lra{\leftrightarrow}
\def\ol{\overline}
\def\opl{\oplus}
\def\ox{\otimes}
\def\td{\tilde}
\def\P{\mathcal P}
\def\srule{\noindent\rule{\linewidth}{0.3mm}}

\title{化学原理综观}
\subtitle{Principle of Chemistry: A Comprehensive Perspective}

\author{Benzoin}
\institute{SA科学社}
\date{\today}
\version{0.0}
\bioinfo{邮箱}{luoweiliang7@pku.edu.cn}

\extrainfo{More is different.}

\setcounter{tocdepth}{3}

\logo{logo1.jpg}
\cover{cover.jpg}

% 本文档命令
\usepackage{array}
\newcommand{\ccr}[1]{\makecell{{\color{#1}\rule{1cm}{1cm}}}}

\definecolor{customcolor}{RGB}{32,178,170}
\colorlet{coverlinecolor}{customcolor}

\begin{document}

\maketitle
\frontmatter

\chapter*{特别声明}

\markboth{Introduction}{前言}

我太菜了

\begin{flushright}
Benzoin\\
\today
\end{flushright}

\tableofcontents

\mainmatter

\chapter{谐振子与氢原子}
\section{量子力学的涟漪}
\subsection{质点力学的初步框架}
\subsubsection{质点的运动满足牛顿运动方程}
\textbf{质点(mass point)} 是一个没有形状和大小,以\textbf{质量(mass)}$m$ 为唯一参数的点,这一模型是 \textbf{牛顿力学(Newtonian mechanics)}理论描摹物质世界的根基。我们习惯于建立一个直角坐标系,就可以用一个向量描述质点的\textbf{位置(position)} $\bm x=(x_1,\cdots,x_D)\in\R^D$,其中 $D$ 是质点所处空间的维数,例如对于三维空间 $D=3$,对于直线或平面上的质点 $D=1,2$。\footnote{只使用到前三个维度时,三个坐标分量又常记作 $x,y,z$,位置又常写作 $\bm r$}

\textbf{运动学(kinematics)}研究质点的运动,即其位置随时间变化的规律:

\begin{itemize}
    \item 质点的运动意味着位置是时间的函数 $\bm x(t)$,这个函数称为\textbf{轨迹(trajectory)}。
    \item 位置对时间的变化率称为\textbf{速度(velocity)}$\bm v(t)=\dot{\bm x}(t)$。
    \item 速度对时间的变化率称为\textbf{加速度(acceleration)}$\bm a(t)=\dot{\bm v}(t)=\ddot{\bm x}(t)$。
\end{itemize}

\textbf{动力学(dynamics)}研究质点的运动与相互作用的关系。牛顿力学用\textbf{力(force)}向量 $\bm F$ 描述质点与外界的相互作用:牛顿第二定律定量地指出:在惯性系中,$\bm F=m\bm a$。因此,只要已知质点的受力,就得到质点的运动方程:
\begin{equation}\label{eq:Nwtn_2}
    \bm F = m\frac{\d^2 \bm x}{\d t^2}
\end{equation}
这是一个二阶常微分方程,结合出发时刻 $t=0$ 的位置初值 $\bm x(0)$ 和速度初值 $\bm v(0)$,理论上就可以确定轨迹方程 $\bm x(t)$,从而完整描述了运动;这也是运动学中引入二阶导数并至此为止的动机。因此,在牛顿力学框架下,物理学的首要问题就是确定 $\bm F$ 的表达式,然后考虑如何求解微分方程~\ref{eq:Nwtn_2},前者概括了我们对物理问题的建模,而后者往往依赖于大量数学和计算方面的技巧。以下两个例子中,我们会看到具体的物理模型负责为牛顿力学框架所需的力提供定量描述。

\srule

\begin{instance}[谐振子]
考虑一个一端固定、沿一条直线振荡、质量忽略不计的弹簧所连接的质点。如果弹簧处于原长时质点位置为 $x=0$,则 \textbf{胡克(Hooke)定律}表明,质点受到的弹力满足 $F=-\kappa x$,其中\textbf{劲度系数(stiffness factor)} $\kappa$ 是弹簧的固定参量。由于弹力是质点的唯一受力,代入方程~\ref{eq:Nwtn_2} 的一维版本,将得到:
\begin{gather*}
-\kappa x = m\ddot{x}\nonumber\\
    \ddot{x} + \omega^2 x=0
\end{gather*}
其中 $\omega=\sqrt{\kappa/m}$ 称为\textbf{角频率(angular frequency)}。

幸运的是,这个足够简单的方程可以直接写出通解的解析式:
\begin{equation*}
    x(t) = A\e^{\i\omega t}+B\e^{-\i\omega t}
\end{equation*}
其中系数 $A,B$ 的取值需要初值来确定。例如,如果 $x(0)=x_0,\dot{x}(0)=v_0$,就可以代入计算:
\begin{gather}\label{eq:1hrmnc_oscltr}
    x(0) = A+B=x_0\nonumber\\
    \dot{x}(0) =(A-B)\i\omega = v_0\nonumber\\
    x(t)= \frac{x_0\omega-\i v_0}{2\omega}\e^{\i\omega t}+\frac{x_0\omega+\i v_0}{2\omega}\e^{-\i\omega t}=x_0\cos\omega t+\frac{v_0}{\omega}\sin{\omega t}
\end{gather}
可见轨迹在初值条件下已经完全确定,是以 $T=2\pi/\omega$ 为周期的的正弦型往复运动,其振幅、相位取决于初值。这种运动模式称为\textbf{简谐运动(simple harmonic oscillation)},这个被弹簧控制的质点也就称为\textbf{谐振子(harmonic oscillator)}。
\end{instance}

\srule

\begin{instance}[氢原子]
氢原子高居元素周期表的首位,是化学中最简单的系统,但正所谓大道至简,其中蕴含着丰富的理论宝藏。氢原子的\textbf{电子(electron)}和\textbf{原子核(nucleus)}可以视作具有\textbf{电荷(charge)}(量)参数 $q$ 的质点,又称\textbf{点电荷(point charge)}。电荷量决定了点电荷之间普遍存在的\textbf{静电力(electrostatic force)}的大小:电子和氢核的电荷量分别为 $-e$ 和 $+e$,位置分别处于 $\bm r_\text{e},\bm r_\text{p}$,则根据\textbf{库仑(Coulumb)定律},电子受到来自氢核的静电力为
\begin{equation*}
    \bm F_\text{e} = -\frac{e^2}{4\pi\epsilon_0}\frac{\bm r_\text{e}-\bm r_\text{p}}{\left|\bm r_\text{e}-\bm r_\text{p}\right|^3}
\end{equation*}
氢核受到来自电子的静电力为
\begin{equation*}
    \bm F_\text{p} = \frac{e^2}{4\pi\epsilon_0}\frac{\bm r_\text{e}-\bm r_\text{p}}{\left|\bm r_\text{e}-\bm r_\text{p}\right|^3}
\end{equation*}
其中 $\epsilon_0$ 为\textbf{真空介电常数(permittivity of vacuum)}。假设电子和核的质量分别为 $m_\text{e}$ 和 $m_\text{p}$,于是在这两个质点的视角下分别代入牛顿运动方程~\ref{eq:Nwtn_2} 的三维版本,要求解的方程变为一个微分方程组:
\begin{equation}\label{eq:H_pnt_chrg}
    \begin{cases}
        \ddot{\bm r}_\text{e} = -\frac{1}{4\pi\epsilon_0}\frac{e^2}{m_\text{e}}\frac{\bm r_\text{e}-\bm r_\text{p}}{\left|\bm r_\text{e}-\bm r_\text{p}\right|^3}\\
        \ddot{\bm r}_\text{p} = -\frac{1}{4\pi\epsilon_0}\frac{e^2}{m_\text{p}}\frac{\bm r_\text{p}-\bm r_\text{e}}{\left|\bm r_\text{e}-\bm r_\text{p}\right|^3}\\
    \end{cases}
\end{equation}
这显然比谐振子更加复杂,我们将在接下来的章节逐步讨论其求解。
\end{instance}

\srule

\subsubsection{二体问题约化为单体问题}
求解氢原子的困难之处在于微分方程组中变量间的关联——质点的位置演化依赖于其受力,但每个质点的受力都依赖于另一个质点的位置。两个质点互相影响对方运动状态的问题称为二体问题,但通过一定的数学处理,我们能够从中拆出两个互不相干的、等效于两个独立质点的轨迹,大大简化了求解系统轨迹的过程。

回顾运动方程~\ref{eq:H_pnt_chrg} 可以发现:两个点电荷所受的力的大小和方向只与这两点之间的相对位置 $\bm r_\text{pe} =\bm r_\text{e}-\bm r_\text{p}$ 有关,而且满足牛顿第三定律 $\bm F_\text e+\bm F_\text p=0$。基于此考虑一般的两体问题:两质点 1,2 的质量为 $m_1,m_2$,位置为 $\bm r_1,\bm r_2$,且满足以下条件:
\begin{enumerate}
    \item 两质点所受的力只与相对位置 $\bm r_{12}=\bm r_1-\bm r_2$ 有关
    \item 两质点所受的力完全来自于其间的相互作用:设质点 $i$ 所受合力 $\bm F_i$,受到来自质点 $j$ 的力为 $\bm F_{ij}$,则
    \begin{equation*}
        \bm F_1(\bm r_{12}) = \bm F_{12}(\bm r_{12}),\quad \bm F_2(\bm r_{12}) = \bm F_{21}(\bm r_{12})
    \end{equation*}
    \item 两质点的相互作用服从牛顿第三定律:$\bm F_{12}(\bm r_{12}) = -\bm F_{21}(\bm r_{12})$。
\end{enumerate}
代入运动方程组 \ref{eq:H_pnt_chrg},就会发现
\begin{equation}\label{eq:2bdy_rltv}
    \ddot{\bm r}_{12} = \ddot{\bm r}_1 - \ddot{\bm r}_2 = \frac{\bm F_1(\bm r_{12})}{m_1} - \frac{\bm F_2(\bm r_{12})}{m_2} = \frac{\bm F_{12}(\bm r_{12})}{\mu},\quad \frac{1}{\mu} = \frac{1}{m_1} + \frac{1}{m_2}
\end{equation}
这一方程简化为单质点牛顿运动方程的形式,仿佛一个拥有质量 $\mu$,位于 $\bm r$ 的质点在力 $\bm F_{12}$ 的作用下运动。因此 $\mu$ 称为两质点体系的\textbf{约化质量(reduced mass)}。

接下来令 $\bm r_\text c=\frac{m_1\bm r_1+m_2\bm r_2}{M}$,即两个质点位置按其质量加权后的平均值,称之为\textbf{质心(mass center)}位置。设总质量为 $M=m_1+m_2$,代入运动方程组 \ref{eq:H_pnt_chrg}:
\begin{equation}\label{eq:2bdy_msscntr_eq}
    \ddot{\bm r}_\text c = \frac{m_1}{M}\ddot{\bm r}_1+\frac{m_2}{M}\ddot{\bm r}_2 = \frac{\bm F_1+\bm F_2}{M}=\bm 0
\end{equation}
这一方程也简化为牛顿运动方程的形式,质心仿佛一个位于 $\bm r_\text c$ 的不受力质点,因此无需力 $\bm F_{12}$ 的具体形式就可以求解该方程:对于初始位置 $\bm r_\text c(0)=\bm r_{\text c,0}$,初始速度 $\dot{\bm r}_\text c(0)=\bm v_{\text c,0}$,质心匀速直线运动:
\begin{equation}\label{eq:2bdy_msscntr}
    \bm r_\text c(t)=\bm r_{\text c,0}+\bm v_{\text c,0}t
\end{equation}
如果定义质点的\textbf{动量(momentum)} $\bm p=m\dot{\bm r}$,考虑两个质点的总动量 $\bm p_\text{tot}=\bm p_1+\bm p_2$,则式~\ref{eq:2bdy_msscntr_eq} 表明 $\ddot{\bm r}_\text c=\dot{\bm p}_\text{tot}=\bm 0$,即总动量为定值。

从 $\bm r_\text c$ 和 $\bm r$ 可以轻易回到 $\bm r_1,\bm r_2$:小心验算可知
\begin{equation}\label{eq:2bdy_bck}
    \bm r_1 = \bm r_\text c+\frac{m_1}{M}\bm r_{12}, \bm r_2 = \bm r_\text c-\frac{m_2}{M}\bm r_{12}
\end{equation}
而轨迹 $\bm r_\text c(t)$ 又如此平凡,所以只要求解出相对位置的轨迹 $\bm r(t)$,整个二体问题就迎刃而解。

作为回顾,两体问题的运动方程是相互耦合的:即一个质点的受力取决于另一个质点的位置。但上述过程将其解耦为~\ref{eq:2bdy_rltv} 和~\ref{eq:2bdy_msscntr_eq} 两个单体问题,使问题得到了极大的简化。

\srule

\begin{instance}[谐振子的约化]
考虑一根弹簧两端固定着两个质量为 $m_1,m_2$ 的质点,它们始终约束在一条直线上运动。弹簧的劲度系数为 $k$,原长为 $l$,设两质点的坐标为 $x_1,x_2$,初始位置为 $x_{10},x_{20}\, (x_{10}>x_{20})$,初始速度为 $v_{10},v_{20}$。根据以上思路:
\begin{itemize}
    \item 先求相对位置 $x_{12}=x_1-x_2$ 的运动,约化质量 $\mu = \frac{m_1m_2}{m_1+m_2}$,质点间相互作用 $\bm F_{12} = -\kappa (x-l)$,代入方程 \ref{eq:2bdy_rltv},得到
    \begin{equation*}
        \ddot{x}_{12} = -\frac{\kappa}{\mu} (x_{12}-l)
    \end{equation*}
    其初始条件为 $x_{12}(0) = x_{10}-x_{20}$,$\dot{x}(0) = \dot{x}_1(0)-\dot{x}_2(0) = v_{10}-v_{20}$,于是直接利用单个谐振子质点的结论 \ref{eq:1hrmnc_oscltr},得到
    \begin{equation*}
        x_{12}(t) = l + \left(x_{10}-x_{20}-l\right)\cos\omega t+\frac{v_{10}-v_{20}}{\omega}\sin\omega t
    \end{equation*}
    其中角频率 $\omega=\sqrt{\kappa/\mu}$。
    \item 再求质心 $x_{\text c}=\frac{m_1x_1+m_2x_2}{M},\ M=m_1+m_2$ 的运动。其初始位置为 $x_{\text c,0}=\frac{m_1x_{10}+m_2x_{20}}{M}$,初始速度为
    \begin{equation*}
        v_{\text c,0} = \dot{x}_{\text c}(0) = \frac{m_1v_{10}+m_2v_{20}}{M}
    \end{equation*}
    代入 \ref{eq:2bdy_msscntr},得到
    \begin{equation*}
        x_{\text c}(t) = \frac{1}{M}\left[m_1x_{10}+m_2x_{20} + \left(m_1v_{10}+m_2v_{20}\right)t\right]
    \end{equation*}
    \item 最后回到两个质点各自的轨迹。代入 \ref{eq:2bdy_bck},得到
    \begin{gather*}
        x_1(t) = \frac{m_1x_{10}+m_2x_{20}+m_1l}{M}+\frac{m_1v_{10}+m_2v_{20}}{M}t + \frac{m_1(x_{10}-x_{20}-l)}{M}\cos\omega t +\frac{m_1(v_{10}-v_{20})}{M\omega}\sin\omega t\\
        x_2(t) = \frac{m_1x_{10}+m_2x_{20}-m_2l}{M}+\frac{m_1v_{10}+m_2v_{20}}{M}t - \frac{m_2(x_{10}-x_{20}-l)}{M}\cos\omega t -\frac{m_2(v_{10}-v_{20})}{M\omega}\sin\omega t
    \end{gather*}
\end{itemize}
这样就解出了两个质点的运动方程。
\end{instance}

\srule

\begin{instance}[氢原子的约化]\label{ins:H_atm_rdc}
在氢原子的点电荷模型中使用前述的两体方法:用代表电子的 e 和代表原子核的 p 替代下标 1,2,如果已知相对位置 $\bm r_\text{ep}=\bm r_\text e-\bm r_\text p$ 的运动,则后续的求质心运动、求两质点分别的运动轨迹都会变得非常简单,因此该问题的核心就是处理以下相对位置的运动方程:
\begin{equation*}
    \ddot{\bm r}_\text{ep} = -\frac{1}{4\pi\epsilon_0}\frac{e^2}{\mu}\frac{\bm r_\text{ep}}{r_\text{ep}^3}
\end{equation*}
它等价于一个质量为 $\mu$ 的粒子,在 $\bm r_\text{ep}$ 处总是受到力 $-\frac{e^2}{4\pi\epsilon_0}\frac{\bm r}{r^3}$ 时的运动。
\end{instance}

\srule

\subsubsection{势能函数代替力描述相互作用}
谐振子或约化后的双点电荷氢原子模型中,质点的受力只与其位置有关,即 $\bm F=\bm F(\bm r)$。然而在高于一维的情况中,有时不必用向量函数 $\bm F:\R^3\ra\R^3$ 来描述质点所受的相互作用与位置的关系,接下来的分析将从\textbf{能量(energy)}视角将其简化。

从一维问题开始,我们熟悉质点运动中的两类能量:
\begin{itemize}
    \item \textbf{势能(potential energy)}$V(x) = -\int^x_{x_0} F(x')\d x'$,它仅是位置的函数,式中的积分下限 $x_0$ 任意,是人为选定的势能零点,所以这个函数 $V(x)$ 任意加减一个常数不改变其物理意义。
    \item \textbf{动能(kinetic energy)}$T(\dot{x})=\frac{1}{2}m \dot{x}^2$,它仅是速度的函数,总是非负数,只有在质点静止时取到 0。
\end{itemize}
动能和势能之和称为\textbf{总能(total energy)},它可以写作位置和速度的函数,也可以进一步写作事件的函数。
\begin{equation*}
    E(t) = E(x(t),\dot x(t))=\frac{1}{2}m(\dot x(t))^2+V(x(t))
\end{equation*}
之所以要引入这一物理量,是因为在整条轨迹上观察总能的变化,可得
\begin{equation*}
    \frac{\d E}{\d t}=m\dot x\ddot x+V'(x)\dot x=\dot x(m\ddot x+V'(x))=\dot x(m\ddot x-F)=0
\end{equation*}
这说明总能是一个\textbf{守恒量(conserved quantity)}。总能守恒的背后是动能和势能不断转化的过程:质点在外力 $F$ 的作用下从位置 $x_1$ 移动到 $x_2$,有 $\int_{x_1}^{x_2}F(x)\text dx=V(x_1)-V(x_2)$,即外力在这段位移上所做的功 $W$ 等于势能的减少量;而根据动能定理,外力做功等于动能的增加量,与势能的减少量相抵。

接下来的例子展示了如何根据质点的受力写出它的势能函数。

\srule

\begin{instance}[一维谐振子势]
$x=0$ 的位置对应于弹簧处于原长,我们将其指定为势能零点即 $V(0)=0$,然后计算积分即可:
    \begin{equation*}
        V(x) = -\int_0^x -\kappa x'\d x'=\frac{1}{2}\kappa x^2=\frac{1}{2}m\omega^2x^2
    \end{equation*}
\end{instance}

\srule

在更高维的情况中,动能的定义自然推广为 $T(\dot{\bm r})=\frac{1}{2}m \dot{\bm r}^2$,为了保证能量守恒,势能函数 $V(\bm r)$ 与力函数 $\bm F(\bm r)$ 需要满足以下关系:

\begin{theorem}[能量守恒定律-v1.0]
    如果一个质点的受力 $\bm F(\bm r)$ 只与其位置 $\bm r$ 有关,则函数 $V(\bm r)$ 满足 $-\nabla V=\bm F$ 等价于在任何符合牛顿第二定律的轨迹上,能量函数
    \begin{equation*}
        E(\bm r,\dot{\bm r})=\frac{1}{2}m|\dot{\bm r}|^2+V(\bm r)
    \end{equation*}
    守恒。
\end{theorem}
\begin{proof}
    首先计算
    \begin{equation}
        \frac{\d E}{\d t}=m\dot{\bm r}\cdot\ddot{\bm r}+\nabla V\cdot\dot{\bm r}=\dot{\bm r}(m\ddot{\bm r}+\nabla V)=\dot{\bm r}(\bm F+\nabla V)
    \end{equation}
    如果能量函数守恒,即 $\text dE/\text dt\equiv 0$,考虑一条非静止的轨迹,即 $\dot{\bm r}\ne 0$,于是可以推出 $-\nabla V=\bm F$。相反,若 $-\nabla V=\bm F$,也会自然得到 $\text dE/\text dt\equiv 0$。
\end{proof}

在一维情况中,我们直接给出了 $V(x)$ 作为 $F(x)$ 积分的形式,它总是满足 $-\nabla V=\bm F$。但在更高维情况中,并非每一个力函数都以 $-\nabla V=\bm F$ 的形式与势能函数对应,只有\textbf{保守力(conservative force)}才满足这一性质,援引向量分析中的结论可知:

\begin{theorem}[保守力的等价条件]\label{thm:cndtn_cnsrvtv_frc}
    如果 $D$ 是 $\R^3$ 中的单连通区域,而 $\bm F(\bm r)$ 是光滑函数,则 $\bm F$ 是保守力当且仅当 $\nabla\times\bm F=\bm 0$。此时若选定势能零点 $\bm r_0$,则存在以第二类曲线积分定义的势能函数
    \begin{equation*}
        V(\bm r) = -\int_{\bm r_0}^{\bm r} \bm F(\bm r')\cdot\bm{\d r'}
    \end{equation*}
    这个积分的值与从 $\bm r_0$ 到 $\bm r$ 的积分路径无关。
\end{theorem}
\begin{remark}
    根据定义,$\bm F:\R^3\ra \R^3$ 是 $V:\R^3\ra\R$ 的梯度场,而在单连通区域上梯度场是无旋的,无旋场的第二类曲线积分具有路径无关性。
\end{remark}

氢原子的点电荷模型在约化后是三维空间中的质点运动,我们首先验证其静电力的保守性:

\srule

\begin{instance}[静电势]\label{ins:Clmb_ptntl}
根据例 \ref{ins:H_atm_rdc} 的讨论,设 $k=\frac{e^2}{4\pi\epsilon_0}$,则约化后的质点受力为 $\bm F=-k\frac{\bm r}{r^3}$。

\begin{enumerate}
    \item 用定理 \ref{thm:cndtn_cnsrvtv_frc} 验证保守力:$\nabla\times\bm F$ 的 $x$ 分量为
\begin{equation*}
    \frac{\p}{\p x}\left(-k\frac{y}{\left(x^2+y^2+z^2\right)^{3/2}}\right) - \frac{\p}{\p y}\left(-k\frac{x}{\left(x^2+y^2+z^2\right)^{3/2}}\right)  = \frac{3k}{2}\frac{2xy-2xy}{\left(x^2+y^2+z^2\right)^{5/2}} = 0
\end{equation*}
对 $y$ 和 $z$ 分量也可以作类似计算,从而验证了 $\nabla\times\bm F=\bm 0$。

    \item 在一条积分路径上计算势能的形式:习惯上,对于静电势能一般将无穷远点设置为势能零点。为了精准地表述无穷远点的概念,不妨在球坐标下考虑点 $\bm r_0 = (r_0,\theta_0,\phi_0)$ 并研究 $r_0\ra\infty$ 时的极限:
\begin{equation*}
    V(\bm r) = -\lim_{r_0\ra\infty}\int_{\bm r_0}^{\bm r} \bm F(\bm r')\cdot\d\bm r'
\end{equation*}
假设 $\bm r$ 在极坐标下的表示为 $(r,\theta,\phi)$,则选择一条先径向靠近原点、后在半径固定的球面上调整角度的路径,即
\begin{align*}
    V(\bm r) &= -\lim_{r_0\ra\infty}\int_{(r_0,\theta_0,\phi_0)}^{(r,\theta_0,\phi_0)} \bm F(\bm r')\cdot\d\bm r' + \int_{(r,\theta_0,\phi_0)}^{(r,\theta,\phi)} \bm F(\bm r')\cdot\d\bm r'\\
    &= \lim_{r_0\ra\infty}\int_{r_0}^r \frac{k}{r'^2}\d r' + 0
\end{align*}
第二个等号中,前一项的积分路径与 $\bm F$ 同向,因此 $\bm F\cdot\d\bm r'$=$F\d r'$;后一项的积分路径在半径为 $r_0$ 的球面上,$\d\bm r'$ 朝向这一球面的切向,与径向的 $\bm F(\bm r')$ 垂直,对积分无贡献。因此
\begin{align*}
    V(\bm r) &= \lim_{r_0\ra\infty}k\int_{r_0}^r\frac{\d r}{r^2} = -\frac{k}{r}
\end{align*}

    \item 验证所得势能与力的关系:求势能函数负梯度的 $x$ 分量:
\begin{equation*}
    -\frac{\p}{\p x}\left(-\frac{k}{\left(x^2+y^2+z^2\right)^{1/2}}\right) = -\frac{1}{2}\frac{2xk}{\left(x^2+y^2+z^2\right)^{3/2}} = -\frac{kx}{r^3}
\end{equation*}
对 $y,z$ 分量作类似计算,可知 $-\nabla V=\bm F$。
\end{enumerate}
\end{instance}

\srule

\subsubsection{具有平移对称性的质点系的动量守恒}
我们已经在谐振子和氢原子模型中处理了两个质点的问题。我们将在这里更详细地描述三维空间中 $N$ 个质点组成的\textbf{质点系(system of mass points)}。 

\begin{itemize}
    \item 位置:$\bm r_1,\cdots,\bm r_N$ 实际可以用一个 $3N$ 维向量来表达:
\begin{equation*}
    \bm r=(\bm r_1,\cdots,\bm r_N)=(x_1,y_1,z_1,\cdots,x_N,y_N,z_N)
\end{equation*}
    向量函数 $\bm r(t)=(\bm r_1(t),\cdots,\bm r_N(t))$ 仍称为质点系的轨迹。
    \item 速度与加速度:$\bm r$ 对时间的导数 $\dot{\bm r}$ 就是其中各质点位置随时间的导数的组合,因此 $\dot{\bm r},\ddot{\bm r}$ 描述了整个质点系的速度和加速度。
    \item 质量:$m_1,\dots,m_N$ 也可以用质量矩阵 $\mbf M = \text{Diag}(m_1,m_1,m_1,\cdots,m_N,m_N,m_N)$ 来表达,值得注意的是,质量矩阵只有对角线是非零的,且每个质点的质量连续重复了三次。
    \item 动量:$\bm p=(\bm p_1,\cdots,\bm p_N)=(\bm m_1\dot{\bm r}_1,\cdots,\bm m_N\dot{\bm r}_N)$,利用前面建立的质量矩阵,也可以写出 $\bm p = \mbf M\dot{\bm r}$,此时 $\bm r\in\R^{N\times 1}$ 视作一个列向量。
    \item 总动能:
    \begin{equation*}
        \sum_{i=1}^N\frac{1}{2}m_i\dot r_i^2 = \frac{1}{2}\bm r^\T \mbf M\bm r
    \end{equation*}
\end{itemize}

如果体系中质点超过两个,二体问题约化技术不再适用,但式 \ref{eq:2bdy_msscntr_eq} 的精神得以延续:若它们的受力满足以下条件:
\begin{enumerate}
    \item 每个质点所受的力只与各质点的相对位置 $\bm r_{ij}=\bm r_i-\bm r_j$ 有关
    \item 两质点所受的力完全来自于其间的相互作用:设质点 $i$ 所受合力 $\bm F_i$,受到来自质点 $j$ 的力为 $\bm F_{ij}$,则
    \begin{equation*}
        \bm F_i(\bm r_{i1},\cdots,\bm r_{iN}) = \sum_{j\ne i}\bm F_{ij}(\bm r_{ij})
    \end{equation*}
    \item 两质点的相互作用服从牛顿第三定律:$\bm F_{ij}(\bm r_{ij}) = -\bm F_{ji}(\bm r_{ij})$。
\end{enumerate}
则延续前文思路,定义质心位置为质量加权平均的质点位置
\begin{equation*}
    \bm r_\text c=\frac{1}{M}\sum_i m_i\bm r_i
\end{equation*}
则可以推出其运动方程
\begin{equation}
    \ddot{\bm r}_\text c=\frac{1}{M}\sum_i m_i\frac{\bm F_i}{m_i} = \frac{1}{M}\sum_i\bm F_i = \frac{1}{M}\sum_i\sum_{j\ne i}\bm F_{ij}=\bm 0
\end{equation}
最后一个等号是因为在对 $i,j$ 的双重求和中,$\bm F_{ij}$ 和 $\bm F_{ji}$ 都会出现且因牛顿第三定律相互抵消。从动量的观点来看,这意味着
\begin{equation*}
    \dot{\bm p}_\text{tot}=\sum_i\dot{\bm p}_i = \sum_i m_i\ddot{\bm r}_i=M\ddot{\bm r}_\text c=\bm 0
\end{equation*}
因此这些质点的总动量为定值,这推广了在二体问题中得到的结论,概述为:
\begin{theorem}[动量守恒定律-v1.0]\label{mmntm_cnsrv_1.0}
    如果一些质点所受的力完全来自于它们之间的相互作用,且这些相互作用服从牛顿第三定律,则它们在按照牛顿第二定律运动的过程中,总动量守恒。
\end{theorem}

质点系的势能函数应 $V(\bm r_1,\cdots,\bm r_N)$ 使得所有质点的受力 $\bm F_i$ 都满足 $\bm F_i = -\nabla_{\bm r_i}V$。因为能量守恒定律可以作推广,其证明与单质点的情况并无本质差异:
\begin{theorem}[能量守恒定律-v1.1]\label{thm:enrgy_cnvrs}
    如果一个质点系中各质点的受力 $\bm F_i(\bm r)$ 只与这些质点的位置有关,则函数 $V(\bm r)$ 满足 $-\nabla_{\bm r_i} V=\bm F_i$ 当且仅当在任何符合牛顿第二定律的轨迹上能量函数
    \begin{equation*}
        E(\bm r,\dot{\bm r})=\frac{1}{2}\bm r^\T \mbf M\bm r+V(\bm r)
    \end{equation*}
    守恒。
\end{theorem}

结合能量观点回顾定理 \ref{mmntm_cnsrv_1.0} 的前提条件:假设质点间力 $\bm F_{ij}(\bm r_{ij})$ 都是保守的,那么这些力可以分别看做来自于质点间势能函数 $V_{ij}(\bm r_{ij})$,满足 $-\nabla V_{ij}=\bm F_{ij}$。于是容易验证,以下势能函数满足定理~\ref{thm:enrgy_cnvrs} 的要求:
\begin{equation*}
    V(\bm r)=\sum_{i,j}V_{ij}(\bm r_{ij})=\sum_{i,j}V_{ij}(\bm r_i-\bm r_j)
\end{equation*}
由于其完全以相对位置 $\bm r_{ij}$ 为自变量,所以可以考虑平移变换 $\mc T(\bm d)$,它的效果为所有位置平移 $\bm d$,即
\begin{equation*}
    \mc T(\bm d)\bm r=(\bm r_1+\bm d,\cdots,\bm r_N+\bm d)
\end{equation*}
变换后的位置代入势能函数得到
\begin{equation*}
    V(\mc T(\bm d)\bm r) = \sum_{i,j}V_{ij}((\bm r_i+\bm d)-(\bm r_j+\bm d)) = \sum_{i,j}V_{ij}(\bm r_i-\bm r_j) = V(\bm r)
\end{equation*}
可见位置经过平移后,势能函数保持不变。

如果势能函数中的诸自变量经过某些变换,势能函数本身却没有发生变化,则称势能函数在这一变换下具有对称性。上述势能函数就具有平移对称性,这一对称性可以自然地导出动量守恒:
\begin{theorem}[动量守恒定律-v2.0]
    如果一个质点系的势能函数 $V(\bm r)$ 满足对于任意 $\bm d\in\R^3$,都有
    \begin{equation*}
        V(\mc T(\bm d)\bm r)=V(\bm r)
    \end{equation*}
    则这个质点系的任意满足牛顿运动方程的轨迹上都满足动量守恒 $\dot{\bm p}=0$。
\end{theorem}
\begin{proof}
    首先取 $x$ 方向单位向量 $\bm e_x$,对于任意长度 $l$ 都有 $V(\bm r_1+l\bm e_x,\cdots,\bm r_N+l\bm e_x)=V(\bm r_1,\cdots,\bm r_N)$,因此有
    \begin{equation*}
        \frac{\d V(\mc T(l\bm e_x)\bm r)}{\d l}=\sum_{i}\nabla_{\bm r_i}V\cdot\bm e_x=-\bm F_i\cdot\bm e_x=F_x=0
    \end{equation*}
    其中 $F_x$ 是合力 $\bm F=\sum_i\bm F_i$ 的 $x$ 分量,同理可证其他两个分量 $F_y=F_z=0$,于是 $\bm F=\bm 0$。于是
    \begin{equation*}
        \sum_i{\dot{\bm p}_i}=\sum_i{m\ddot{\bm r}_i}=\sum_i\bm F_i=\bm F=\bm 0
    \end{equation*}
\end{proof}

与力的观点相比,由势能对称性得到守恒量的过程展现出了能量观点更强大的分析能力,它远不止是将向量函数化为标量函数这样简单。因此,我们还希望在更丰富的对称性上重复这一过程以简化问题。

\subsubsection{有心力问题中角动量守恒}
在约化后的氢原子问题中,我们通过~\ref{ins:Clmb_ptntl} 中的推导得到了势能函数 $V(\bm r)=-\frac{k}{r}$。仔细观察会发现,它写作了 $V(r)$ 而非 $V(\bm r)$ 的形式,也即只需位置到原点的距离即可描述。形如此类的势能函数又称为有心势,其动力学问题又称为\textbf{有心力(central force)问题}。这一名字来自于以下性质:
\begin{equation*}
    \bm F(r)=-\nabla V(r)=-V'(r)\frac{\bm r}{r}=F(r)\frac{\bm r}{r}
\end{equation*}
即力的方向所在直线总是穿过位于原点的“力心”。由于绕原点的旋转只改变 $\bm r$ 的方向而不改变 $r$ 的大小,因此有心力问题具有(绕原点的)旋转对称性。仿照平移对称性的思路,可作以下研究:

考虑绕 $z$ 轴逆时针旋转 $\theta$ 角的变换 $\mc R_z(\theta)$,它是绕原点旋转之一例。这个变换不会影响 $z$ 坐标,因此我们只考虑 $xy$ 平面上的向量 $\bm r=(x,y)$,变换后得到
\begin{equation*}
    \bm r':=\mc R_z(\theta)\bm r=(x\cos\theta-y\sin\theta, x\sin\theta+y\cos\theta):=(x',y')
\end{equation*}
具有旋转对称性的势能函数满足 $V(\mc R_z(\theta)\bm r)=V(\bm r)$,于是
\begin{align*}
    \frac{\d V(\mc R_z(\theta)\bm r)}{\d\theta} &= \frac{\p V}{\p x'}\frac{\p x'}{\p\theta}+\frac{\p V}{\p y'}\frac{\p y'}{\p\theta} \\
    &= \left(\frac{\p V}{\p x}\frac{\p x}{\p x'}+\frac{\p V}{\p y}\frac{\p y}{\p x'}\right)(-x\sin\theta-y\cos\theta)+\left(\frac{\p V}{\p x}\frac{\p x}{\p y'}+\frac{\p V}{\p y}\frac{\p y}{\p y'}\right)(x\cos\theta-y\sin\theta) \\
    &= \left(\frac{\p V}{\p x}\cos\theta-\frac{\p V}{\p y}\sin\theta\right)(-x\sin\theta-y\cos\theta)+\left(\frac{\p V}{\p x}\sin\theta+\frac{\p V}{\p y}\cos\theta\right)(x\cos\theta-y\sin\theta) \\
    & = -y\frac{\p V}{\p x}+x\frac{\p V}{\p y}\\
    & = m(-x\ddot{y}+y\ddot{x})\\
    & = m(-x\ddot{y}+y\ddot{x}+\dot{x}\dot{y}-\dot{x}\dot{y})\\
    & = -m\frac{\d}{\d t}(x\dot y-y\dot x)=-\frac{\d}{\d t}(xp_y-yp_x)=0
\end{align*}
其中第三个等号可以将 $\bm r$ 视作 $\bm r'$ 顺时针旋转 $\theta$ 得到,即 $\bm r=\mc R(-\theta)\bm r'$,从而求出 $\bm r$ 诸分量对 $\bm r'$ 诸分量的偏导数。$p_x=m\dot{x},p_y=m\dot{y}$ 分别为动量 $\bm p$ 的两个坐标分量。

这一番推导表明 $xp_y-yp_x$ 是运动中的守恒量。类似分析绕 $x$ 轴和 $y$ 轴的旋转,还可以得到守恒量 $yp_z-zp_y$ 和 $zp_x-xp_z$。此时回到三维位置 $\bm r=(x,y,z)$ 和动量 $\bm p=(p_x,p_y,p_z)$,可以把绕 $x,y,z$ 轴的旋转得到的三个守恒量也整合为一个三维向量:
\begin{equation*}
    \bm L=(yp_z-zp_y,zp_x-xp_z,xp_y-yp_x)=\bm r\times\bm p
\end{equation*}
$\bm L$ 称为质点的角动量(angular momentum),于是我们得到以下结论:
\begin{theorem}[角动量守恒定律-v1]
    如果一个质点的势能函数 $V(\bm r)$ 满足对于任意旋转变换 $\mc R$,都有
    \begin{equation*}
        V(\mc R\bm r)=V(\bm r)
    \end{equation*}
    则这个质点的任意满足牛顿运动方程的轨迹上都满足角动量守恒 $\dot{\bm L}=0$。
\end{theorem}

从 $\bm L=\bm r\times\bm p$ 可知 $\bm r,\bm p$ 始终垂直于 $\bm L$:因此在角动量守恒时可得以下推论:
\begin{itemize}
    \item 如果 $\bm L=\bm 0$,则 $\bm r,\bm p$ 始终同向,于是质点在一条直线上运动。
    \item 如果角动量守恒且非零,则建立坐标系时不妨将 $\bm L$ 设立为 $z$ 轴正方向,此时 $\bm p$ 将永远保持 $p_z=0$;因此若让 $xy$ 平面穿过 $\bm r$ 的初值,那么将永远保持 $z=0$,即质点在一个平面上运动。
\end{itemize}
因此,有心势中质点总是在与角动量垂直的固定平面上运动。假设坐标系的 $xy$ 平面就架设在这个平面上,质点的位置可以简化为二维坐标 $(x,y)$;而由于有心力只关心 $r=\sqrt{x^2+y^2}$,所以不妨用极坐标 $(r,\theta)$ 表示质点的位置,在 $\theta\in\left(-\frac{\pi}{2},\frac{\pi}{2}\right)$ 时 $\theta = \arctan\frac{y}{x}$,于是
\begin{gather*}
    \frac{\p\theta}{\p x}=-\frac{y}{x^2+y^2},\quad \frac{\p\theta}{\p y}=-\frac{x}{x^2+y^2}\\
    \frac{\d\theta}{\d t} = \dot{x}\frac{\p\theta}{\p x} + \dot{y}\frac{\p\theta}{\p y}=\frac{x\dot y-y\dot x}{x^2+y^2}
\end{gather*}
当 $\theta$ 位于其他角度区间时也可通过其他反三角函数得到,于是
\begin{equation}\label{eq:anglr_vlcty}
    L=m(x\dot y - y\dot x)=m(x^2+y^2)\frac{\d\theta}{\d t}=mr^2\dot\theta=mr^2\omega
\end{equation}
角速度(angular velocity)矢量 $\bm \omega$ 的大小定义为 $\dot{\theta}$,方向定义为与 $\bm L$ 相同。角动量守恒时 $L$ 为定值,在这个平面上的 $r(t)$ 和 $\theta(t)$ 的关系也随之确定,解出其中一个便可由方程~\ref{eq:anglr_vlcty} 解出另一个,并通过 $x=r\cos\theta,y=r\sin\theta$ 复原为直角坐标下的轨迹。

至此,由平移对称性得到的动量守恒帮助我们将二体问题约化为单体问题;由旋转对称性得到的角动量守恒帮助我们将三维有心力问题约化为二维问题。
然而,并非所有对称性都容易从势能函数的数学形式直接观察得到。接下来我们将跳过潜藏极深的对称性引入一个极为特殊的守恒量,并利用它彻底解决静电势的动力学问题。

\subsubsection{满足平方反比律的有心力导出 LRL 向量守恒}
在有心力问题中尝试求以下时间导数:
\begin{align*}
    \frac{\d}{\d t}(\bm p\times\bm L) & = \dot{\bm p}\times\bm L + \bm p\times\dot{\bm L}\\
    & = F(r)\frac{\bm r}{r}\times(\bm r\times\bm p)\\
    & = \frac{F(r)}{r}[(\bm r\cdot\bm p)\bm r-r^2\bm p)]\\
    & = \frac{mF(r)}{r}\left(\frac 1 2\frac{\d(r^2)}{\d t}\bm r-r^2\dot{\bm r}\right)\\
    & = \frac{mF(r)}{r}\left(r\dot{r}\bm r-r^2\dot{\bm r}\right)\\
    & = mF(r)r^2\left(\dot r\frac{\bm r}{r^2}-\frac{\dot{\bm r}}{r}\right)\\
    & = -mF(r)r^2\frac{\d}{\d t}\left(\frac{\bm r}{r}\right)
\end{align*}
其中第三个等号用到了向量恒等式 $\bm a\times(\bm b\times\bm c)=\bm b(\bm a\cdot\bm c)-\bm c(\bm a\cdot\bm b)$。
对于库仑力的情况, $F(r) = -\frac{k}{r^2}$,有:
\begin{equation}\label{eq:invrs_sqr}
    \frac{\d}{\d t}(\bm p\times\bm L)=-mk\frac{\d}{\d t}\left(\frac{\bm r}{r}\right)\implies \frac{\d}{\d t}\left(\bm p\times\bm L - mk\frac{\bm r}{r}\right)=\bm 0
\end{equation}
可以定义无量纲的\textbf{拉普拉斯-隆格-楞次向量(Laplace-Runge-Lenz vector)},简称 LRL 向量:
\begin{equation*}
    \bm A = \frac{\bm p\times\bm L}{mk}-\frac{\bm r}{r}
\end{equation*}
它是静电势下质点运动的守恒量。从以上推导过程可知,得到这一守恒量的关键条件在于式~\ref{eq:invrs_sqr} 中利用的 $F(r)$ 平方反比律,这是有心力中极为特殊的一种,我们总结为以下结论:
\begin{theorem}[LRL 向量守恒定律-v1]\label{thm:LRL_vctr_cnvrs}
    如果一个质点的势能函数形如 $V(\bm r)=-\frac{k}{r}$,则它的任意满足牛顿运动方程的轨迹上都满足 LRL 向量守恒 $\dot{\bm A}=\bm 0$。
\end{theorem}

结合以上所得的守恒量,我们终于可以开始求解氢原子的动力学问题:首先计算
\begin{equation*}
    \bm A\cdot\bm r=\frac{\bm r\cdot(\bm p\times\bm L)}{mk^2}-\frac{r^2}{r}=\frac{\bm L\cdot(\bm r\times\bm p)}{mk^2}-r=\frac{L^2}{mk^2}-r
\end{equation*}
如果让 $z$ 轴与守恒的 $\bm L$ 同向,$x$ 轴与守恒的 $\bm A$ 同向,轨迹所在平面为 $xy$ 平面,有极坐标轨迹 $(r(t),\theta(t))$,则 $\bm A\cdot\bm r=Ar\cos\theta$,结合上式得到轨迹方程
\begin{equation}\label{eq:cnc_crv}
    r(t)=\frac{L^2}{mk(1+A\cos\theta(t))}
\end{equation}
与方程~\ref{eq:anglr_vlcty} 联立,结合初始条件算出 $L,A$ 即可求解。而在求解之前,$r(\theta)$ 就已经显露出圆锥曲线的模样,其中 $A$ 即为离心率:
\begin{table}[!htbp]
    \centering
    \caption{LRL 矢量大小 $A$ 对轨迹形状的影响}
    \begin{tabular}{c|cccc}
    \toprule
    离心率 & $A=0$    & $0<A<1$ & $A=1$ & $A>1$\\
    \midrule
    轨迹曲线 &圆    & 椭圆 & 抛物线 & 双曲线\\
    \bottomrule
    \end{tabular}
    \label{tab:LRL_vctr_trjctry}
\end{table}

\srule

\begin{instance}[圆轨道氢原子的求解]
在氢原子的例子中,$A\ge 1$ 时电子与核的相对位置 $r$ 具有开放的轨迹,这意味着电子将在短暂地靠近核后一去不复返,这显然有悖于我们对稳定的氢原子的认识。$0\le A<1$ 时轨道是闭合的,由于椭圆轨道有固定的长短轴取向,对于各 $\theta$ 并非各向同性,我们不妨选用 $A=0$ 圆轨道解消这一方向来理解氢原子:
\begin{enumerate}
    \item 运动模式:
    \begin{itemize}
        \item 方程~\ref{eq:cnc_crv} 简化为 $r(t)\equiv\frac{L^2}{mk}$,说明 $r$ 为定值,与圆轨道吻合。
        \item 方程~\ref{eq:anglr_vlcty} 简化为 $L=mr^2\omega(t)$,说明 $\omega=\dot{\theta}$ 为定值,又因为圆轨迹上 $\bm r,\bm p$ 垂直故 $L=rp=mvr$,所以速度 $v=\omega r$ 也为定值,$\bm r$ 做匀速圆周运动。
    \end{itemize}
    \item 初值应满足的条件:
\begin{equation}\label{eq:unfrm_crclr}
    r=\frac{L^2}{mk}=\frac{m^2v^2r^2}{mk}\implies mv^2=-\frac{k}{r}
\end{equation}
故轨迹的初位置对应的 $r(0)=r_0$,初速度 $\dot{r}(0)=v_0$ 就需要满足 $mv_0^2=-\frac{k}{r_0}$
    \item 能量:
    由于 $r,v$ 都为定值,所以动能 $\frac{1}{2}mv^2$ 和静电势能 $-\frac{k}{r}$ 都守恒,始终等于由初值条件解算的值 $\frac{1}{2}mv_0^2$ 和 $-\frac{k}{r_0}$,且根据初值方程~\ref{eq:unfrm_crclr},负势能是动能的二倍,总能量为
    \begin{equation*}
        E=\frac{1}{2}mv_0^2-\frac{k}{r_0}=-\frac{k}{2r_0}=-\frac{1}{2}mv_0^2
    \end{equation*}
\end{enumerate}
\end{instance}
\srule

% 然而这样的物理图像仍然有严重的缺陷:经典电磁学给出了静电势的形式作为以上推导的根基,也预言了变速运动的电荷将辐射电磁波。这一原理促成了让电子高速旋转以充当高亮度光源的同步辐射技术,但同时宣告了绕转质子的电子将因为辐射不断损失能量并最终坠入原子核。因此,用经典力学和电磁学解释氢原子只能是一种美好幻想,更不必提更复杂的化学体系。为了解释物质世界的稳定存在,我们不得不寻求更精致的理论的帮助。

\subsection{电动力学的初步框架}

\subsubsection{三个尺度与密度视角}

上一节中反复使用的质点和点电荷都属于点模型,仿佛体积、形状和结构可以忽略不计的微观\textbf{粒子(particle)}。在研究宏观的物体时,如果体积、形状和结构确实不重要,那么点模型也可以作为一种合理的近似。但化学作为一门兼顾微观和宏观的学问,仅仅屈就于这种妥协是不够的。我们可以将物质分为以下三个研究尺度。
\begin{itemize}
    \item 在微观尺度,谈粒子不谈体积,每一个粒子可以具有质量 $m$ 和电荷 $q$。
    \item 在介观尺度,我们在一个物体所占据的空间区域中划分 $\lambda$-网格:考虑点阵 $\{(\lambda l,\lambda s,\lambda t)\mid l,s,t\in\Z\}$,其中每个点 $\bm r_{lst}=(\lambda l,\lambda s,\lambda t)$ 都位于区域 $D_{lst}(\lambda)=\{(x,y,z)\mid \lambda l\le x<\lambda(l+1), \lambda s\le y<\lambda(s+1), \lambda t\le z<\lambda(t+1)\}$。其体积 $V(D_{lst}(\lambda))\equiv\lambda^3$  与粒子的典型尺度相比仍然很大,所以每个区域中都有数量极多的 $N_{lst}(\lambda)$ 个粒子,我们定义\textbf{粒子数密度(particle number density)}为单位体积内的粒子数,于是 $D_{lst}(\lambda)$ 中的粒子数密度是 $\rho_{\text n,lst}=N_{lst}(\lambda)/\lambda^3$;但 $\lambda$ 与物体的宏观尺寸相比又很小,$D_{lst}(\lambda)$ 足够小以至于我们可以认为以下粒子数密度函数描述了物体每个点周围的性质。
    \begin{equation*}
        \rho_{\text n}(\bm r) = \rho_{\text n,lst}, \text{如果} \bm r\in D_{lst}(\lambda)
    \end{equation*}
    同理,$D_{lst}(\lambda)$ 的总质量 $m_{lst}(\lambda)$ 由其中每个粒子的质量所贡献,其中的\textbf{质量密度(mass density)}是 $\rho_{\text m,lst}=m_{lst}(\lambda)/\lambda^3$,于是有质量密度函数
    \begin{equation*}
        \rho_{\text m}(\bm r) = \rho_{\text m,lst}, \text{如果} \bm r\in D_{lst}(\lambda)
    \end{equation*}
    $D_{lst}(\lambda)$ 的总电荷量 $m_{lst}(\lambda)$ 由其中每个粒子的质量所贡献,其中的\textbf{电荷密度(charge density)}是 $\rho_{\text q,lst}=m_{lst}(\lambda)/\lambda^3$,于是有电荷密度函数
    \begin{equation*}
        \rho_{\text q}(\bm r) = \rho_{\text q,lst}, \text{如果} \bm r\in D_{lst}(\lambda)
    \end{equation*}
    
    使用密度函数,就可以刻画物体中粒子、质量、电荷等物理量分布的不均匀性。如果某点处的密度大,说明粒子在此处贡献的此种物理量较为集中。
    \item 在宏观尺度,由 $N$ 个粒子构成的物体占据了体积为 $V$ 的空间区域,它具有质量 $M$ 和电荷 $Q$。以粒子数为例,总粒子数 $N$ 是所有区域 $D_{lst}$ 中粒子数的总和,即
    \begin{equation*}
        N = \sum_{l,s,t} N_{lst} = \sum_{l,s,t} \rho_{\text n,lst}V(D_{lst})
    \end{equation*}
    只要 $\lambda$ 足够小,就可以将对网格的求和视作对全空间积分
    \begin{equation*}
        N = \int_{\R^3}\text d \Omega\,\rho_{\text n}(\bm r)
    \end{equation*}
    对于质量和电荷是类似的:分别有
    \begin{equation*}
        M = \int_{\R^3}\text d \Omega\,\rho_{\text m}(\bm r),\quad Q = \int_{\R^3}\text d \Omega\,\rho_{\text q}(\bm r)
    \end{equation*}
    因此密度函数本身也携带了物体作为总体的信息。
\end{itemize}

密度函数甚至可以覆盖点模型:考虑一个位于原点的点电荷 $q_0$ 的电荷密度,为方便起见记为 $q_0\delta(\bm r)$。根据以上论证,它具有以下两个性质:
\begin{itemize}
    \item 由于除了位于原点的点电荷外没有其他电荷,所以只有包含原点的区域中电荷密度才非零。对于其他点,自然可以(通过给 $\lambda$-网格加细)找到一个不包含原点的网格区域覆盖它,这个区域内的电荷密度为 0。这相当于只要 $\bm r\ne 0$ 就有 $q_0\delta(\bm r)=0$
    \item 电荷密度在全空间积分等于总电荷量,对于点电荷即
    \begin{equation*}
        \int\d \Omega\,q_0\delta(\bm r)=q_0\implies \int\d V\,\delta(\bm r)=1
    \end{equation*}
\end{itemize}

这意味着对于任何在原点附近足够光滑的函数 $f(\bm r)$,计算积分 $\int\d\Omega\,f(\bm r)\delta(\bm r)$ 时,扣除原点附近的一个任意小区域之外,都有 $\delta(\bm r)=0$,因此这一部分的积分贡献为 0;而在原点附近足够小的区域,可以认为该区域上函数的值变化不大,用定值 $f(\bm 0)$ 代替,于是这个积分实际上相当于提取出了 $f(\bm r)$ 在原点处的取值。
\begin{equation*}
    \int\d\Omega\,f(\bm r)\delta(\bm r) = \int\d\Omega\,f(\bm 0)\delta(\bm r) = f(\bm 0)\int\d\Omega\,\delta(\bm r) = f(\bm 0)
\end{equation*}

我们无法写出 $\delta(\bm r)$ 的常规表达式。事实上它并非一般意义下的函数,而只能在 $\int\d\Omega\,f(\bm r)\delta(\bm r)$ 这样的积分意义下理解。这类特殊函数称为\textbf{狄拉克(Dirac)函数}。

借此机会,可以考察一个电荷分布产生的静电力和点电荷产生的静电力的联系:

\srule

\begin{instance}[电荷分布产生的静电力]
    考虑一个电荷分布 $\rho(\bm r)$ 对一个位于 $\bm r$ 的点电荷 $q$ 的作用力。根据以上分析,电荷分布实际上可以看作来源于各区域 $D_i$ 的总电荷 $q_i$ 产生的作用力之合力,这些区域足够小,以至于与电荷分布相比可以看作位于 $\bm r_i\in D_i$ 的点电荷 $q_i$,只要 $\bm r_i\in D_i$,就有 $\frac{\bm r-\bm r'}{|\bm r-\bm r'|^3}$ 变化不大。代入库仑定律得到
    \begin{equation*}
        \bm F(\bm r) = \sum_i \frac{1}{4\pi\epsilon_0}\frac{qq_i\bm (\bm r-\bm r_i)}{|\bm r-\bm r_i|^3} 
    \end{equation*}
    将小区域上的求和改为积分,就有
    \begin{equation}\label{eq:chrg_dstrbtn_frc}
        \bm F(\bm r)= \frac{1}{4\pi\epsilon_0}\int\d^3 r'\frac{q\rho(\bm r')\bm (\bm r-\bm r')}{|\bm r-\bm r'|^3}
    \end{equation}
    将满足以上性质的密度代入~\ref{eq:chrg_dstrbtn_frc} 会发现位于 $\bm r$ 的点电荷 $q$ 受到的作用力为 
\begin{equation*}
    q_0f(\bm 0)=\frac{1}{4\pi\epsilon_0}\frac{qq_0\bm r}{r^3}
\end{equation*}
与库仑定律的结论完全吻合,密度视角回归了点模型中的物理结论。
\end{instance}

\srule

以上的过渡允许我们自然地把点模型中的物理规律迁移到密度函数:如果带有参数 $w$ (电荷、质量等)、位于 $\bm r$ 的点模型的某个物理量需要用 $wf(\bm r)$ 来计算,则 $w$ 对应的密度分布 $\rho_w(\bm r)$ 在此物理量上产生的总贡献为
\begin{equation}\label{eq:dnsty_cntrbtn}
    \int\d \Omega f(\bm r)\rho_w(\bm r)
\end{equation}
在上例中 $w$ 即电荷 $q$,$f(\bm r)=\frac{1}{4\pi\epsilon_0}\frac{q(\bm r-\bm r')}{|\bm r-\bm r'|^3}$。接下来我们将从这一视角研究电荷分布的相互作用。

\subsubsection{静电场}

库仑定律指出:一个位于 $\bm r'$ 的\textbf{固定}电荷 $q_0$ 在全空间任意一点 $\bm r$ 处都会对试探电荷 $q$ 产生静电力
\begin{equation*}
    \bm F=\frac{1}{4\pi\epsilon_0}\frac{qq_0(\bm r-\bm r')}{|\bm r-\bm r'|^3}
\end{equation*}
从试探电荷的视角来看,仿佛这个固定电荷产生了一个\textbf{电场(electric field)} $\bm E$,使得位于电场中一点 $\bm r$ 的点电荷 $q$ 受到电场的相互作用为 $\bm F(\bm r)=q\bm E(\bm r)$,此处
\begin{equation*}
    \bm E(\bm r)=\frac{1}{4\pi\epsilon_0}\frac{q_0(\bm r-\bm r')}{|\bm r-\bm r'|^3}
\end{equation*}
对于一般情况,对于一个固定电荷分布 $\rho(\bm r)$,根据前面讨论,在~\ref{eq:dnsty_cntrbtn} 中代入 $f(\bm r) = \frac{1}{4\pi\epsilon_0}\frac{(\bm r-\bm r')}{|\bm r-\bm r'|^3}$,可得其产生的电场为
\begin{equation*}
    \bm E(\bm r)=\frac{1}{4\pi\epsilon_0}\int\d^3 r'\frac{\rho(\bm r')(\bm r-\bm r')}{|\bm r-\bm r'|^3}
\end{equation*}

用矢量分析方法来处理这个电场,能够得到两个重要结论。
\begin{theorem}[电场的高斯(Gauss)定律]\label{thm:elctrc_fld_gss}
    电荷分布 $\rho(\bm r)$ 按库仑定律产生的电场 $\bm E(\bm r)$ 满足
    \begin{equation*}
        \nabla\cdot\bm E(\bm r)=\frac{\rho(\bm r)}{\epsilon_0}
    \end{equation*}
\end{theorem}
\begin{proof}
    \begin{equation}\label{eq:divE}
        \nabla\cdot\bm E(\bm r) = \frac{1}{4\pi\epsilon_0}\int\d^3 r'\rho(\bm r')\nabla\cdot\frac{\bm r-\bm r'}{|\bm r-\bm r'|^3}
    \end{equation}
    此处的关键问题在于求 $\nabla\cdot\frac{\bm r-\bm r'}{|\bm r-\bm r'|^3}$ 的表达式。不妨首先考虑 $\bm r'=\bm 0$ 的情况,此时 $\frac{\bm r}{r^3}$ 是一个仅有径向分量的函数,适合使用球坐标而非直角坐标来研究。
    我们知道球坐标中的点 $(r,\theta,\phi)$ 和直角坐标中的点 $(x,y,z)$ 的变换关系:
    \begin{equation}\label{eq:crtsn2sphrcl}
        \begin{cases}
            x=r\cos\theta\sin\phi\\
            y=r\sin\theta\sin\phi\\
            z=r\cos\phi
        \end{cases}
    \end{equation}
    且球坐标中的矢量场可以写成直角坐标系下的分量形式
    \begin{equation*}
        \bm f(x,y,z)=f_x(x,y,z)\bm e_x+f_y(x,y,z)\bm e_y+f_z(x,y,z)\bm e_z
    \end{equation*}
    其中 $\bm e_x,\bm e_y,\bm e_z$ 分别是 $x,y,z$ 方向的单位矢量。但是在球坐标中,我们希望不借助这些来自于直角坐标的单位矢量来描述矢量场,而是希望在每个点 $(r,\theta,\phi)$ 处建立一组新的、互相垂直的单位矢量,分别是指向 $r$ 增大方向的 $\bm e_r(r,\theta,\phi)$,指向 $\theta$ 增大方向 $\bm e_\theta(r,\theta,\phi)$ 和指向 $\phi$ 增大方向的 $\bm e_\phi(r,\theta,\phi)$。该点的矢量场按这三个矢量正交分解为
    \begin{equation*}
        \begin{cases}
            f_r(r,\theta,\phi) = \bm f(r,\theta,\phi)\cdot\bm e_r\\
            f_\theta(r,\theta,\phi) = \bm f(r,\theta,\phi)\cdot\bm e_\theta\\
            f_\phi(r,\theta,\phi) = \bm f(r,\theta,\phi)\cdot\bm e_\phi
        \end{cases}
    \end{equation*}
    几何上可以通过简单的二维旋转从 $\bm e_x,\bm e_y,\bm e_z$ 得到 $\bm e_r,\bm e_\theta,\bm e_\phi$。
    \begin{itemize}
        \item $\bm e_\phi$ 由 $\bm e_y$ 绕 $z$ 轴方向(在 $xy$ 平面上)旋转 $\phi$ 得到。
        \begin{equation*}
            \bm e_\phi = -\sin\phi\bm e_x + \cos\phi\bm e_y
        \end{equation*}
        \item $\bm e_r$ 由 $\bm e_x$ 绕 $y$ 轴方向(在 $zx$ 平面上)旋转 $\theta-\pi/2$,得到 $\sin\theta\bm e_x+\cos\theta\bm e_z$。再绕最初的 $z$ 轴方向(在最初的 $xy$ 平面上)旋转得到
        \begin{equation*}
            \bm e_r = \sin\theta\cos\phi\bm e_x+\sin\theta\sin\phi\bm e_y+\cos\theta\bm e_z
        \end{equation*}
        \item $\bm e_\theta$ 由 $\bm e_x$ 绕 $y$ 轴方向(在 $zx$ 平面上)旋转 $\theta$,得到 $\cos\theta\bm e_x-\sin\theta\bm e_z$。再绕最初的 $z$ 轴方向(在最初的 $xy$ 平面上)旋转得到
        \begin{equation*}
            \bm e_\theta = \cos\theta\cos\phi\bm e_x+\cos\theta\sin\phi\bm e_y-\sin\theta\bm e_z
        \end{equation*}
    \end{itemize}
    厘清这一关系之后,我们知道 $\nabla\cdot\bm f=\frac{\p f_x}{\p x}+\frac{\p f_y}{\p y}+\frac{\p f_z}{\p z}$,以其中一项为例
    \begin{equation*}
        \frac{\p f_x}{\p x} = \frac{\p(\bm f\cdot\bm e_x)}{\p x}= \frac{\p}{\p x}\left(f_r\bm e_r\cdot\bm e_x + f_\theta\bm e_\theta\cdot\bm e_x+f_\phi\bm e_\phi\cdot\bm e_x\right)
    \end{equation*}
    根据以上三式可以接下来得到
    \begin{equation*}
        \frac{\p f_x}{\p x} = \frac{\p }{\p x}(f_r\sin\theta\cos\phi+f_\theta\cos\theta\cos\phi -f_\phi\sin\phi)
    \end{equation*}
    等式右边的括号中只剩下了 $r,\theta,\phi$ 的函数,再利用以下变换
    \begin{equation*}
        \frac{\p}{\p x} f(r,\theta,\phi) = \frac{\p r}{\p x}\frac{\p f}{\p r} + \frac{\p \theta}{\p x}\frac{\p f}{\p\theta}+\frac{\p\phi}{\p x}\frac{\p f}{\p\phi}
    \end{equation*}
    其中的 $\frac{\p r}{\p x},\frac{\p \theta}{\p x},\frac{\p\phi}{\p x}$ 可以通过 \ref{eq:crtsn2sphrcl}。如此推演就可计算出完全用 $r,\theta,\phi$ 表达的 $\nabla\cdot\bm f$:
    \begin{equation*}
        \nabla\cdot\bm f=\frac{1}{r^2}\frac{\p}{\p r}(r^2f_r)+\frac{1}{r\sin\theta}\frac{\p}{\p\theta}(\sin\theta f_\theta)\frac{1}{r\sin\theta}\frac{\p f_\phi}{\p\phi}
    \end{equation*}
    令上式中的 $\bm f(\bm r) = \frac{\bm r}{r^3}$,于是只要 $r\ne 0$,就有
    \begin{equation}\label{eq:dvrn3}
        \nabla\cdot\frac{\bm r}{r^3} = \frac{1}{r^2}\frac{\p}{\p r}\left(r^2\frac{1}{r^2}\right) = \frac{1}{r^2}\cdot 0 = 0
    \end{equation}
    而考虑任何一个包含原点的区域 $\Omega$,在这个区域上对 $\nabla\cdot\frac{\bm r}{r^3}$ 积分,根据高斯散度定理,有
    \begin{equation*}
        \int_\Omega\d\Omega\, \nabla\cdot\frac{\bm r}{r^3} = \oint_{\p\Omega}\d\phi\d\theta\frac{\bm r}{r^3}\cdot\bm e_r r^2\sin\theta
    \end{equation*}
    根据 \ref{eq:dvrn3},这个积分的区域只要包含原点就与边界无关。不妨取 $\p\Omega$ 为以原点为球心的单位球面,于是积分的值为 $4\pi$。与上一章对点电荷密度的分析对照,$\nabla\cdot\frac{\bm r}{r^3}$ 在任何不包含原点的区域上的积分为 0,在任何包含原点的区域上积分为 $4\pi$,于是
    \begin{equation}\label{eq:dvrgnc_sqr_invrs}
        \nabla\cdot\frac{\bm r}{r^3} = 4\pi\delta(\bm r)
    \end{equation}
    代入~\ref{eq:divE} 得到
    \begin{equation}
        \nabla\cdot\bm E(\bm r) = \frac{1}{4\pi\epsilon_0}\int\d^3 r'\, 4\pi\delta(\bm r-\bm r')\rho(\bm r') = \frac{\rho(\bm r')}{\epsilon_0}
    \end{equation}
\end{proof}

\begin{theorem}[恒定电场的无旋定律]\label{thm:elctrc_fld_crl}
    固定电荷分布 $\rho(\bm r)$ 按库仑定律产生的电场 $\bm E(\bm r)$ 满足
    \begin{equation*}
        \nabla\times\bm E(\bm r)=\bm 0
    \end{equation*}
\end{theorem}
\begin{proof}
    根据例~\ref{ins:Clmb_ptntl} 的推导,静电力是保守力,$\nabla\times\bm F(\bm r)=\bm 0\implies\nabla\times\frac{\bm r}{r^3}=0$。在微分式中将 $\bm r$ 无损地代换为 $\bm r-\bm r'$,代入积分式得到
    \begin{equation}
        \nabla\times\bm E(\bm r) = \frac{q_0}{4\pi\epsilon_0}\int\d^3r'\,\left(\nabla\times\frac{\bm r-\bm r'}{|\bm r-\bm r'|^3}\right) = \bm 0
    \end{equation}
\end{proof}

仿照定理~\ref{thm:cndtn_cnsrvtv_frc},我们可以定义一个以位置为自变量的标量函数 $U(\bm r)$,称为\textbf{电势(electric potential)},它和电场的关系是
\begin{equation*}
    \int_{\bm r_0}^{\bm r_1}\bm E(\bm r)\cdot\bm{\d\bm r'} = U(\bm r_1)-U(\bm r_0),\quad \bm E(\bm r) = -\nabla U(\bm r)
\end{equation*}
这一积分与 $\bm r_0,\bm r_1$ 之间的路径无关。由于一个固定电荷分布 $\rho(\bm r)$ 产生的电场对试探电荷的作用力 $\bm F(\bm r)$ 与其电场强度 $\bm E(\bm r)$ 之间以试探电荷 $q$ 作为比例系数,所以,容易验证,这个固定电荷分布产生的电势 $U(\bm r)$ 和试探电荷在这一电场中获得的电势能 $V(\bm r)$ 之间也有类似的关系 $V(\bm r) = qU(\bm r)$。

\subsubsection{恒定电流与静磁场}
前面我们研究了固定电荷分布产生的电场及其对试探电荷的作用,接下来我们将初步研究运动的电荷分布。

在一根通电的导线中,携带电荷的电子源源不断地流过,定向迁移的电荷产生\textbf{电流(electric current)},电流的强度表示为单位时间通过某个截面的净电荷量。我们可以将导线视作数学意义上的线,截面从中截取一点,此处在短时间 $\d t$ 内通过少量电荷 $\d q$,于是该点处的电流强度(简称电流)为:
\begin{equation*}
    I = \frac{\d q}{\d t}
\end{equation*}
实验表明,导线中的电流产生了与自然界中的永磁体一样的\textbf{磁场(magnetic field)},它的强度和方向用\textbf{磁感应强度(magnetic induction intensity)} $\bm B$ 表示。当导线弯曲成曲线 $L'$ 时,其中每一点都有沿切线方向、大小为此处电流强度的矢量 $\bm I(\bm r')\,(\bm r '\in L')$,于是这条通电导线在 $\bm r$ 处产生的磁场 $\bm B(\bm r)$ 可以写作整条导线上的积分,称为\textbf{毕奥-萨伐尔(Biot-Savart)定律}:
    \begin{equation*}
        \bm B(\bm r) = \frac{\mu_0}{4\pi}\int_{L'}\frac{\bm I(\bm r')\times(\bm r-\bm r')}{|\bm r-\bm r'|^3}\text dl'
    \end{equation*}
其中 $\mu_0=4\pi\times 10^{-7}\,\si{N/A^2}$ 称为\textbf{真空磁导率(permeability of vacuum)}。这一磁场的效果乃使位置 $\bm r$ 处速度为 $\bm v$ 的运动电荷 $q$ 受到\textbf{洛伦兹(Lorentz)力},
\begin{equation}\label{eq:lrntz_frc}
    \bm F(\bm r)=q\bm v\times\bm B(\bm r)
\end{equation}
因此磁场和电场在试探电荷上展现出了两类不同的作用。

将运动的电荷扩展到运动的电荷密度时,假设短时间 $\d t$ 内某个小区域的电荷以速度 $\bm v$ 穿过了垂直于速度方向的、 $\bm r$ 附近面积为 $\d A$ 的小截面,于是 $\bm r$ 附近体积为 $v\d A\d t$ 的小长方体内的电荷量 $\rho(\bm r)v\d A\d t$ 穿过了这个小截面,截面上的电流强度 $\d I=\rho(\bm r)\bm v\d A$,于是沿用密度语言,称 $\bm r'$ 处的\textbf{电流密度(electric current density)}是单位截面积上的电流,电流密度矢量因此为 $\bm J(\bm r)=\rho(\bm r)\bm v$。对于任意运动的电荷密度,可以视作空间中每一处都有一根小导线,它带有 $\d A'$ 的截面和 $\d l'$ 的长度,承载着 $\bm J(\bm r')$ 的电流,于是毕奥-萨伐尔定律改写为
\begin{equation*}
    \bm B(\bm r) = \frac{\mu_0}{4\pi}\int\d A'\d l'\,\frac{\bm J(\bm r')\times(\bm r-\bm r')}{|\bm r-\bm r'|^3} = \frac{\mu_0}{4\pi}\int\d^3r'\,\frac{\bm J(\bm r')\times(\bm r-\bm r')}{|\bm r-\bm r'|^3}
\end{equation*}

用矢量分析方法来处理这个磁场,能够得到一个重要结论。

\begin{theorem}[磁场的无源定律]
    电流密度 $\bm J$ 按照毕奥-萨伐尔定律产生的磁场 $\bm B$ 满足
    \begin{equation*}
        \nabla\cdot\bm B=0
    \end{equation*}
\end{theorem}
\begin{proof}
    \begin{equation*}
        \nabla\cdot\int\d^3r'\,\frac{\bm J(\bm r')\times(\bm r-\bm r')}{|\bm r-\bm r'|^3} = \int\d^3r'\,\left[\nabla\cdot\frac{\bm J(\bm r')\times(\bm r-\bm r')}{|\bm r-\bm r'|^3}\right]
    \end{equation*}
    根据矢量分析的恒等式 $\nabla\cdot(\bm A\times\bm B)=\bm B\cdot(\nabla\times\bm A)-\bm A\cdot(\nabla\times\bm B)$,被积函数可以拆分成两项:
    \begin{equation*}
        \nabla\cdot\frac{\bm J(\bm r')\times(\bm r-\bm r')}{|\bm r-\bm r'|^3} = \frac{\bm r-\bm r'}{|\bm r-\bm r'|^3}\cdot(\nabla\times \bm J(\bm r'))-\bm J(\bm r')\cdot\left(\nabla\times\frac{\bm r-\bm r'}{|\bm r-\bm r'|^3}\right)
    \end{equation*}
    由于 $\bm J(\bm r')$ 不依赖于 $\bm r$,所以被对 $\bm r$ 的旋度算符 $\nabla\times$ 作用后得到 $\bm 0$;而示例~\ref{ins:Clmb_ptntl} 的计算说明 $\nabla\times\frac{(\bm r-\bm r')}{|\bm r-\bm r'|^3}=\bm 0$。
\end{proof}

根据电流密度的定义,由于电流密度 $\bm J(\bm r)$ 是 $\bm r$ 附近小截面 $\d A$ 的\textbf{垂直方向}上以速度 $\bm v$ 穿过截面的电荷密度 $\rho(\bm r)$ 产生的,所以对于任意形状的曲面 $\mc S$,穿过曲面的电流是其上每个小截面法向上的电荷密度的贡献,有以下第二类曲面积分的形式。
\begin{equation*}
    I = \int_{\mc S}\bm J\cdot\bm{\d A}
\end{equation*}
对于包围了区域 $\mc V$ 的闭合曲面,有
\begin{equation*}
    \oint_{\mc S}\bm J\cdot\bm{\d A}t = \int_{\mc V}\d V\,(\nabla\cdot\bm J)
\end{equation*}
而根据\textbf{电荷守恒定律(law of conservation of charge)},$\mc V$ 中电荷的减少量应等于其表面上电荷的流出量,即表面上电流对时间的积分。假设电流密度是时间和位置的函数 $\bm J(\bm r,t)$,则体现为以下恒等式:
\begin{equation*}
    \int_{t_0}^t\d t\,\int_{\mc V}\d V\,(\nabla\cdot\bm J(\bm r,t)) = -\int_{t_0}^t\d t\,\frac{\d}{\d t}\int_{\mc V}\d\bm r\,\rho(\bm r,t)=-\int_{t_0}^t\d t\,\int_{\mc V}\d\bm r\,\left(\frac{\p\rho(\bm r,t)}{\p t}\right)
\end{equation*}
由于体积 $\mc V$ 和时间区间 $[t_0,t]$ 的任意性,可以提炼出以下微分方程,称为\textbf{连续性方程(continuity equation)}
\begin{equation}\label{eq:cntnty}
    \nabla\cdot\bm J(\bm r,t) = -\frac{\p\rho(\bm r,t)}{\p t}
\end{equation}

如果 $\frac{\p J(\bm r,t)}{\p t} =\bm 0$,即各处的电流密度与时间无关,我们就称之为\textbf{恒定电流(steady current)},上述方程意味着恒定电流能够保持不随时间变化的电荷密度 $\rho(\bm r)$。在这种特殊情况下,我们能导出以下重要结论:

\begin{theorem}[磁场的安培(Ampere)定律 v1.0]
    恒定电流密度 $\bm J(\bm r)$ 按照毕奥-萨伐尔定律产生的磁场 $\bm B$ 满足
    \begin{equation*}
        \nabla\times\bm B(\bm r)=\mu_0\bm J(\bm r)
    \end{equation*}
\end{theorem}
\begin{proof}
    \begin{equation*}
        \nabla\times\int\d^3r'\,\frac{\bm J(\bm r')\times(\bm r-\bm r')}{|\bm r-\bm r'|^3} = \int\d^3r'\,\left[\nabla\times\frac{\bm J(\bm r')\times(\bm r-\bm r')}{|\bm r-\bm r'|^3}\right]
    \end{equation*}
    根据矢量分析的恒等式 $\nabla\times(\bm A\times\bm B)=\bm A(\nabla\cdot\bm B)-(\bm A\cdot\nabla)\bm B + (\bm B\cdot\nabla)\bm A - \bm B(\nabla\cdot\bm A)$,被积函数可以拆分成两项(仅包含前两项。后面的两项涉及到 $\bm r$ 的微分算符对 $\bm J(\bm r') $ 的作用,因而为零):
    \begin{equation*}
        \nabla\times\frac{\bm J(\bm r')\times(\bm r-\bm r')}{|\bm r-\bm r'|^3} = \bm J(\bm r')\left(\nabla\cdot\frac{\bm r-\bm r'}{|\bm r-\bm r'|^3}\right) - (\bm J(\bm r')\cdot\nabla)\frac{\bm r-\bm r'}{|\bm r-\bm r'|^3}
    \end{equation*}
    根据式~\ref{eq:dvrgnc_sqr_invrs} 的结果,第一项恰好等于 $4\pi\bm J(\bm r')$。而对于第二项,通过微分算符内的变量代换可得
    \begin{equation*}
        (\bm J(\bm r')\cdot\nabla)\frac{\bm r-\bm r'}{|\bm r-\bm r'|^3} = -(\bm J(\bm r')\cdot\nabla')\frac{\bm r-\bm r'}{|\bm r-\bm r'|^3}
    \end{equation*}
    考虑其中的 $x$ 分量:根据矢量分析的恒等式 $(\bm A\cdot\nabla)f = \nabla\cdot(f\bm A)-f(\nabla\cdot\bm A)$,有
    \begin{equation*}
        (\bm J(\bm r')\cdot\nabla')\frac{x-x'}{|\bm r-\bm r'|^3} = \nabla'\cdot\left(\frac{x-x'}{|\bm r-\bm r'|^3}\bm J(\bm r')\right) - \frac{x-x'}{|\bm r-\bm r'|^3}(\nabla'\cdot\bm J(\bm r'))
    \end{equation*}
    根据连续性方程~\ref{eq:cntnty},恒定电流的情况下上式第二项为零。因此只需在积分式内考虑第一项:假设闭合曲面 $\mc S$ 包围了区域 $\mc V$,则
    \begin{equation*}
        \int_{\mc V}\d\bm r'\,\left[\nabla'\cdot\left(\frac{x-x'}{|\bm r-\bm r'|^3}\bm J(\bm r')\right)\right]=\int_{\mc V}\d\bm r'\,\frac{x-x'}{|\bm r-\bm r'|^3}\bm J(\bm r')
    \end{equation*}
    选择足够大的区域 $\mc V$ 使远处 $\mc S$ 上的 $\bm J=\bm 0$,即可知这一项为零。
\end{proof}

\subsubsection{电磁感应与 Maxwell 方程组}

实验表明,在大多数情况下,能够维持的恒定电流要归功于电荷所受的某种力。例如,设单位电荷量所受的力为 $\bm f$,则显然电场对其的贡献为 $\bm f=\bm E$,\textbf{欧姆定律(Ohm's Law)}指出 $\bm J=\mbf\sigma\bm f$,其中比例系数 $\mbf\sigma$ 称为\textbf{电阻率(resitivity)}。反过来讲,电荷所受的力并不仅仅来源于电场:式~\ref{eq:lrntz_frc} 表明磁场也能使(运动的)电荷受力。我们还知道接入电路中的化学电池等元件也能产生电流,因此电荷所受的力并非一定直接来自于电磁场对电荷的作用。

对于一段有向的电路 $L$,若其上每一处单位电荷量的受力为 $\bm f(\bm r)$,则可以定义电动势(electromotive force)
\begin{equation*}
    \mc E = \int_L\,\bm f(\bm r)\cdot\bm{\d l}
\end{equation*}
例如若 $\bm f$ 的来源是某个电场,$\bm r_0,\bm r_1$ 是 $L$ 的起点和终点,则
\begin{equation*}
    \mc E = \int_{\bm r_0}^{\bm r_1} \bm E(\bm r)\cdot\bm{\d r}=U(\bm r_0)-U(\bm r_1)
\end{equation*}
这说明电场驱动下电路的电动势是起点和终点电势之差。而法拉第通过实验总结了另一电动势之来源:对于一个闭合有向电路 $L$,如果它是包围了曲面 $S$,且曲面中通过了磁场 $\bm B$,则有

\begin{theorem}[法拉第(Faraday)电磁感应定律]
    \begin{equation*}
        \mc E = -\frac{\d}{\d t}\int_S\bm B\cdot\bm {\d S}
    \end{equation*}
\end{theorem}
等式右端的第二类曲线积分 $\Phi=\int_S\bm B\cdot\bm {\d S}$ 又称为回路 $L$ 的\textbf{磁通量(magnetic flux)}。假设磁通量的变化是由磁场 $\bm B(\bm r,t)$ 的变化引起的,且将这一电动势视作电场的作用,则有
\begin{equation*}
    \oint_L\bm E(\bm r,t)\cdot\bm{\d r} = -\int_S\frac{\p\bm B(\bm r,t)}{\p t}\cdot\bm {\d S}
\end{equation*}
使用斯托克斯定理,可以转化为
\begin{equation*}
 \int_S\left(\nabla\times\bm E(\bm r,t) + \frac{\p\bm B(\bm r,t)}{\p t}\right)\cdot\bm {\d S} = 0
\end{equation*}
根据 $L$ 及其包围的 $S$ 的任意性,可得法拉第电磁感应定律的微分形式
\begin{equation}
    \nabla\times\bm E(\bm r,t) = -\frac{\p\bm B(\bm r,t)}{\p t}
\end{equation}
这一定律是电场无旋定律~\ref{thm:elctrc_fld_crl} 的补充:后者的无旋电场仅来自于固定电荷分布,而前者来自于变化的磁场。这一变化磁场产生的电场因而称为\textbf{涡旋电场(vortex electric field)}。

法拉第电磁感应定律首次为电场和磁场的规律加入了时间参数,因此或许可以试图将这些定律全部拓展到变化的电磁场。但麦克斯韦(Maxwell)注意到,矢量分析公式 $\nabla\cdot(\nabla\times A)=0$ 会面临失效:
\begin{equation*}
    \nabla\cdot(\nabla\times B(\bm r,t)) = \mu_0\nabla\cdot\bm J(\bm r,t)
\end{equation*}
根据连续性方程~\ref{eq:cntnty},等式右侧只有在固定电荷密度 $\frac{\p\rho}{\p t}=0$ 的情况下才等于 0。对于非固定电荷的情况,根据电场的高斯定律~\ref{thm:elctrc_fld_gss}
\begin{equation*}
    \mu_0\nabla\cdot\bm J(\bm r,t) = -\frac{\p\rho(\bm r,t)}{\p t} = -\epsilon_0\nabla\cdot\frac{\p\bm E(\bm r,t)}{\p t}
\end{equation*}
麦克斯韦创造性地为安培定律加入了位移电流(displacement current)项:

\begin{theorem}[磁场的安培(Ampere)定律 v2.0]
    \begin{equation*}
        \nabla\times\bm B=\mu_0\bm J+\mu_0\epsilon_0\frac{\p\bm E(\bm r,t)}{\p t}
    \end{equation*}
\end{theorem}

另一方面,法拉第电磁感应定律并不会破坏矢量分析的结果,因为
\begin{equation*}
    \nabla\cdot(\nabla\times E(\bm r,t)) = -\frac{\p}{\p t}(\nabla\cdot\bm B(\bm r,t))=0
\end{equation*}
至此,这些公式在数学形式上自洽,总结为以下方程组

\begin{theorem}[麦克斯韦(Maxwell)方程组]
    设空间中有电荷分布 $\rho$ 和电流密度分布 $\bm J$,则空间中的电场 $\bm E$ 和磁场 $\bm B$ 满足:
    \begin{align*}
        \nabla\cdot\bm E & =\frac{\rho}{\epsilon_0}\\
        \nabla\times\bm E & = -\frac{\p\bm B}{\p t}\\
        \nabla\cdot\bm B & =0\\
        \nabla\times\bm B & =\mu_0\bm J+\mu_0\epsilon_0\frac{\p\bm E}{\p t}
    \end{align*}
\end{theorem}

\subsubsection{电磁场的能量}

\iffalse
\subsubsection{拉莫尔公式}
得到了辐射功率的拉莫尔(Larmor)公式
\begin{equation*}
    P=\frac{\mu_0q^2a^2}{6\pi c}
\end{equation*}


\subsection{波动力学的初步框架}

\subsubsection{谐振子链上的振动传播形成弹性波}


如果把无穷多个质量都为 $m$ 的质点用劲度系数为 $\kappa$ 的弹簧在一条直线上连成一串,这些质点的一维坐标是 $x_n$,弹簧的自然长度都为 $l$。假设第 $n$ 个质点在 $x=nl$ 位置附近,偏移量为 $u_n=x_n-nl$,于是两个相邻质点间的距离都可以表示为 $l+u_{n+1}-u_{n}$。首先注意到 $\ddot{x}_n=\ddot{u}_n$,而且每个质点都受到来自两边方向相反的拉力,分别正比于质点间距偏离弹簧原长的量 $u_{n+1}-u_n$ 和 $u_n-u_{n-1}$,因此可以列出每一个质点的运动方程:
\begin{equation}
    m\ddot{u}_n =\kappa(u_{n+1}+u_{n-1}-2u_n)
\end{equation}




此时推测解的形式为 $u_n=A\e^{\i(sn-\omega t)}$,其中有两个参数 $\omega$ 和 $s$,代入方程,得到
\begin{equation}
    -m\omega^2 u_n=\kappa(\e^{\i s}+\e^{-\i s}-2) u_n
\end{equation}
从中解出
\begin{equation}
    \omega^2 = \frac{2\kappa}{m}(1-\cos s)
\end{equation}
因此参数 $\omega,s$ 只要满足这一关系,所猜测的形式就是方程的解。




\subsection{微观质点的波粒二象性}

\subsubsection{粒子和波以 Planck 常数相联系}

\subsubsection{Bohr 模型解释了氢原子光谱}

\subsubsection{遵循 Schrödinger 方程的波包像质点一样运动}

\subsubsection{波函数的物理意义是空间中的概率密度}

\section{单粒子问题的量子力学框架}

\subsection{态空间的线性代数}

\section{角动量}

\section{氢原子势}

\section{多粒子问题的量子力学框架}

\chapter{多电子原子}

\chapter{多原子分子}



本章的目标是从经典力学的语言过渡到量子力学的语言,以描述化学中最简单和基本的系统:氢原子的电子结构。为了达到这一目标,首先要摆脱牛顿视角的束缚,重新建立经典质点力学的理论框架。然后在早期量子现象观察的基础上搭建量子力学的线性代数语言,借此提出量子力学的基本公设。然后将经典力学中的概念迁移到量子力学体系中,以提出轨道角动量算符为高潮。再利用角动量的性质以及对库仑势的分析,求解氢原子的电子布居。最后补充了张量积的概念,用以澄清质子和电子的运动关系。

除主线目标之外,本章还涉及到一些与主线相关而为后文做铺垫的重要内容。方形势模型将用于特殊的分子轨道、能带及箱中粒子的建模,谐振子和刚性转子模型将用于分析分子光谱,角动量的一般定义为引入自旋自由度留下了空间,角动量耦合将为多电子原子的分析打下基础。


\section{前期准备}

在进入量子力学的“力学”部分之前,首先要阐明“量子”观念的来由,并搭建一套完善的数学结构和符号体系以便今后的讨论。

\subsection{早期量子论}

普朗克(Planck)在对黑体辐射的研究中提出了光量子假设:他指出,频率为 $\nu$ 的光束中饱含能量为 $h\nu$ 的\textbf{光子}作为其基本单位,其中 $h$ 为实验测得的\textbf{普朗克常数}。这一假设在爱因斯坦 ( Einstein)对光电效应的解释中得到了进一步强化:

\begin{definition}[光电效应]\label{def:elecphoto} 

光电效应是指原子核外的电子吸收一定频率的光从而激发为自由电子的过程。用一束紫外线轰击金属,可以观察到出射电子的最大动能 $E_\mathrm{k}$ 只与光的频率 $\nu$ 有关:
\begin{equation}
   \label{a}
   E_\mathrm{k} = \frac 1 2 m_\e v^2 = h\nu - W
\end{equation}

其中 $m_\e$ 为电子质量,$v$ 为光电子的运动速率,而 $W$ 为\textbf{逸出功},是金属的特征性质。
\end{definition}

在光电效应中还观察到以下现象:

\begin{itemize}
   \item 存在截止频率 $\nu_0$ 满足 $h\nu_0=W$,凡是低于该频率的光,无论光强多大都无法引发光电流。
   \item 光电流强度只与光强有关,与光的频率无关。
   \item 光电流的响应极迅速,几乎在光线到达的瞬时就有光电子产生。
   \item 若要用反向电压阻止光电子,最小电压为 $U$ 满足 $eU=h\nu-W$,其中 $e$ 为电子电量。
\end{itemize}

基于此假设光可以视作具有能量和动量的粒子流,其基本单元称为光子,具体表现为: 
\begin{postulate}[爱因斯坦关系]\label{pos:einstrel} 
设光的波长为 $\lambda$,光子的能量为 $E$,动量为 $\bm p$,则有
\begin{equation}
   E = h\nu = \hbar\omega,\quad \bm p=\hbar\bm k
\end{equation}

其中\textbf{圆频率} $\omega = 2\pi\nu$,\bf{波矢}的方向朝向光波的方向,且大小满足 $|\bm k|=2\pi/\nu$。\bf{约化普朗克常数} $\hbar = h/2\pi$。
\end{postulate}

这一假设集中体现了光的波粒二象性:它将描绘粒子的动量、能量和描绘波的频率、波矢联系在一起,其中的桥梁便是普朗克常数。此处应指出
\begin{equation*}
   h \approx \SI{6.626e-34}{J\cdot s}
\end{equation*}

德布罗意(de Broglie)天才的创造又把关系式 \ref{pos:einstrel} 以同样的形式推广到了实物粒子。
\begin{postulate}[德布罗意关系]\label{pos:debrel} 
   实物粒子具有波粒二象性:设实物粒子的动量为 $\bm p=m\bm v$,能量为 $E$,则物质波的波矢和频率满足
   \begin{equation}
      E = h\nu = \hbar\omega,\quad \bm p=\hbar\bm k
   \end{equation}
   还可以写为
   \begin{equation}
      \lambda = \frac h p
   \end{equation}
   
\end{postulate}

这一关系式的深刻之处不仅在于物质波概念的提出,更重要地是将描述运动的动量和描述空间尺度的波矢联系在一起,这一点的重要性将贯穿整个量子力学体系。

\subsection{态空间的线性代数}
量子力学的颠覆性不止在于上述重要而零散的结论,更在于一套与经典物理学迥然不同的数学模型。在冯诺依曼等人的努力下,泛函分析成为了公理化量子力学的数学底色,而其通俗版本就是用更方便的线性代数来叙述的。代数是研究带有运算的集合的学问,我们将首先定义集合,再逐渐补充其中的计算。

\subsubsection{态空间作为向量空间}

在集合之上赋予所谓\textbf{线性运算},将产生向量空间的结构,接下来将澄清这些线性运算的性质,它们有些会成为计算时的直觉,有些则蕴含深刻的物理意义。

\begin{definition}[态空间中的态矢]\label{def:statespace}
   态空间是依附于数域 $\C$ 的向量空间(定义详见 \ref{def:cmplx_vctr_spc}),其中的向量称为\textbf{态矢},我们暂且将其记作 $\psi$ 的形式。
\end{definition}

映射本身也是可供研究的数学对象,可以定义映射之间的运算:例如给出线性映射的加法和数乘概念:
\begin{itemize}
    \item 对于任意 $A, B\in\mc L(\ms E,\mc F),\psi\in\ms E$,$(A+B)\psi=A\psi+B\psi$
    \item 对于任意 $A\in\mc L(\ms E,\mc F),\psi\in\ms E$,$(aA)\psi=a(A\psi)$
\end{itemize}
则容易验证 $\mc L(\ms E,\mc F)$ 也构成向量空间。

复数集 $\C$ 本身也可以视作一个依附于复数域 $\C$ 的向量空间,并沿用我们熟悉的复数运算形式。所以从向量空间到 $\C$ 的映射值得深入研究,根据上面的讨论,这种映射也能构成一个向量空间。

\begin{definition}[对偶态空间中的态矢]\label{def:dualspace}
我们将 $\ms E$ 上的线性泛函组成的 $\mc L(\ms E,\C)$ 特别称为 $\ms E$ 的对偶空间 $\ms E^\dg$,其中的元素,即这些线性泛函也是\textbf{态矢},暂且记作 $\vphi$ 的形式。
\end{definition}

在线性运算的加持下,描述一个向量空间不需要其中全部的向量。例如态空间 $\ms E$ 中若存在一组态矢 $\{e_i\}$ 使得对于所有 $\psi\in\ms E$ 都有唯一的一组系数 $\{a_i\}$ 使得 $\psi=\sum_ia_i e_i $,则称这组右矢为 $\ms E$ 的一组基。而给出 $\ms E$ 中的基之后就能诱导出对偶空间 $\ms E^\dg$ 中的基,原空间和对偶空间就这样联系起来:

\begin{definition}[对偶基]\label{def:dualbasis}
以上述的一组基为例,如果线性泛函组 $\{\vphi_i\}$ 满足:
\begin{equation*}
    \vphi_i(e_j)=\delta_{ij}
\end{equation*}
则称其为原空间的基 $\{e_i\}$ 在对偶空间中的对偶基。
\end{definition}

原空间的算符在对偶空间也有所对应:能够定义 $A\in\mc L(\ms E)$ 的对偶算符 $A^\dg\in\mc L(\ms E^\dg)$,它由下式确定:
\begin{equation}
    A^\dg\vphi=\vphi\circ A
\end{equation}
注意到对于一个原空间矢量 $\ket\psi$,$A\psi$ 仍给出一个原空间矢量。而 $\vphi$ 是对偶空间矢量,即线性泛函,它作用于原空间矢量 $\psi$ 得到一个数。因而 $\vphi\circ A$ 对每个 $\psi$ 给出一个数,也是一个线性泛函,从而位于对偶空间中。因此 $A^\dg$ 是对偶空间到自身的映射,且以上每一步都是线性的:它确实是对偶空间上的算符。

\subsubsection{态空间作为内积空间}

定义更多的运算会使集合的结构变得更复杂,而更复杂的结构产生了更丰富的性质:有了适当的内积运算之后,态空间便升格为内积空间。澄清这个运算的细节同样是为了建立计算上的直觉和物理上的内涵。

\begin{definition}[态空间上的内积]\label{def:innerproduct}
$\ms E$ 上的内积是一个映射 $\Braket{\cdot,\cdot}:\ms E^2\rightarrow\C$,它要满足以下性质:
\begin{enumerate}
    \item 正性:$\Braket{{\psi},{\psi}} \ge 0$
    \item 定性:$\Braket{{\psi},{\psi}} = 0$ 当且仅当 ${\psi}={0}$
    \item 第二个位置加性:$\Braket{{\psi},{\vphi}+{\chi}}=\Braket{{\psi},{\vphi}}+\Braket{{\psi},{\chi}}$
    \item 第二个位置齐性:$\Braket{{\psi},a{\vphi}}=a\Braket{{\psi},{\vphi}+{\chi}}$ 
    \item 共轭对称性:$\Braket{{\psi},{\vphi}}=\Braket{{\vphi},{\psi}}^*$ 
\end{enumerate}
以上 $\psi,\vphi,\chi$ 都是 $\ms E$ 中的任意态矢
\end{definition}

值得注意的是:上述定义并没有指出这一内积的\textbf{具体运算规则},但只要满足了这些条件,且当 $\ms E$ 是有限维空间时就有以下重要结论:
\begin{theorem}[里斯(Rietz)表示定理]\label{thm:Rietz}
对于任意的 ${\vphi}\in\ms E^\dg$,都存在唯一的右矢 ${\vphi'}\in\ms E$,使得对于任意的 ${\psi}\in\ms E$ 都有 ${\vphi}({\psi})=\Braket{{\vphi'},{\psi}}$。
\end{theorem}

这里使用了 $\vphi$ 和 $\vphi'$ 来彰显这种对应关系,自然地将 $\ms E^\dg$ 中的一个态矢和 $\ms E$ 中的一个态矢联系起来,使得然而一个线性泛函在矢量上的作用与矢量之间的内积具有了等价性。一种更优美的记法是将态空间 $\ms E$ 中的矢量 $\psi$ 加以括号称为\textbf{右矢} $\ket\psi$,对偶态空间 $\ms E^\dg$ 中的矢量 $\vphi$ 加以括号称为\textbf{左矢} $\bra\vphi$ 进而采取以下记号来简记内积运算:
\begin{equation}\label{eq:iptolm}
    \braket{\vphi|\psi}=\Braket{\vphi',\psi}=\vphi(\psi)
\end{equation}

内积产生了正交概念,并自然地诱导出范数,因而在内积空间中能够得到一种特殊的基:

\begin{definition}[正交归一基]\label{def:orthonormalbasis}
$\ms E$ 的正交归一基 $\{\ket{u_i}\}$ 首先是一组基,且要满足
\begin{equation*}
    \braket{u_i|u_j}=\delta_{ij}
\end{equation*}
\end{definition}

这里使用了克罗内克(Kronecker)delta 符号,当 $i=j$ 时取值为 1 否则为 0。

通过格拉姆-施密特(Gram-Schmidt)正交化,我们实际上能够从任何一组基得到一组正交归一基。$\ms E$ 中的任何一个右矢 $\ket{\psi}$ 在正交归一基 $\{\ket{u_i}\}$ 上的分解都可以用内积表达:

\begin{corollary}[完备性关系式]\label{cor:completeness}
\begin{equation}
    \ket{\psi}=\sum_i\braket{u_i|\psi}\ket{u_i}=\sum_i\ket{u_i}\braket{u_i|\psi}
\end{equation}
这也就是说
\begin{equation}
    \sum_i\ket{u_i}\bra{u_i}=\mr 1
\end{equation}
\end{corollary}

\begin{remark}
值得解释的是第二式:首先注意到 $\bra{u_i}$ 是线性泛函,它将连缀在其右方的右矢映成 $\C$ 中的数,然后与 $\ket{u_i}$ 数乘,又得到一个右矢;也就是说每一个 $\ket{u_i}\bra{u_i}$ 都把 $\ms E$ 中的右矢映成 $\ms E$ 中的右矢。且很容易验证:这是一个线性映射,同时也是一个算符,即得到 $\ket{u_i}\bra{u_i}\in\mc L(\ms E)$。由于 $\mc L(\ms E)$ 是向量空间,所以 $\sum_i\ket{u_i}\bra{u_i}\in\mc L(\ms E)$。而这里的完备性关系式指出:这是一个恒等算符,即把任何 $\ms E$ 中的元素映向它自己。因此完备性关系式可以自然地插入任何线性算式中的任何右矢之前,并(在大多数情况下都可以大胆地)把求和号移到最外边(,而忽略了无穷级数可能带来的一致收敛性问题)。

在不致混淆的前提下,把恒等算符的数倍记作这个数本身,因此恒等算符记为 1。
\end{remark}

结合内积定义 \ref{def:innerproduct} 中的前两条,我们正式定义态矢的范数(无所谓左右矢,所以这里剥离了括号) $\|\psi\|=\braket{\psi|\psi}$,易见它是非负的,且只在 $\ket{\psi}=\ket{0}$ 时取 $0$。从完备性关系式 \ref{cor:completeness} 和正交归一性质 \ref{def:orthonormalbasis} 可以推出:范数也可以在正交归一基 $\{\ket{u_i}\}$ 下表达,正如同解析几何中的向量长度公式:

\begin{theorem}[勾股定理]\label{thm:Pythagorean}
\begin{equation*}
   \|\psi\|=\sum_i|\braket{u_i|\psi}|^2 
\end{equation*}
\end{theorem}

在一般的内积式中插入完备性关系式,就推广为更一般的运算法则:
\begin{equation}\label{eq:ipcommon}
   \braket{\vphi|\psi}=\sum_i\braket{\vphi|u_i}\braket{u_i|\psi}=\sum_i\braket{u_i|\vphi}^*\braket{u_i|\psi}
\end{equation}
而“插入完备性关系式”这一操作,将作为一个常用的推导技巧渗透进
今后的推导中,例如能够用其验证以下结论:

\begin{corollary}[左矢的分量]\label{cor:bracomponent}
如果 $\{\ket{u_i}\}$ 是 $\ms E$ 的一组正交归一基,则有
\begin{equation*}
    \bra{\psi}=\sum_i\braket{\psi|u_i}\bra{u_i}=\sum_i\braket{u_i|\psi}^*\bra{u_i}
\end{equation*}
\end{corollary}

\subsubsection{厄米算符和本征系统}

里斯表示定理实际上埋下了比对偶更深的伏笔,它允许了一类算符之间的关系存在:
\begin{definition}[算符的伴随]\label{def:adjoint}
若 $A\in\mc L(\ms E)$,则 $A$ 的伴随 $A^*\in\mc L(\ms E)$,这个算符使得对于任意的 $\vphi,\psi\in\ms E$,有
\begin{equation*}
  \Braket{\vphi,A\psi}=\Braket{A^*\vphi,\psi}
\end{equation*}
\end{definition}
如果令 $\ket{\chi} = A^*\ket{\vphi}$,简记 $\bra{\chi} = \bra{\vphi}A$,则这一关系可以记作
\begin{equation}
    \bra{\vphi}(A\ket{\psi})=(\bra{\vphi}A)\ket{\psi}=\braket{\vphi|A|\psi}
\end{equation}
根据对偶映射的定义,从这一关系也可以看出简记的合理性
\begin{equation}
    A^\dg\bra{\vphi}=\bra{\vphi}\circ A=\bra{\vphi}A
\end{equation}
这就是说,对偶算符作用于左矢得到的新左矢作用于某个右矢,等价于伴随算符作用于这个左矢对应的右矢后与某个右矢内积。

\begin{remark}
我们可以扩大符号 $^*$ 的含义,对于一个含有左矢、右矢、算符和标量的式子,对其整个施以 $^*$ 的结果是
\begin{enumerate}
    \item 左矢、右矢、算符逆序排列,标量总是可以移到最前面
    \item 左矢变为对应的右矢、右矢变为对应的左矢、算符变为其伴随、标量变为其复共轭
\end{enumerate}
一个简单的例子是:
\begin{equation}
    \braket{\vphi|A|\psi}^*=\braket{\psi|A^*|\vphi}
\end{equation}
\end{remark}

伴随操作引出了一类特殊的算符:

\begin{definition}[厄米算符]\label{def:hermite}
如果 $A=A^*$,就称 $A$ 为\textbf{厄米(Hermite)算符}。
\end{definition}

这类算符的特殊性质将从其产生的\textbf{不变子空间}中看出。所谓不变子空间要相对于某个算符 $A$ 而言,其中每个向量经 $A$ 仍映到这个空间中。而最重要的不变子空间就是一维的不变子空间,它意味着其中的向量经 $A$ 映射后与原来仍然线性相关,算符的作用退化为一个数乘。

\begin{definition}[本征值和本征向量]\label{def:eigen}
如果算符 $A\in\mc L(\ms E)$ 和非零右矢 $\ket{\psi}\in\ms E$ 满足
\begin{equation*}
    A\ket{\psi}=a\ket{\psi}
\end{equation*}
其中 $a\in\C$,则 $a$ 是 $A$ 的本征值,$\ket{\psi}$ 是这一本征值的本征(右)矢。一个算符的全体本征值又称为这个算符的谱。
\end{definition}

对于一般的算符而言,本征值可以是任何复数,本征向量之间没有必然的联系;但厄米算符的本征值和本征向量具有特殊性质:

\begin{theorem}[谱定理]\label{thm:spectrum}
厄米算符 $A\in\mc L(\ms E)$ 的所有本征值都是实数, 属于不同本征值的本征向量相互正交,且存在一组本征向量是 $\ms E$ 的正交归一基。
\end{theorem}

这意味着态空间 $\ms E$ 上的一切 可以依照其中的一个厄米算符的正交归一的本征向量组成的基来进行。我们可以给不同的本征值编号为 $a_n$,则有对应的本征矢 $\ket{u_n}$,但很多时候存在一组线性无关的右矢 $\{\ket{u_n^i},i=1,\cdots,g_n\}$,它们都是 $a_n$ 的本征矢。这些本征矢张成了 $a_n$ 对应的本征子空间 $\ms E_n$,而 $\ms E_n$ 的维数(几何重数,在这里与代数重数相等),也即上述 $g_n$ 可能取到的最大值,也称为本征值 $a_n$ 的\textbf{简并度}。

\subsubsection{算符的对易性}
算符之间的复合是函数复合关系。对于一般的算符(不一定是线性的),算符集合与其复合运算构成了一个\textbf{群},服从结合律而不服从交换律。所以,如果我们做以下定义
\begin{definition}[对易子]\label{def:commutator}
算符 $A,B\in\mc L(\ms E)$ 的对易子为
\begin{equation*}
    [A,B]=AB-BA
\end{equation*}
\end{definition}
就意味着对易子本身也产生一个算符,它一般不为 0。

\begin{remark}
至此,这里出现的 0 已经展现出三种含义:数字 $0\in\C$,零矢 $0\in\ms E$ 或 $0\in\ms E^\dg$,和恒等算符的 0 倍算符 $0$(它将任意态矢映成零矢)。
\end{remark}

容易通过计算证明有关对易子的以下结论,它们也将成为计算中的直觉:
\begin{theorem}[对易子的基本代数性质]\label{thm:cmmttr_algbr_prprt}
\begin{align}
    & [A,B] = -[B,A]\\
    & [A,(B+C)] = [A,B]+[A,C]\\
    & [A,BC] = [A,B]C + B[A,C]\\
    & [A,[B,C]]+[B,[C,A]]+[C,[A,B]]=0\\
    & [A,B]^*=[B^*,A^*]
\end{align}
\end{theorem}

特殊情况 $[A,B]=0$ 即表明 $A$ 和 $B$ \textbf{对易}(可交换)。能够证明,两个对易厄米算符的本征系统具有一些重要性质:

\begin{theorem}[对易厄米算符的本征系统]\label{thm:commutableHermite}
若厄米算符 $A,B\in\mc L(\ms E)$ 对易,则有
\begin{enumerate}
    \item 若 $\ket{\psi}$ 是 $A$ 的本征矢,则 $B\ket{\psi}$ 也是 $A$ 的本征矢。
    \item 若 $\ket{\psi_1},\ket{\psi_2}$ 是 $A$ 的不同本征值对应的本征矢,则矩阵元 $\braket{\psi_1|B|\psi_2}=0$
    \item 存在一组由 $A,B$ 的共同本征矢组成的 $\ms E$ 的正交归一基
\end{enumerate}
\end{theorem}

第一条性质说明 $A$ 的本征子空间在 $B$ 下不变;第二条性质说明 $B$ 不但能保证 $A$ 的本征子空间不变,还能保留 $A$ 不同本征值对应的本征矢之间的正交关系;第三条性质则意味着我们可以找到一组形如 $\ket{u_{n,p}^i}$ 的本征矢,其下标的含义是
\begin{align*}
    A\ket{u_{n,p}^i}&=a_n\ket{u_{n,p}^i}\\
    B\ket{u_{n,p}^i}&=b_p\ket{u_{n,p}^i}
\end{align*}

如果 $a_n$ 的简并度大于 $1$,则单由这个本征值只能确定一个 $A$ 的多维本征子空间,这些本征子空间实际上对态空间 $\ms E$ 进行了一次剖分。而在这个子空间内又可以找到对应于不同 $b_p$ 的本征矢,从而对应 $B$ 的不同本征子空间与 $A$ 的同一个本征子空间的交,从而进一步剖分 $A$ 的本征子空间(能够证明,向量空间的交仍是向量空间)。如果继续引入新的厄米算符,且它与之前的算符都对易,则用其本征子空间继续切割……最终期望的结果就是得到的子空间都是一维的:

\begin{definition}[对易厄米算符完全集]\label{def:CSCO}
厄米算符 $A,B,C,\cdots\in\mc L(\ms E)$ 构成对易厄米算符完全集(C.S.C.O),要求
\begin{enumerate}
    \item 这些算符两两对易
    \item $A,B,C,\cdots$ 的本征值 $a_n,b_p,c_r,\cdots$ 共同确定一个一维的本征子空间,也即在相差一个常系数的意义下确定一个本征矢,这个本征矢因此可以记作 $\ket{a_n,b_p,c_r,\cdots}$
\end{enumerate}
\end{definition}

\begin{remark}
在提及 C.S.C.O. 时,通常暗示这组算符是容量最小的。因为当一组算符已经是 C.S.C.O. 时,添加一个与其中所有算符都对易的厄米算符仍得到一组 C.S.C.O,因此只需选择最小的 C.S.C.O.。
\end{remark}

\subsection{表象及其变换}

迄今为止,态空间和算符都是抽象的数学对象:它没有提供计算的接口和直观的图像。但是线性代数允许我们采取一种自由的方式来将这些抽象的对象转换为我们所熟悉的数字和数列,从而可以被计算和描绘。

\subsubsection{同构思想}
我们已经得到了右矢在正交归一基 $\{\ket{u_i}\}$ 上的分量以及左矢在其对偶基上 $\{\bra{u_i}\}$ 的分量式。如果设 $c_i=\braket{u_i|\psi}, b_i=\braket{u_i|\vphi}, b_i^*=\braket{\vphi|u_i}$,则右矢 $\ket{\psi}$ 可以表示为列向量 $\bm c =(\cdots,c_i,\cdots)\T$,同理 $\ket{\vphi} $ 表示为 $\bm b=(\cdots,b_i,\cdots)\T$。于是通过一组基(尤其是正交归一基),我们实际上把完全抽象的态矢和 $\C^{n\times 1}$ 中的向量一一对应地联系在一起。$\C^{n\times 1}$ 也是向量空间,以列向量为例,它有一组\textbf{标准基} $\bm e_j$:第 $j$ 个分量为 $1$,其他分量都为 $0$ 的列向量。这实际建立一个向量空间之间的同构,以右矢为例,这样的同构 $\Phi_\text{ket}:\ms E\ra\C^{n\times 1}$ 定义为:
\begin{equation}
   \Phi_\text{ket}(\ket\psi) = \Phi_\text{ket}\left(\sum_i c_i\ket{u_i}\right)=\sum_i c_i\bm e_i = \begin{pmatrix}
   \vdots \\ c_i \\ \vdots
   \end{pmatrix} = \bm c
\end{equation}
则 $\ms E$ 中的内积 $\Braket{\cdot,\cdot}_{\ms E}$ 也被这一同构所保持到 $\C^{n\times 1}$ 中的标准内积 $\Braket{\cdot,\cdot}_{\C^{n\times 1}}$,同时也可视作代表左矢的行向量与代表右矢的列向量直接相乘:

\begin{equation}
   \braket{\vphi|\psi}=\Braket{\vphi,\psi}_{\ms E}=\Braket{\sum_i b_i\bm e_i,\sum_i c_i\bm e_i}_{\C^n}=\sum_i b^*_ic_i=\Braket{\bm b,\bm c}_{\C^n} = \bm b^*\bm c
\end{equation}
此处 $\bm b^*$ 是一个行向量,即 $\bm b$ 的共轭转置 $(\cdots, b_i^*,\cdots)$。而连缀的 $\bm b^*\bm c$ 遵循矩阵乘法的意义,因此只要
经过一个同构 $\Phi_\text{bra}:\ms E^\dg\ra \C^{1\times n}$,使得:
\begin{equation}
   \Phi_\text{bra}(\bra\vphi) = \Phi_\text{bra}((\ket\vphi)^*)=\Phi_\text{bra}\left(\left(\sum_i b_i\ket{u_i}\right)^*\right)=\Phi_\text{bra}\left(\sum_i b_i^*\bra{u_i}\right)=\sum_i b_i^*\bm e_i^* = \begin{pmatrix}
   \cdots & b_i^* & \cdots
   \end{pmatrix} = \bm b^*
\end{equation}
就有
\begin{equation}
    \braket{\vphi|\psi} = \Phi_\text{bra}(\bra\vphi)\Phi_\text{ket}(\ket\psi)
\end{equation}
实现了左矢(作为线性泛函)在右矢上的作用、两个态矢的内积、两个 $\C^n$ 中的向量的内积、行向量与列向量相乘四个概念的大统一。

\begin{remark}
在不需要区分行向量和列向量时,可以将 $\C^{1\times n}$ 和 $\C^{n\times 1}$ 统一为 $\C^n$。尽管它们是生活在不同的向量空间中的不同数学对象,但可以构造显然的同构使二者等价。
\end{remark}

循着这种思想来考察任意一个算符 $A\in\mc L(\ms E)$,任取 $\ms E$ 的一组正交归一基 $\{\ket{u_i}\}$,将所得的 $A\ket{u_j}\in\ms E$ 在这组基上重新分解,得到的 $\ket{u_i}$ 前的系数记作 $A_{ij}$,也即
\begin{equation}
   A_{ij} = \bra{u_i}A\ket{u_j}
\end{equation}
将其放在一个矩阵的第 $i$ 行和第 $j$ 列,于是 $\mbf A=(A_{ij})$ 称为 $A$ 在基 $\{\ket{u_i}\}$ 下的矩阵,它唯一确定了算符 $A$。这样就产生了算符到矩阵空间的同构 $\Phi_\text{op}:\mc L(\ms E)\ra\C^{n\times n}$,它使得 $\Phi_\text{op}(A)=\mbf A$。

容易验证,$A^\dg$ 在 $\{\ket{u_i}\}$ 的对偶基  $\{\bra{u_i}\}$ 下的矩阵是 $\mbf A\T$,即 $\mbf A$ 的转置。而 $A^*$ 在 $\{\ket{u_i}\}$ 下的矩阵是 $\mbf A^*$,即 $\mbf A$ 的共轭转置。

最后增补从 $\C$ 到 $\C$ 的同构 $\Phi_\text{num}(z)=z$,即恒等映射,则此时含有算符的算式也可以在同构意义下变成矩阵和向量运算:沿用以上记号,我们有
\begin{equation}
   \braket{\vphi|A|\psi}=\Phi_\text{num}(\braket{\vphi|A|\psi}) = \Phi_\text{bra}(\bra{\vphi})\Phi_\text{op}(A)\Phi_\text{ket}(\ket{\psi})=\bm b^*\mbf A\bm c
\end{equation}

综上所述,我们通过同构,把完全抽象的态矢和算符与具体的数列和矩阵联系在一起,态矢与算符的作用变为数列和矩阵的乘法,$^*$ 的作用变为统一的共轭转置操作,且在最终通过内积归结为数时得到了相同的结果,从而保持了之前定义的所有数学结构和数学结论。而这种联系的建立完全依赖于正交归一基的选择。因此选定了一组正交归一基就确定了一个\textbf{表象},在特定的表象下,算符和态矢组织为一些 $\C$ 中的数的阵列,从而可以通过熟悉的方式进行计算。而同构又是可逆的,在 $\C$ 背景下的阵列中完成计算之后,又可以通过正交归一基返回到 $\ms E$ 背景下的抽象表示中。

而选定不同的正交归一基,态矢在基上的分量就会发生变化,算符的矩阵元也会发生变化。例如另选一组正交归一基 $\{\ket{v_i}\}$,则 $\ket{\psi}$ 在新基上的分量为
\begin{equation}
   c_i'=\braket{\psi|v_i} = \sum_j\braket{\psi|u_j}\braket{u_j|v_i}
\end{equation}

$A$ 在新基下的矩阵 $\mbf A'$ 的矩阵元为
\begin{equation}
   A_{ij}'=\braket{v_i|A|v_j} = \sum_k\sum_l\braket{v_i|u_k}\braket{u_k|A|u_l}\braket{u_l|v_j}=\sum_{k,l}\braket{v_i|u_k}A_{kl}\braket{u_l|v_j}
\end{equation}

于是,利用两组正交归一基之间的内积关系,结合基的完备性关系式,就实现了基的变换,也称为\textbf{表象变换}。如果定义矩阵 $\mbf U = (U_{ij})$,其中 $U_{ij} = \braket{u_i|v_j}$,则上式重构为
\begin{equation}
    \mbf A' = \mbf U^*\mbf A\mbf U
\end{equation}
于是 $\mbf U$ 就是这一基变换的\textbf{过渡矩阵}。

\subsubsection{无限情况下的简单讨论}
以上的讨论都建立在 $\ms E$ 是一个有限维向量空间的基础之上,也就是说存在长度有限的基。但在无限维情况,乃至不可数无限维情况下,一些结论不仅形式会发生变化,还可能彻底失效。

首先,无限维情况下,对基能够“线性表示”所有右矢的要求变得模糊,完备性的要求应该作为替代。在内积空间中,我们专注于正交归一基,以下定义也适合于有限维:
\begin{definition}[正交归一完备基]\label{def:orthonormalcomplete}
根据 $\ms E$ 性质的不同,$\ms E$ 的正交归一完备基可能是可数的 $\ket{u_i}$,其中 $i$ 取自某个可数集合,一般默认为整数或其子集;或不可数的 $\ket{w_\alpha}$,其中 $\alpha$ 取自某个不可数集合,一般默认为实数或其中的某个区间。它们需要满足:
\begin{enumerate}
   \item 正交归一性
   \begin{align}
      \label{eq:orthonormaldis}
      &\braket{u_i|u_j}=\delta_{ij}\\
      \label{eq:orthonormalcon}
      &\braket{w_\alpha|w_{\alpha'}}=\delta(\alpha-\alpha')
   \end{align}
   \item 完备性
   \begin{align}
      \label{eq:completedis}
      &\sum_i\ket{u_i}\bra{u_i}=I\\
      \label{eq:completecon}
      &\int\d\alpha\ket{w_\alpha}\bra{w_\alpha}=I
   \end{align}
   其中 $I:\ms E\ra\ms E$ 是恒等算符。
\end{enumerate}
\end{definition}

此时本征系统仍能够张成整个空间的厄米算符拥有特称:

\begin{definition}[观测算符]\label{def:observable}
   $\ms E$ 上的算符 $A$ 是\bf{观测算符},是指其满足以下两条性质
   \begin{enumerate}
      \item $A$ 是厄米算符。

      \item 存在一组 $A$ 的本征矢构成的向量组构成了 $\ms E$ 的一组正交归一基 $\{\ket{u_n^i}:i=1,\cdots,g_n\}$
      
   \end{enumerate}
\end{definition}

同样地,C.S.C.O. 的含义也变为\textbf{对易观察算符完全集},其中的算符都是观察算符。

不可数无限维情况(或不严谨地简称为连续情况)下,许多求和号需要改成积分号。如对勾股定理 \ref{thm:Pythagorean} 的改造:

\begin{equation*}
   \|\psi\|=\int\d\alpha |\braket{w_\alpha|\psi}|^2 
\end{equation*}

所有插入完备性关系式得到的新公式的求和号也相应变为积分(一般情况下也忽略对所涉及无穷积分的一致收敛性的讨论),如内积的运算法则 \ref{eq:ipcommon}
\begin{equation}
    \braket{\vphi|\psi}=\int\d\alpha\braket{\vphi|w_\alpha}\braket{w_\alpha|\psi}=\int\d\alpha\braket{w_\alpha|\vphi}^*\braket{w_\alpha|\psi}
\end{equation}

再看表象:此时右矢的分量不应该再写为以 $i$ 为下标的数列,而是以 $\alpha$ 为自变量的函数值 $c(\alpha)=\braket{w_\alpha}{\psi}$,也即
\begin{equation}
    \ket{\psi} = \int\d\alpha\ket{w_\alpha}\braket{w_\alpha|\psi} = \int\d\alpha\ket{w_\alpha}c(\alpha)
\end{equation}

矩阵元也变为一个二元函数的值,即 $\braket{w_\alpha|A|w_{\alpha'}}=A(\alpha,\alpha')$。因此,右矢在表象下变为一个(一条纵轴上的)一元复值函数,左矢在表象下变为(一条横轴上的)一元复值函数,而算符在表象下变为一个(第一个变量在纵轴,第二个变量在横轴的)二元复值函数。$^*$ 关系依然是纵轴与横轴之间的共轭转置,矩阵相乘法则也做相应推广:如果记 $\braket{w_\alpha}{\vphi}=b(\alpha)$,则得到
\begin{equation}
   \braket{\vphi|A|\psi}=\Phi_\text{num}(\braket{\vphi|A|\psi}) = \Phi_\text{bra}(\bra{\vphi})\Phi_\text{op}(A)\Phi_\text{ket}(\ket{\psi})=\iint\d\alpha\d\alpha'b^*(\alpha)A(\alpha,\alpha')c(\alpha')
\end{equation}

基的变换公式则写为:另选一组正交归一基 $\{\ket{x_\alpha}\}$,则 $\ket{\psi}$ 在新基上的分量为
\begin{equation}
   c'(\alpha)=\braket{\psi|x_\alpha} = \int\d\alpha'\braket{\psi|w_{\alpha'}}\braket{w_{\alpha'}|x_\alpha}
\end{equation}

$A$ 在新基下的矩阵元为
\begin{equation}
   A'(\alpha,\alpha')=\braket{x_\alpha|A|x_{\alpha'}} = \int\d\beta\int\d\beta'\braket{x_\alpha|w_\beta}\braket{w_\beta|A|w_{\beta'}}\braket{w_{\beta'}|x_{\alpha'}}=\iint\d\beta\d\beta'\braket{x_\alpha|w_\beta}A(\beta,\beta')\braket{w_{\beta'}|x_{\alpha'}}
\end{equation}



\subsubsection{位置表象}

一个仍待考虑的问题是:我们为可数情况表象下的 $\C^n$ 设置了自然的正交归一完备基 $\{\bm e_i\}$,它与 $\{\ket{u_i}\}$ 确立的表象相对应。而当不可数情况下 $c(\alpha)$ 所在的空间已经变成一个函数空间 $\mc F$ 时,其自然产生的基它应该由一族函数构成,且每个函数用 $\alpha'$ 标记以与 $\ms E$ 的基中的一个 $\ket{w_{\alpha'}}$ 对应,记这样的基为 $\xi_{\alpha'}(\cdot)$,那么它的具体形式是什么?

为了揭示问题的物理意义,不妨将定义具体化,并在指标的维数上加以推广:把连续指标 $\alpha$ 设为\bf{位矢} $\bm r\in\R^3$(这里虽然将一维的数指标变成了三维的数组指标,但本质上不产生差异,它们都是不可数集合中的对象),把 $\ket{w_{\bm r}}$ 简记为 $\ket{\bm r}$。由于上述的 $c(\cdot)$ 的具体形式只与态矢 $\psi$ 有关,所以将其记为 $c(\bm r):=\psi(\bm r):\R^3\ra\C$,并称作以位矢为自变量的\bf{波函数}。

出于数学上的考虑,只有“足够正规”的波函数才能纳入函数(向量)空间 $\mc F$,此处的正规性至少要求 $L^2$ 黎曼可积。然后假设态空间 $\ms E_{\bm r}$ 通过 $\{\ket{\bm r}\}$ 确立的表象与 $\mc F$ 同构,这个表象使每个波函数 $\psi(\bm r)\in \mc F$ 与一个右矢 $\ket{\psi}$ 一一对应。回顾同构映射能以这样的形式表达:
\begin{equation}
   \Phi_\text{ket}(\ket{\psi})=\Phi_\text{ket}\left(\int\d\alpha c(\alpha)\ket{w_\alpha}\right)=\int\d\alpha' c(\alpha')\xi_{\alpha'}(\alpha)=c(\alpha)
\end{equation}
所以在此处的具体形式为
\begin{equation}
   \Phi_\text{ket}(\ket{\psi})=\Phi_\text{ket}\left(\int\d\bm r \psi(\bm r)\ket{\bm r}\right)=\int\d\bm r' \psi(\bm r')\xi_{\bm r'}(\bm r)=\psi(\bm r)
\end{equation}

这对任何满足要求的 $\psi(\cdot)$ 都成立。数学上的观察可知(或者说,根据定义),此时 $\xi_{\bm r_0}$ 的形式为 Dirac $\delta$ 函数
\begin{equation}
   \xi_{\bm r_0}(\bm r)=\delta(\bm r-\bm r_0)
\end{equation}
然而这个函数并不是有界的,因为我们知道
\begin{equation}
    \delta(\bm r) = \begin{cases}
        0, \bm r\ne 0\\
        \infty, \bm r=0
    \end{cases}
\end{equation}
因此不是平方可积的,这致使 $\xi_{\bm r_0}\notin\mc F$。但它确实能够以线性组合的方式表达 $\mc F$ 中的任何函数:它与 $\bm e_i$ 有异曲同工之妙,都只表示了自变量在某一点处函数的单位值而其他处为 0。此时不妨大胆地将其视作 $\mc F$ 之外的 $\mc F$ 的广义基,并且根据原有的类比,$\ms E_{\bm r}$ 的连续基 $\ket{\bm r}$ 与之通过同构映射 $\Phi_\text{ket}$ 相对应。仿照离散情况处理内积的手法,有以下形式推导
\begin{equation}
   \braket{\varphi|\psi} = \int\d\bm r\,\braket{\varphi|\bm r}\braket{\bm r|\psi}=\int\d\bm r\,\braket{\bm r|\varphi}^*\braket{\bm r|\psi}=\int\d\bm r\,\varphi(\bm r)^*\psi(\bm r)
\end{equation}

于是我们验证了,经过同构映射 $\Phi_\text{ket}$,$\ms E$ 空间的内积变成了函数空间 $\mc F$ 上的标准积分内积。还可以验证 $\xi_{\bm r}$ 在这种内积下的正交归一性
\begin{equation}
   \braket{\bm r_0|\bm r_0'}=\int\d\bm r\,\delta^*(\bm r-\bm r_0)\delta(\bm r-\bm r_0')=\delta(\bm r_0-\bm r_0')
\end{equation}

这无疑是很不平凡的结论,具有明确的物理意义。右矢 $\ket{\psi}$ 在 $\ket{\bm r_0}$ 上的分量是某个称为波函数的函数 $\psi(\bm r)$ 在位矢 $\bm r_0$ 处的取值。为此,我们不惜将 $\{\ket{\bm r_0}\}$ 称为\bf{广义右矢},它身在 $\ms E_{\bm r}$ 之外(如果它在 $\ms E_{\bm r}$ 之内,根据我们选定的 $\ms E_{\bm r}$ 和 $\Phi_\text{ket}$ 应该有 $\Phi_\text{ket}(\ket{\bm r_0})=\xi_{\bm r_0}(\bm r)\in \mc F$,产生了矛盾),却构成了 $\ms E_{\bm r}$ 的一组行之有效的基,因此确定了一个表象,称之为\bf{位置表象}。允许空间之外的态矢作为基表示空间之内的态矢,这也是无限维带来的奇特性质之一。

\subsubsection{动量表象}

再考虑一个非 $L^2$ 可积的函数
\begin{equation}
   v_{\bm k}(\bm r)=\left(\frac{1}{2\pi\hbar}\right)^{3/2}\e^{i\bm k\cdot\bm r}
\end{equation}
这是一个真正的“波函数”,描述了一个波矢为 $\bm k$ 的三维平面波。引入这一函数的动机在于:
\begin{itemize}
    \item 傅里叶变换指出 $\mc F$ 中的函数可以分解成平面波的线性组合,因此平面波是有特殊数学意义的基。
    \item 傅里叶正向变换的积分参数 $\bm k$ 代表平面波的波矢,而 de Broglie 关系式 \ref{pos:debrel} 将微观粒子的动量和物质波的波矢联系在一起,这赋予了平面波基以物理意义
\end{itemize}

因此提出动量为指标的平面波
\begin{equation}
   \label{eq:planewave}
   v_{\bm p}(\bm r)=\left(\frac{1}{2\pi\hbar}\right)^{3/2}\e^{i\bm p\cdot\bm r/\hbar}
\end{equation}
用广义右矢 $\ket{\bm p_0}$ 代表 $v_{\bm p_0}(\bm r)$,指数前的系数保证了验证正交归一性
\begin{equation}
   \braket{\bm p_0|\bm p_0'}=\left(\frac{1}{2\pi\hbar}\right)^3\int\d\bm r\e^{i(\bm p_0'-\bm p_0)\cdot\bm r/\hbar}=\delta(\bm p_0'-\bm p_0)
\end{equation}
然后就可以求出任意 $\psi\in\mc F$ 对应的 $\ket{\psi}\in\ms E_{\bm r}$ 与其内积
\begin{equation}
   \braket{\bm p_0|\psi}=\int\d\bm r\left(\frac{1}{2\pi\hbar}\right)^{3/2}\e^{-i\bm p_0\cdot\bm r/\hbar}\psi(\bm r)
\end{equation}
容易看出这就是 $\psi(\bm r)$ 的傅里叶换式的变种。这里对 $\mc F$ 中函数平方可积的限制又起到了作用,因为这是傅里叶变换存在的必要条件,且能够得到傅里叶积分:
\begin{align}
   \label{eq:fourierintegral}
   \psi(\bm r) & =\int\d\bm p_0\left(\frac{1}{2\pi\hbar}\right)^{3/2}\e^{i\bm p_0\cdot\bm r/\hbar}\int\d\bm r\left(\frac{1}{2\pi\hbar}\right)^{3/2}\e^{-i\bm p_0\cdot\bm r/\hbar}\psi(\bm r)\\
   \ket{\psi} & =\int\d\bm p_0\braket{\bm p_0|\psi}\ket{\bm p_0}
\end{align}
这就验证了这组广义右矢的完备性。因而 $\{\ket{p_0}\}$ 也是一组广义基,其确定了波函数的\bf{动量表象}。仿照 $\braket{\bm r_0|\psi}=\psi(\bm r_0)$ 的方式,我们能够定义动量表象下的波函数
\begin{equation}
   \label{eq:wavefuncp}
   \td{\psi}(\bm p)=\braket{\bm p|\psi}=\int\d\bm r\left(\frac{1}{2\pi\hbar}\right)^{3/2}\e^{-i\bm p\cdot\bm r/\hbar}\psi(\bm r)
\end{equation}

\section{物理构建}

以上我们处理了一些基本的数学结构,并定义了讨论量子力学问题所必须的符号。接下来将揭示量子力学核心的物理内容:物理系统的状态和演化的描述,以及“观测”行为的关键作用。

\subsection{基本公设}

量子力学的根基经常用一套公理体系示人,简述如下:

\begin{postulate}[态空间与右矢]\label{pos:state&ket}
   以时间 $t_0$ 为参数的右矢 $\ket{\varphi(t_0)}$ 可以描述一个孤立的物理系统在 $t_0$ 时刻的物理状态,它定义在复内积空间:\bf{态空间} $\ms E$ 中。
\end{postulate}

这条假设奠定了讨论物理系统状态的基础。时间变量是态空间的一个背景变量,我们既可以在固定的背景下讨论问题,即省略时间变量,隐式地取任意时刻 $t=t_0$;又可以讨论背景的变化引起的系统演化。这条假设也将一个抽象的数学结构具体化:态空间的向量空间结构中蕴含的加法封闭性,带来了\textbf{态叠加原理}。而在表象下,物理状态的描述又可以转移到函数空间中,如固定时刻 $t=t_0$ 时位置表象下的波函数 $\psi(\bm r;t_0)$,和考虑时间演化的含时波函数 $\Psi(r,t)$。

但态本身是不可言说的,即使在表象下,我们也难以将波函数与物理实在联系到一起。从实证的角度,首先应该关注的是能被我们测量的、描述系统状态在某一方面性质的物理量,以及物理量如何嵌入到态空间的结构中。

\begin{postulate}[物理量与算符]\label{pos:measurableop} 
   每个可观测的物理量 $\mc A$ 都是态空间 $\ms E$ 中的一个\bf{观测算符} $A$。
\end{postulate}

在用右矢描述状态后,这条假设又将可观测物理量的概念引入了现有系统。物理量的直观印象是一个实数,但算符不能用一个数来概括;算符天然是作用在右矢上的映射,接下来就要考虑这种映射的意义——它怎样生成我们所关心的物理量的取值。

\begin{postulate}[物理量的观测值]\label{pos:measurableval} 
   物理量 $\mc A$ 的所有可能的观测值都是其对应的观测算符 $A$ 的本征值之一。
   
\end{postulate}

这条假设真正体现了态空间作为内积空间的意义。首先根据谱定理 \ref{thm:spectrum},观测算符作为厄米算符,其所有本征值都是实数,这与我们对物理量的认识不谋而合。在此后的具体讨论中还会发现,它也为量子的概念正名:因为很多观测算符的本征值是离散的(可数的),经典物理认知中物理量的连续取值在量子力学的世界观中,经常要退化成阶梯状的离散取值,就如同从流水般的光束到分立的光子一样。

但态与观测值的关系仍尚不明了,一个自然的问题也就此产生:如果系统状态对应的态矢本身是 $A$ 的本征矢,则可以大胆推测这个态的观测值一定是对应的本征值;但如果态矢不是本征矢,根据观测算符的性质,我们只能将态矢分解到本征矢构成的基上,那么观测值怎样从这些本征矢对应的本征向量中选择呢?

\begin{postulate}[得到观测值的概率]\label{pos:measurablepr} 
   针对本征值的集合(谱)的结构,可以讨论以下两种最常见的情况:
   \begin{enumerate}
      \item \bf{(可能简并的)离散谱情况}
      
      如果系统处于归一化的态 $\ket{\psi}$ 中,对其物理量 $\mc A$ 的测量得到对应的观测算符 $A$ 的本征值 $a_n$ 的概率为
      \begin{equation}
         \P(a_n) = \sum_{i=1}^{g_n}|\braket{u_n^i|\psi}|^2
      \end{equation}
      其中 $g_n$ 是本征值 $a_n$ 的简并度,$\{\ket{u_n^i}\}$ 是张成 $a_n$ 对应的本征子空间 $\ms E_n$ 的正交归一基。

      \item \bf{非简并连续谱情况}
      
      如果系统处于归一化的态 $\ket{\psi}$ 中,对其物理量 $\mc A$ 的测量得到对应的观测算符 $A$ 的本征值区间 $\alpha$ 到 $\alpha+\d\alpha$ 的概率密度为
      \begin{equation}
         \d\P(\alpha) = |\braket{w_\alpha|\psi}|^2\d\alpha
      \end{equation}
      其中 $\{\ket{w_\alpha}\}$ 是 $\alpha$ 对应归一化本征矢。
      
   \end{enumerate}
   
\end{postulate}

因此,即使是多个本征值对应的本征矢的混合态在观测时也只能取到一个本征值。实际上,在观测发生时,系统的状态就不可避免地坍缩到了这个本征值之上,我们用投影的概念来描述这种坍缩。

\begin{definition}[投影算符]\label{def:project}
   若厄米算符有简并度为 $g_n$ 的本征值 $a_n$ 对应于本征矢 $\{\ket{u_n^i}\}$,则到 $a_n$ 所对应的本征子空间 $\ms E_n$ 的投影算符定义为
   \begin{equation}
    P_n=\sum_{i=1}^{g_n}\ket{u_n^i}\bra{u_n^i}
\end{equation}
由于不同的本征子空间是完全正交的,这一部分实际上是 $\ket{\psi}$ 到子空间的正交投影。
\end{definition}

\begin{postulate}[观测对系统的影响]\label{pos:measurablestate} 
   假设系统观测前的状态对应 $\ket{\psi}$,若一次观测得到了物理量 $\mc A$ 对应的观测算符 $A$ 的本征值 $a_n$,则系统在观测后的状态变化为
   \begin{equation*}
         \frac{P_n\ket{\psi}}{\braket{\psi|P_n|\psi}}
      \end{equation*}
   其中 $P_n$ 是到 $a_n$ 所对应的本征子空间 $\ms E_n$ 的投影算符。
\end{postulate}

至此,对某个确定的物理状态的描述和观测已经完善。接下来将时间参量 $t_0$ 考虑为时间变量 $t$,来描述系统的演化。

\begin{postulate}[薛定谔(Schrödinger)方程]\label{pos:Seq} 
   态(右)矢 $\ket{\psi(t)}$ 随时间的演化遵循以下的 \bf{Schrödinger 方程}:
   \begin{equation}
      \i\hbar\frac{\d}{\d t}\ket{\psi(t)}=H(t)\ket{\psi(t)}
   \end{equation}
   其中 $H(t)$ 是系统在态空间中的那部分能量对应的观测算符,称为 \bf{哈密顿(Hamilton)算符}
   
\end{postulate}

这是一个对时间 $t$ 的一阶微分方程。虽然这是出现在态空间上的方程但取表象之后就会转换为常见的函数空间中的微分方程,从而可以被熟悉的方式求解。因此如果确定了初值条件 $\ket{\psi(t_0)}$,态 $\ket{\psi(t)}$ 的时间演化也随之完全确定。然而其中关键的哈密顿算符的构造要考虑到系统的力学性质,因此需要对最基本的力学量算符有具体的认识。

\subsection{力学描述}

经典力学的核心变量是位置和动量,它们共同组成了质点在\textbf{相空间}中的广义坐标。哈密顿力学是经典力学的描述形式之一,它给出了相空间中的轨迹方程,因而定义了一种力学。而量子力学中,物理量除了时间都被算符化,因此首先要找到位置和动量的算符才能建立类似的力学。

\subsubsection{位置算符}
如果将视野放入具有物理意义的态空间 $\ms E_{\bm r}$ 中,利用先前的同构关系 $\Phi$ 及其连带的一系列性质,我们可以通过研究位置表象下基上的分量,即波函数 $\psi(\bm r)$ ,来研究态矢 $\ket{\psi}$。因为此处的 $\bm r$ 的含义是位矢,所以自然想到应该存在某个算符 $\bm R$ 对应着位矢这一物理量。然而位矢实际上具有三个独立的分量,即 $\bm r=(x,y,z)$,可以分开处理,即寻找算符 $X,Y,Z\in\ms E_{\bm r}$,它们分别对应 $x,y,z$ 即位矢的坐标分量。将这三个算符以某种方式合并在一起,才得到矢量算符 $\bm R$。根据三个坐标的对称性,我们只需对 $X$ 进行分析。

接下来构造算符 $X$,确定了算符在基上的行为就确定了算符的一切。我们希望它在每一个基 $\ket{\bm r_0}$ 上表现为:$X\ket{\bm r_0}=x_0\ket{\bm r_0}$,也即基是它的本征向量,而本征值就是 $\ket{\bm r}$ 中 $\bm r$ 对应的坐标 $x$,这样才能说该算符代表位置这一物理量。然后我们参考正交归一关系式 \ref{eq:orthonormalcon} 写出矩阵元:
\begin{equation}
   \braket{\bm r_0'|X|\bm r_0}=x_0\braket{\bm r_0'|\bm r_0}=\delta(\bm r_0'-\bm r_0)
\end{equation}
由于 $\bm r_0'\ne\bm r_0$ 时矩阵元都为 $0$,因此这是一个对角“矩阵”,从而显然满足共轭对称性 $X=X^*$,即 $X$ 是厄米算符。而且其对应于不同本征值 $\bm r_0$ 的本征向量组 $\ket{\bm r_0}$ 确实是 $\ms E_{\bm r}$ 的一组基,因而 $X$ 是观测算符。

接下来考虑 $X$ 在任意 $\ket{\psi}$ 上的作用:由于 $\ket{\psi}$ 不一定是本征向量,我们还需要把 $X\ket{\psi}$ 投影到基上来观察:此处再次用到插入完备性关系式 \ref{eq:completecon} 的技巧。
\begin{align*}
   \braket{\bm r_0|X|\psi}&=\int\d\bm r\braket{\bm r_0|X|\bm r}\braket{\bm r|\psi}=\int\d\bm r x\braket{\bm r_0|\bm r}\braket{\bm r|\psi}\\
   & =\int\d\bm r x\delta(\bm r_0-\bm r)\psi(\bm r)=x_0\psi(\bm r_0)
\end{align*}
以上关系总结为用波函数表达的关系:在表象下的波函数空间 $\mc F$ 上,我们能找到与 $X$ 对应的函数变换 $\hat X$,也称其为算符,其性质为
\begin{equation}
   \label{eq:posopwavfuncx}
   \braket{\bm r|X|\psi}=x\psi(\bm r)=\hat X\psi(\bm r)
\end{equation}

对于坐标 $y,z$ 也这样构造出 $Y,Z$,然后形式上令算符 $\bm R=(X,Y,Z)$,则每个 $\ket{\bm r_0}=\ket{x_0,y_0,z_0}$ 都是其本征向量,对应于本征值组 $\bm r_0 = (x_0,y_0,z_0)$。根据假设 \ref{pos:measurableval},位矢的所有可能的观测值就这样成为了观测算符 $\bm R$ 的所有本征值组。而关系式 \ref{eq:posopwavfunc} 也被数组化。
\begin{equation}
   \label{eq:posopwavfunc}
   \braket{\bm r|\bm R|\psi}=\bm r\psi(\bm r)=\hat{\bm R}\psi(\bm r)
\end{equation}

\subsubsection{位置表象下的动量算符}
我们在位置表象下完成了上述讨论,动量表象下依然可以完成这样的构造,得到一个分量上的算符 $P_x$,和矢量算符 $\bm P=(P_x,P_y,P_z)$,使得动量的所有可能的观测值成为了观测算符 $\bm P$ 的所有本征值组。考虑形如 $\ket{\bm p_0}$ 的基,我们采用以下关系
\begin{align}
   P_x\ket{p_0}&=p_{x0}\ket{p_0}\\
   \braket{\bm p_0'|\bm p_0}&=\delta(\bm p_0'-\bm p_0)\\
   \braket{\bm p|P_x|\psi}&=p_x\td\psi(\bm p)\\
   \braket{\bm p|\bm P|\psi}&=\bm p\td\psi(\bm p)
\end{align}

此时根据前面对位置和动量的讨论,应该确立以下公设来支撑以下的推导
\begin{postulate}[位置与动量本征态的关系]
    系统的两个可观测物理量:位置 $\bm r$ 和动量 $\bm p$ 分别对应于位置算符 $\bm X$ 和动量算符 $\bm P$。它们分别有连续本征矢 $\{\ket{\bm r}\}$ 和 $\{\ket{\bm p}\}$,而这两套本征矢之间的关系体现为:
    \begin{equation}
        \braket{\bm r|\bm p} = \left(\frac 1{2\pi\hbar}\right)\e^{-i\bm p\cdot\bm r/\hbar}
    \end{equation}
\end{postulate}

注意此时尚未处理好 $\mc F$ 空间中的算符 $\hat{\bm P}$,因为 $\td\psi(\bm p)$ 并不是位置表象下的波函数。为了得到 $\hat{\bm P}$ 的在位置表象下的形式,应该求解经 $\bm P$ 作用后的 $\ket{\psi}$ 的位置表象,即:
\begin{equation*}
\hat{\bm P}\psi(\bm r) = \braket{\bm r|\bm P|\psi}
\end{equation*}为了将其转化为已经处理过的问题,再次调用完备性关系式 \ref{eq:completecon} 反复插入恒等算符,
\begin{align*}
   \braket{\bm r|\bm P|\psi} &= \int\d\bm p\braket{\bm r|\bm P|\bm p}\braket{\bm p|\psi}\\
   &=\int\d\bm p\braket{\bm r|\bm P|\bm p}\td\psi(\bm p)\\
   &=\int\d\bm p'\int\d\bm p\braket{\bm r|\bm p'}\braket{\bm p'|\bm P|\bm p}\td\psi(\bm p)\\
   &=\int\d\bm p'\int\d\bm p\braket{\bm r|\bm p'}\bm p\braket{\bm p'|\bm p}\td\psi(\bm p)\\
   &=\int\d\bm p'\braket{\bm r|\bm p'}\bm p'\td\psi(\bm p)
\end{align*}
根据位置表象下的 $\ket{\bm p}$,得到
\begin{equation}
   \braket{\bm r|\bm p'}=\int\d\bm r'\,\delta^*(\bm r-\bm r')\left(\frac{1}{2\pi\hbar}\right)^{3/2}\e^{i\bm p'\cdot\bm r'/\hbar}=\left(\frac{1}{2\pi\hbar}\right)^{3/2}\e^{i\bm p'\cdot\bm r/\hbar}
\end{equation}
再代入动量表象波函数与位置表象波函数的关系 \ref{eq:wavefuncp},得到
\begin{equation}
   \label{eq:momopwavfunc1}
   \braket{\bm r|\bm P|\psi}=\int\d\bm p'\,\left(\frac{1}{2\pi\hbar}\right)^{3/2}\e^{i\bm p'\cdot\bm r/\hbar}\bm p'\int\d\bm r\,\left(\frac{1}{2\pi\hbar}\right)^{3/2}\e^{-i\bm p'\cdot\bm r/\hbar}\psi(\bm r)
\end{equation}
先考虑积分的后半段,采用分部积分法:
\begin{align*}
   \int\d\bm r\,\e^{-i\bm p'\cdot\bm r/\hbar}\psi(\bm r) &= \frac{\i\hbar}{\bm p'}\int\d\bm r\,\nabla_{\bm r}\e^{-i\bm p'\cdot\bm r/\hbar}\psi(\bm r)\\
   &=\frac{\i\hbar}{\bm p'}\left(\left.\psi(\bm r)\e^{-i\bm p'\cdot\bm r/\hbar}\right|_{-\infty}^{\infty}-\int\nabla_{\bm r}\psi\e^{-i\bm p'\cdot\bm r/\hbar}\right)\\
   &=\frac{\i\hbar}{\bm p'}\left(0-\int\nabla_{\bm r}\psi\e^{-i\bm p'\cdot\bm r/\hbar}\right)\\ 
   &=\frac{\hbar}{\i\bm p'}\int\d\bm r\nabla_{\bm r}\psi\e^{-i\bm p'\cdot\bm r/\hbar}
\end{align*}

其中第 3 行利用了 $\psi(\bm r)$ 作为 $\mc F$ 空间中的函数的平方可积要求,它在无穷远点的值一定是趋于 0 的,因此会消去边界项。将这一结果代回式 \ref{eq:momopwavfunc1},得到
\begin{align*}
   \bra{\bm r}\bm P\ket{\psi}&=\int\d\bm p'\left(\frac{1}{2\pi\hbar}\right)^{3/2}\e^{i\bm p'\cdot\bm r/\hbar}\bm p'\frac{\hbar}{\i\bm p'}\int\d\bm r\left(\frac{1}{2\pi\hbar}\right)^{3/2}\nabla_{\bm r}\psi\e^{-i\bm p'\cdot\bm r/\hbar}\\
   &=\frac{\hbar}{\i}\int\d\bm p'\left(\frac{1}{2\pi\hbar}\right)^{3/2}\e^{i\bm p'\cdot\bm r/\hbar}\int\d\bm r\left(\frac{1}{2\pi\hbar}\right)^{3/2}\nabla_{\bm r}\psi\e^{-i\bm p'\cdot\bm r/\hbar}
\end{align*}
根据傅里叶积分 \ref{eq:fourierintegral} 可得最终的表达式:
\begin{equation}
    \hat{\bm P}\psi(\bm r) = \bra{\bm r}\bm P\ket{\psi}=\frac{\hbar}{\i}\nabla_{\bm r}\psi(\bm r)
\end{equation}

对于一个坐标分量,自然有
\begin{equation}
   \hat{P}_x\psi(\bm r) = \bra{\bm r}P_x\ket{\psi}=\frac{\hbar}{\i}\frac{\p }{\p x}\psi(\bm r)
\end{equation}

\subsubsection{量子化规则}

在哈密顿经典力学体系下,我们习惯用坐标和动量来描述力学问题,将其称为\textbf{正则变量},其他力学量通常表示为坐标和动量的函数 $\bm{\mc A}=\bm{\mc A}(\bm r,\bm p, t)$。而量子力学框架下,力学量表现为算符,于是树立以下规则:$\bm{\mc A}$ 对应的算符
\begin{equation}
    \label{eq:quantization}
   A(t)=\bm{\mc A}(\bm R, \bm P, t)
\end{equation}

此处出现的第一个问题是:我们在 $\mc L(\ms E)$ 中定义了算符的加法、数乘和复合,构成了一个环结构。对于多项式函数 $p$,由于多项式环的存在,可以通过自然同构,将多项式原本的自变量替换为算符 $\bm A$,得到良定义的算符 $p(\bm A)$。然而对于任意函数,即使像 $f(x)=1/x$ 那样简单,在算符的环中也是未定义的。但只要是解析函数,就可以在全平面用幂级数展开处理
\begin{equation}
    f(z)=\sum_{n=0}^\infty a_nz^n
\end{equation}
然后作以下定义:
\begin{equation}
    f(A)=\sum_{n=0}^\infty a_nA^n
\end{equation}
这就相当于用无穷维的算符多项式,或称算符项级数,定义了算符的(解析)函数。多项式函数是解析函数,其幂级数展开就是其本身,与定义前的认识一致。但这个级数的收敛问题需具体问题具体分析,因为它与算符的本征值和幂级数本身的收敛半径相关。

如果算符 $A$ 是厄米的且 $f(z)$ 是实函数,容易验证 $f(A)$ 也是厄米的。如果 $\ket{\psi_\lambda}$ 是 $A$ 的本征值 $\lambda$ 对应的本征矢,则代入定义 ,就得到了算符的函数的本征值
\begin{equation}
    \label{eq:opfunceigenval}
    f(A)\ket{\psi_\lambda}=\sum_{n=0}^\infty a_n\lambda^n\ket{\psi_\lambda}=f(\lambda)\ket{\psi_\lambda}
\end{equation}
于是如果 $A$ 是观测算符,则其本征矢同时也是 $f(A)$ 的本征矢,可以期待 $f(A)$ 也是观测算符,即可以代表所需的物理量。

此处出现的第二个问题是:不能保证所得算符总是厄米算符。如
\begin{equation}
   \bm{\mc A}=\bm r\cdot\bm p\implies A=\bm R\cdot\bm P
\end{equation}
然而
\begin{equation}
   (\bm R\cdot\bm P)^*=\bm P^*\cdot\bm R^*=\bm P\cdot\bm R
\end{equation}
这里矢量算符的点积即是
\begin{equation}
   \bm R\cdot\bm P=XP_x+YP_y+ZP_z
\end{equation}
然而,取位置表象稍作计算,就能验证位置和动量算符之间的对易关系
\begin{theorem}[正则对易关系]\label{thm:commuterp}
\begin{align}
    [R_i,R_j]&=0\\
    [P_i,P_j]&=0\\
    [R_i,P_j]&=\i\hbar\delta_{ij}
\end{align}
其中 $i,j\in{x,y,z}$
\end{theorem}
这意味着 $\bm R\cdot \bm P\ne \bm P\cdot \bm R$,因此所构造的 $A$ 是非厄米的。为了规避此类经典力学中不会出现的问题,还要引入对称化规则,其基本精神是,修改 $A$ 为:
\begin{equation}
   A=\frac 1 2(\bm R\cdot\bm P+\bm P\cdot\bm R)
\end{equation}

考虑最简单的情况,一个粒子在不含时、只与位置有关的标量势场中运动,则系统的总能量写作动能与势能之和,也即哈密顿量:
\begin{equation}
   \mc H(\bm r, \bm p)=\frac{\bm p^2}{2m}+V(\bm r)
\end{equation}
其中 $V(\bm r)$ 需要代入具体的势能函数。则根据上述量子化规则,此时哈密顿算符和薛定谔方程为
\begin{align}
   H&=\frac{\bm P^2}{2m}+V(\bm R)\\
   \i\hbar\frac{\d}{\d t}\ket{\psi(t)}&=\left[\frac{\bm P^2}{2m}+V(\bm R)\right]\ket{\psi(t)}
\end{align}

\subsection{概率诠释}

量子力学在观念上的巨大突破之一就是对决定论的否认。公设~\ref{pos:measurablepr} 中内禀的随机性之起源仍然是个谜,掷骰子的上帝位于何处?概率诠释仍不够解答人们对量子力学的困惑。然而此处为了以化学问题为重,不得不采取“闭嘴计算”的态度,但仍要在现有框架下深挖一些基本概念。

\subsubsection{概率的良定义}
系统的某个物理量的观测值对应的概率是态空间-态矢-算符的唯一物理意义,但此前没有讨论的是当一个本征值有大于 $1$ 的简并度时,在它的本征子空间内,正交归一基可以任意选择。由于公设 \ref{pos:measurablepr} 的概率算式涉及具体的基,我们需要证明,这个概率与基的选择无关。以离散谱情况为例,考虑
\begin{equation}
    \ket{\psi_n}=\sum_{i=1}^{g_n}\ket{u_n^i}\braket{u_n^i|\psi}=P_n\ket{\psi}
\end{equation}

投影算符的重要性质是 $P_n^2=P_n$,直观上讲,即投影两次和投影一次的效果相同,或者 $P_n$ 在 $\ms E_n$ 上的限制实际上是恒等算符。还能够证明 $P_n$ 是厄米算符。此时观察到
\begin{equation}
    \P(a_n)=\braket{\psi|P_n^*P_n|\psi}=\braket{\psi|P_n^2|\psi}=\braket{\psi|P_n|\psi}
\end{equation}

因此,概率 $\P(a_n)$ 实际上是投影算符 $P_n$ 的矩阵元,它只与 $P_n$ 本身有关,也就是和 $a_n$ 的本征子空间本身有关,而与这个子空间内部基的选择无关。

如果用非归一化的态 $\ket{\psi}$ 来描述系统,则所得的概率还应该除以归一化因子 $|\braket{\psi|\psi}|^2$,这是概率归一性的必然要求。由于归一化因子的存在,对 $\lambda\text e^{\i\theta}\ket{\psi}$ 和 $\ket{\psi}$ 所做的观测得到的结果在概率上完全相同,因此在描述整个系统的态矢上添加全局系数是没有物理意义的。因此以后提到“\textbf{唯一确定}”某个态矢的时候,可以在相差一个系数的意义下理解,也就是确定了一个一维子空间就确定了这个态。但是,如果系统被描述为多个态的混合,则任意两个态前系数的\textbf{模长之比}与\textbf{相角之差}具有物理意义。

既然概率与基无关,就可以在不同表象下用波函数来计算概率。从位置表象来看,任意的位矢 $\bm r$ 都是位置算符 $\bm R$ 的本征值,其本征矢为 $\ket{\bm r}$。因此对于一个归一化的态 $\psi$,所描述的系统位于 $\bm r$ 到 $\bm r+\d\bm r$ 的概率为
\begin{equation}
    \d\P(\bm r)=|\braket{\bm r|\psi}|^2\d\bm r=|\psi(\bm r)|^2\d\bm r
\end{equation}

同理分析动量表象下的动量算符 $\bm P$ 的本征值和本征矢,可知所描述的系统的动量在 $\bm p$ 到 $\bm p+\d\bm p$ 的概率为
\begin{equation}
    \d\P(\bm p)=|\braket{\bm p|\psi}|^2\d\bm p=|\td\psi(\bm p)|^2\d\bm p
\end{equation}

\subsubsection{观测值的均值与方差}
观测算符的本征矢和被观测系统的态矢共同决定了观测值的概率分布,因此可以讨论这一概率分布的一些重要特征:

首先考虑观测量 $A$ 的均值,假设其离散的本征值为 $\{a_n\}$ 对应于本征矢 $\{u_n^i\}$,对于归一化的态 $\ket{\psi}$ 有:
\begin{align*}
    \Braket{A}_\psi &= \sum_n a_n\P(a_n)=\sum_n a_n\sum_{i=1}^{g_n}|\braket{\psi|u_n^i}|^2\\
    &=\sum_n a_n\sum_{i=1}^{g_n}\braket{\psi|u_n^i}\braket{u_n^i|\psi}\\
    &=\sum_n \sum_{i=1}^{g_n}\braket{\psi|a_n|u_n^i}\braket{u_n^i|\psi}\\
    &=\sum_n \sum_{i=1}^{g_n}\braket{\psi|A|u_n^i}\braket{u_n^i|\psi}\\
    &=\braket{\psi|A\left[\sum_n\sum_{i=1}^{g_n}\ket{u_n^i}\bra{u_n^i}\right]|\psi}=\braket{\psi|A|\psi}
\end{align*}

对于连续情况下的本征矢 $\{w_\alpha\}$,也类似有
\begin{equation}
    \Braket{A}_\psi = \int\alpha|\braket{w_\alpha|\psi}|^2\d\alpha=\braket{\psi|A|\psi}
\end{equation}

如果态未经归一化,则根据概率归一化原则,所得结果还要除以 $|\braket{\psi|\psi}|^2$。下标 $\psi$ 是为了特指这一均值是与对具体的态 $\psi$ 的测量联系在一起的。

观测量 $A$ 的\textbf{不确定度}即是标准差,可以用均值写作 $\Delta A=\sqrt{\Braket{A-\Braket{A}}^2}$,概率论上的推导也表明
\begin{equation}
    \Delta A = \sqrt{\Braket{A^2}-\Braket{A}^2}
\end{equation}

\subsubsection{不确定度关系式}
量子力学不仅在单个物理量上引入了随机性,还限制了我们在一个态的一组物理量上同时获得任意精确的观测结果。

\begin{theorem}[不确定度关系式]\label{def:indefinite}
   给定系统的任意一个态 $\ket{\psi}$,则两个观测量 $A,B$ 在其上的不确定度满足以下关系
   \begin{equation}
       \Delta A\cdot\Delta B\ge\frac{|\Braket{[A,B]}|}2
   \end{equation}
\end{theorem}

\begin{proof}
    考虑一个右矢 $\ket{\vphi}=(A+\i\lambda B)\ket{\psi}$,其中 $\lambda\in\R$,根据模方的正定性,有
    \begin{align*}
        \braket{\vphi|\vphi}& =\bra{\vphi}(A-\i\lambda B)(A+\i\lambda B)\ket{\psi}\\
        &=\bra{\psi}A^2\ket{\psi}+\bra{\psi}(\i\lambda AB-\i\lambda BA)\ket{\psi}+\bra{\psi}\lambda^2B^2\ket{\psi}\\
        &=\Braket{A^2}+\i\lambda\Braket{[A,B]}+\lambda^2\Braket{B^2}\ge 0
    \end{align*}
    此时亦可知 $\Braket{[A,B]}$ 只能为纯虚数或零,否则这个不等式将失去意义。于是考虑所得二次式的判别式,得到
    \begin{equation*}
        |\Braket{[A,B]}|^2-4\Braket{A^2}\Braket{B^2}\le 0
    \end{equation*}
    再引入两个辅助观测量 $A'=A-\Braket{A},B'=B-\Braket{B}$,然后容易发现 $[A',B']=[A,B]$,因此
    \begin{equation*}
        |\Braket{[A',B']}|^2-4\Braket{A'^2}\Braket{B'^2}\le 0
    \end{equation*}
    根据不确定度的定义,直接推出
    \begin{equation*}
        \sqrt{\Braket{A'^2}\Braket{B'^2}}=\Delta A\Delta B\ge \frac{|\Braket{[A,B]}|}2
    \end{equation*}
\end{proof}

根据基本公理 \ref{pos:measurablestate},观测一定会影响系统状态。因而这一关系描述的实际上是对一个系统的两个物理量同时进行观测的状况:如果两个物理量对应的观测算符不对易,次序不同的观测将带来不同的结果,从这一角度也能够理解这两个物理量不可同时被任意精确地测量。

特别地,若两个物理量对应的观察算符 $Q,P$ 的对易子 $[Q,P]=\i\hbar$,则称 $Q,P$ 为\textbf{共轭}的物理量。从基本对易关系 \ref{thm:commuterp} 可知,在同一方向上的坐标和动量分量就是共轭物理量。根据以上结论,共轭物理量的不确定度满足
    \begin{equation}
        \Delta Q\Delta P\ge\frac\hbar 2
    \end{equation}

因而粒子的位置和动量不可同时被确定。由于 $\hbar/2$ 的数量级极小,在宏观物体上这种测量的不确定性不明显,从而符合经典物理的认知。在三维位置和动量构成的\textbf{相空间}中,宏观质点的运动状态可以用一个六维向量来描述。然而对于微观粒子而言,我们因此不再拥有这样描述的权利。

\section{基本模型}

对于一个不随时间变化(即不显含 $t$ )的哈密顿算符。如果其具有离散本征值 $\{E_n\}$ 作为系统的总能量,则它们对应着本征矢 $\{\ket{u_{n,\tau}}\}$,其中 $\tau$ 是包括 $H$ 的一组 C.S.C.O. 中对应其他算符的本征值的标记,用于唯一确定本征矢。由于 $H$ 是观察算符,任意态矢 $\ket{\psi}$ 都可以分解到这些本征矢上;又因为 $H$ 不含时,本征系统固定,但 $\ket{\psi}$ 随时间演化,这是由于分解的系数随时间变化,即
\begin{equation}
   \ket{\psi(t)}=\sum_n\sum_\tau c_{n,\tau}(t)\ket{u_{n,\tau}},\ c_n(t)=\braket{u_{n,\tau}|\psi(t)}
\end{equation}

如果将薛定谔算符投影到各个本征矢上,则得到
\begin{equation}
   \i\hbar\frac{\d}{\d t}\braket{u_{n,\tau}|\psi(t)}=\braket{u_{n,\tau}|H|\psi(t)}
\end{equation}
根据 $H$ 的厄米性质,可知
\begin{equation}
   \bra{u_{n,\tau}}H=(H^*\ket{u_{n,\tau}})^*=(H\ket{u_{n,\tau}})^*=(E_n\ket{u_{n,\tau}})^*=E_n^*\bra{u_{n,\tau}}=E_n\bra{u_{n,\tau}}
\end{equation}
因此上式实际上是
\begin{equation}
   \i\hbar\frac{\d}{\d t}c_{n,\tau}(t)=E_nc_{n,\tau}(t)
\end{equation}
使其从向量空间 $\ms E$ 上的微分方程变为 $\C$ 上的微分方程,于是可结合初值条件解之:
\begin{equation}
   c_{n,\tau}(t)=c_{n,\tau}(t_0)\e^{-\i E_n(t-t_0)/\hbar}
\end{equation}
这也就是说
\begin{equation}
   \ket{\psi(t)}=\sum_n\sum_{\tau}c_{n,\tau}(t_0)\e^{-\i E_n(t-t_0)/\hbar}\ket{u_{n,\tau}}
\end{equation}
容易推广到 $H$ 有连续谱的情况
\begin{equation}
   \ket{\psi(t)}=\int\d E\sum_{\tau}c_{\tau}(E,t_0)\e^{-\i E(t-t_0)/\hbar}\ket{w_{E,\tau}}
\end{equation}

如果初态恰好在 $H$ 的某个本征子空间内,则演化的结果就是在给态添加了全局含时相位因子,根据此前的概率诠释,这种变化在物理上是不可区分的。因此,在 $H$ 的某个本征子空间内的态称为\textbf{定态}。但是,不同能量的定态随时间演化的频率不同,因此不同能量的定态混合时,将产生随时间变化的有意义的物理结果。

另外,将时间变量剥离后,求解此类薛定谔方程的核心就变成了由不同形式的 $H$ 决定的本征值问题,即\textbf{定态薛定谔方程} $H\ket{\psi}=E\ket{\psi}$。

以下的模型正是一些基本定态问题的求解。为了方便起见可以在位置表象下处理方程
\begin{equation}
    \left(\frac{\bm P^2}{2m}+V(\bm R)\right)\ket{\psi}=E\ket{\psi}
\end{equation}

注意到其中 $V(R)$ 作用于态矢后在位置表象下的行为
\begin{align}
    \label{eq:potfuncpos}
    \braket{\bm r|V(\bm R)|\psi} &= \int\d\bm r'\,\braket{\bm r|V(\bm R)|\bm r'}\braket{\bm r'|\psi} = \int\d\bm r'\,V(\bm r')\braket{\bm r|\bm r'}\psi(\bm r')\\&=\int\d\bm r'\,V(\bm r')\delta(\bm r'-\bm r)\psi(\bm r')=V(\bm r)\psi(\bm r)
\end{align}

从而得到微分方程

\begin{equation}\label{eq:sttnry_stt_schrdngr_eqtn}
    \nabla^2\psi(\bm r)+\frac{2m}{\hbar^2}[E-V(\bm r)]\psi(\bm r)=0
\end{equation}

对于不同的模型,区别主要在于哈密顿算符 $H$ 中势能项 $V(\bm R)$ 的形式差异。而这种差异给定态问题的解法带来了极为丰富多彩的变化,例如有时不必轻易取表象,而在态空间上先进行分析。其中产生的一些技巧蕴涵着深刻的物理意义。我们也将看到,这是一类观测算符本征值问题——它往往能得到所关注体系的新表象,因此首要的目标是得到以下信息:
\begin{enumerate}
    \item 哈密顿量 $H$ 以及其他观测算符的谱:这关系到系统可能取到各物理量的观测值,以及与经典力学体系的最主要区别。
    \item 本征矢之间的关系和本征值的简并度:这关系到这些右矢如何构成一组新表象,以及如何在新表象下剖析各个观测算符。
    \item 位置表象(或动量表象)下的本征矢:也即波函数,这关系到如何将旧表象和新表象联系在一起,体系在位形(或动量)空间中的概率分布以及与经典力学体系的联系。
\end{enumerate}

\subsection{方形势}

方形势是指这样一种理想模型:在位置表象下,哈密顿算符中的势能项对应的势能函数 $V(\bm r)$ 是按照粒子所在空间的直角坐标下的各坐标分量区间划分的\textbf{分片常函数}。一维情况下,待求解的二阶线性常微分方程是:
\begin{equation}\label{SEq1dSqrPotntl}
    \frac{\d^2}{\d x^2}\psi(x)+\frac{2m}{\hbar^2}[E-V(x)]\psi(x)=0
\end{equation}

方形势问题的一个重要之处在于:借用微积分的思想,如果某个非方形势在给定区间的空间变化足够平缓,就可以将其视作分片常函数来处理。因此首先需要知道分片常函数各种情况下解的性质,以及不同片之间的解如何衔接。

\subsubsection{常值区域与间断点}
在实轴上的不同区间内,$V(x)$ 取到不同的值,退化成二阶线性常微分方程;而在各区间的端点处,$V(x)$ 呈现出不连续性。

首先考虑 $V(x)\equiv V$ 的区域内部,不顾边界条件对能量本征值 $E$ 的限制,从最普遍的情况求解方程,于是可知对于任意的 $E$,都能求出波函数的通解;作为二阶常微分方程的通解,它有两个待定系数。而作为齐次方程,它的通解 $\psi(x)\equiv 0$ 没有概率密度可言,因此不能表示一个有物理意义的态,应该被舍去。我们只关心以下非平凡解: 

\begin{itemize}
    \item $E>V$ 的情况
    \begin{equation}\label{SEq1dSqrPotntlSlt1}
        \psi(x)=A_1\e^{\i kx}+A_2\e^{-\i kx},\ k=\frac{\sqrt{2m(E-V)}}{\hbar}
    \end{equation}
    \item $E<V$ 的情况
    \begin{equation}\label{SEq1dSqrPotntlSlt2}
        \psi(x)=B_1\e^{\rho x}+B_2\e^{-\rho x},\ \rho=\frac{\sqrt{2m(V-E)}}{\hbar}
    \end{equation}
    \item $E=V$ 的情况
    \begin{equation}\label{SEq1dSqrPotntlSlt3}
        \psi(x)=C_1x+C_2
    \end{equation}
\end{itemize}

而在 $V$ 不同的区域交界处,我们希望知道波函数怎样受到势能的不连续性影响,换言之,在这些奇点处波函数的行为如何。当 $V(x)$ 连续时,可以期待方程 $\ref{SEq1dSqrPotntl}$ 有光滑的解。但如果出现一个间断点 $x_1$,我们仍可以构造连续函数 $V_\varepsilon(x)$,其中 $\varepsilon$ 是一个充分小的正数,且在区间 $[x_1-\varepsilon,x_1+\varepsilon]$ 之外,都有 $V_\varepsilon(x)=V(x)$。$V_\varepsilon(x)$ 对应的波函数为 $\psi_\varepsilon(x)$,也即

\begin{equation}
    \frac{\d^2}{\d x^2}\psi_\varepsilon(x)+\frac{2m}{\hbar^2}[E-V_\varepsilon(x)]\psi_\varepsilon(x)=0
\end{equation}

于是可以将 $V(x)$ 和 $\psi(x)$ 分别视作 $V_\varepsilon(x)$ 和 $\psi_\varepsilon(x)$ 在 $\varepsilon\rightarrow 0$ 时的极限,直接对该方程在 $[x_1-\eta,x_1+\eta]$ 上积分,得到

\begin{equation}
    \left.\frac{\d}{\d x}\psi_\varepsilon(x)\right|_{x=x_1+\eta}-\left.\frac{\d}{\d x}\psi_\varepsilon(x)\right|_{x=x_1-\eta}=\frac{2m}{\hbar^2}\int_{x_1-\eta}^{x_1+\eta}[V_\varepsilon(x)-E]\psi_\varepsilon(x)\d x
\end{equation}

考虑波函数 $\psi_\varepsilon(x)$ 的概率意义,它在任何区间内都应该是有界的。因此在同时取 $\varepsilon,\eta\ra 0$ 的极限时,右端随着积分区间的紧缩取到 $0$,左端则显示 $\psi(x)$ 在 $V(x)$ 的间断点处是连续可微的。

因此,解决一维的方形势问题的基本方法就是:在不同的区间内解方程 \ref{SEq1dSqrPotntl},简化为二阶常系数线性常微分方程问题,根据不同的情况可以得到三类解 \ref{SEq1dSqrPotntlSlt1}、\ref{SEq1dSqrPotntlSlt2}、\ref{SEq1dSqrPotntlSlt3}。然后考虑以下边界条件:
\begin{itemize}
    \item 势能间断点处波函数连续
    \item 势能间断点处波函数的导数连续
    \item 波函数在无穷远处趋于0(或波函数平方可积)
\end{itemize}

前两条也称之为\textbf{衔接条件}。这些边界条件部分或全部锁定通解中的待定系数之间的关系,有时在归一化后,能使得波函数在相差一个相位因子的意义下被确定;同时这些边界条件也可能对能量本征值 $E$ 加以限制。

\subsubsection{无限深势阱:束缚态与量子化}
无限深势阱经过平移,可化作以下形式的(广义)势能函数:
\begin{equation}
    V(x)=\begin{cases}
        0 & 0\le x\le a\\
        \infty & \text{otherwise}
    \end{cases}
\end{equation}

首先考察区间 $[0,a]$ 以外的区域,它永远属于 $E<V$ 的情况。在通解 $\ref{SEq1dSqrPotntlSlt2}$ 中令 $V\ra\infty$,将得到 $\rho\ra\infty$。在右侧,根据平方可积条件,$\psi(x)$ 在无穷远处的有界将要求 $B_1=0$,而另一项也随 $\rho$ 趋于 $0$;在左侧情况类似。综上所述,势阱外区域的必然有 $\psi(x)\equiv 0$

对于势阱内区域,在通解 $\ref{SEq1dSqrPotntlSlt1}$ 中代入 $V=0$,于是得到
\begin{equation}
    \psi(x)=A_1\e^{\i kx}+A_2\e^{-\i kx},\quad 0\le x\le a,\quad k=\sqrt{\frac{2mE}{\hbar^2}}
\end{equation}
但是考虑与势阱外的衔接条件,连续性要求 $\psi(0)=\psi(a)=0$,于是可以导出
\begin{equation}
    \psi(x)=2\i A_1\sin kx,\quad k=\frac{n\pi}{a},\quad n\in\N_+
\end{equation}
归一化后得到所有可能的解
\begin{equation}\label{eq:1d_infnt_ptntl_wll_wv_fnctn}
    \psi_n(x)=\sqrt{\frac 2 a}\sin\left(\frac{n\pi x}{a}\right)
\end{equation}
每个波函数对应于能量本征值,这些本征值都是非简并的
\begin{equation}\label{eq:1d_infnt_ptntl_wll_egnvl}
    E_n=\frac{n^2\pi^2\hbar^2}{2ma^2}=\frac{n^2h^2}{8ma^2}
\end{equation}
可见能量本征值不再能取到任意值。这与经典物理的预言的不同:在一个经典的无限深坑底,一个质点在无摩擦情况下可以拥有任意的机械能值,但在量子力学中势阱中粒子的能量只能拾级而上,离散的能量之间的值是不可能达到的。正因如此,我们将每一个能量本征值对应的全体本征态称为一个\textbf{能级},出现能量上简并的本征态时,也成该能级是简并的。

同时也会发现,量子力学不允许此情形下的粒子以能量为 $0$ 的态出现在势阱中——最低能级 $E_1\ne 0$ 又称为\textbf{零点能},波函数 $\psi_1(x)$ 对应的态 $\ket{\psi_1}$ 又称为\textbf{基态},其动能的平均值(因为势能 $V=0$,也就是总能的平均值)$\bra{\psi_1}H\ket{\psi_1}=E_1\ne 0$。这在经典物理中不可想象,“静止”是不被允许的;然而这也侧面反映了不确定度关系式的普遍性:势阱中静止的粒子将能够同时确定其固定的位置和为 0 的动量,从而与量子力学的结论相矛盾。但所得的解的形式又与经典物理相关——它是两端固定在势阱两壁的,所有可能的驻波的形式。

更一般地,这一模型中的粒子可称之为\textbf{束缚态},原因在于其经典模型中质点只能在有限的空间范围内运动,从而被“束缚”在该范围之内。势阱模型中,位置表象波函数只在有限的范围内可取非零值,结合概率诠释,就意味着粒子出现在该范围之外的概率为0。从数学上看,束缚态发生能量量子化现象很多时候是一种必然——该范围的边界处(或无穷远处)的衔接条件要求波函数必须在边界处为0。这种有限区间上的齐次边界条件,结合波函数所满足的、来自厄米算符 $H$ 的二阶齐次微分方程,常常构成一个 \textbf{斯图姆-刘维尔(Sturm-Liouville)型本征值问题},简称 S-L 本征值问题。此类问题的性质是具有离散的本征值,且本征矢(或者从微分方程的解来看,是本征函数)构成在该体系的束缚态空间(或,满足微分方程和边界条件的函数空间)中的正交归一基,因此形成了一个新的表象。

而我们回顾万能的位置表象和动量表象,所对应的 $\delta$ 函数基和平面波基都不是离散基。而在束缚态问题中,表象又可以建立在离散的基之上。从连续到离散的“退化”类似于(满足一定条件的)任意函数的傅里叶变换到(满足一定条件的)任意周期函数的傅里叶展开的区别——它反映了周期函数空间平移对称性的规制。同样地,束缚态问题内蕴的对称性,也暗藏在所得的离散基 $\{\ket{\psi_n}\}$ 中,可惜为了直观和方便计算,只能从位置表象下管中窥豹。从位置表象下观察到的波函数 $\psi_n(x)$ 的值实际是一个二元函数 $f(n,x)$,它是位置表象与束缚态表象之间的转换系数,在这一视角下,该问题得到的三角函数形式的解就是问题对称性的反映。

该模型可以简化讨论某些具有长程直链型共轭体系的有机分子的紫外-可见吸收问题:将其电子的运动范围视作一维势箱,能够根据以上结果粗估其能级特性,从而预言它和特定波长的光的相互作用。

\subsubsection{势阶:反射与透射}
势阶经过平移,可化作以下形式的势能函数:
\begin{equation}
    V(x)=\begin{cases}
        0 & x<0\\
        V_0 & x\ge 0
    \end{cases}
\end{equation}

首先考虑 $E>V_0$ 的情况。根据已经得到的通解,令
\begin{equation*}
    k_1=\sqrt{\frac{2mE}{\hbar^2}},\quad k_2=\sqrt{\frac{2m(E-V_0)}{\hbar^2}}
\end{equation*}

就能得到两个区域的通解
\begin{align}
    \psi_-(x)&=A_1\e^{\i k_1x}+A_1'\e^{-\i k_1x} & x<0\\
    \psi_+(x)&=A_1\e^{\i k_2x}+A_1'\e^{-\i k_2x} & x\ge 0
\end{align}

根据间断点 $x=0$ 处的衔接条件,我们希望求出待定系数之间的关系,也即 $A_1',A_2,A_2'$ 与 $A_1$ 之比,但仅凭一点处的两个方程(函数连续、导数连续)无法确定三个关系。从光学的角度考察通解的形式,会发现两侧的通解都可以诠释为一个形如 $A\e^{\i kx}$ 的右行波和形如 $A\e^{-\i kx}$ 的左行波。如果令 $A_2'=0$,就可视作来自左端(无穷远处)向右行进的波(系数为 $A_1$),在势阶处形成反射波(系数为 $A_1'$)和透射波(系数为 $A_2'$)。而后可以求出以下比值:
\begin{equation*}
    \frac{A_1'}{A_1}=\frac{k_1-k_2}{k_1+k_2},\quad \frac{A_2}{A_1}=\frac{2k_1}{k_1+k_2}
\end{equation*}
为了得到归一化的态矢,可以将波函数归一化从而确定 $A_1$。

在 $E<V_0$ 的情况中,$\psi_+(x)$ 的形式变为
\begin{equation}
    \psi_+(x)=B_2\e^{\rho_2x}+B_2'\e^{-\rho_2x},\quad \rho_2=\sqrt{\frac{2m(V_0-E)}{\hbar^2}}
\end{equation}

考虑平方可积性,要求 $x\ra+\infty$ 时波函数有界,于是 $B_2=0$。此时再无需设定没有来自右端的左行波,因为此时已经得到与光学相似的结论:来自左端的右行波将发生全反射,并在正半轴留下指数衰减的\textbf{隐失波}。直接求解衔接条件,得到
\begin{equation*}
    \frac{A_1'}{A_1}=\frac{k_1-\i\rho_2}{k_1+\i\rho_2},\quad \frac{A_2}{A_1}=\frac{2k_1}{k_1+\i\rho_2}
\end{equation*}
复数比值也暗示了波在反射界面的相位突变。

\subsubsection{势垒:隧道效应}
势垒经过平移,可化作以下形式的势能函数:
\begin{equation}
    V(x)=\begin{cases}
        V_0 & 0\le x\le l\\
        0 & \text{otherwise}
    \end{cases}
\end{equation}
只考虑 $E<V_0$ 的情况,也即势垒的高度超过了粒子的总能。如果依照前面的惯例,将这一体系视作来自左侧的行波的传播,按照经典物理的语言,粒子将无法穿透过高的势垒。

此时定义
\begin{equation*}
    k_1=\sqrt{\frac{2mE}{\hbar^2}},\quad \rho_2=\sqrt{\frac{2m(V_0-E)}{\hbar ^2}}
\end{equation*}
就得到三个区域的通解
\begin{align}
    \psi_1(x)&=A_1\e^{\i k_1x}+A_1'\e^{-\i k_1x} & x<0\\
    \psi_2(x)&=A_2\e^{\i \rho_2x}+A_2'\e^{-\i \rho_2x} & 0\le x\le l\\
    \psi_3(x)&=A_3\e^{\i k_1x}+A_3'\e^{-\i k_1x} & x\ge l
\end{align}
仿照之前的假设,我们要求 $A_3'=0$,即不含有来自右侧的行波,然后根据边界条件确定剩下的系数与 $A_1$ 的比值,能够得到的一个重要结果是
\begin{equation*}
    \left|\frac{A_3}{A_1}\right|^2=\frac{4E(V_0-E)}{4E(V_0-E)+V_0^2\sinh^2\frac{\sqrt{2m(V_0-E)}l}{\hbar}}
\end{equation*}

这意味着 $\psi_3(x)$ 对概率密度有贡献,也即离子有概率出现在势垒右侧。此现象仿佛粒子洞穿了高墙而非越过它,是经典物理未曾解释的现象,因此称之为隧道效应。它在扫描隧道显微镜(STM)的设计中起到了核心作用,能够用于观察纳米级别的形貌结构。

\subsection{一维谐振子}

经典物理中的一维弹簧振子满足以下简谐运动的动力学方程
\begin{equation}
    m\frac{d^2x}{\d t^2}=kx
\end{equation}
右侧可以写成 $kx=-V'(x)$ 的势能项,再令 $\omega=\sqrt{k/m}$,就得到了总能量
\begin{equation}
    E=\frac{p^2}{2m}+\frac 1 2 m\omega^2x^2
\end{equation}
其机械能守恒不言而喻。在量子力学中,用位置和动量算符代替其中的物理量,就得到了体系的哈密顿算符:
\begin{equation}
    H=\frac{P^2}{2m}+\frac 1 2 m\omega^2X^2
\end{equation}
由于 $H$ 不显含时间,我们关注其定态问题,在位置表象下化作以下微分方程
\begin{equation}
    \left[-\frac{\hbar^2}{2m}\frac{\d^2}{\d x^2}+\frac 1 2 m\omega^2 x^2\right]\psi(x)=E\psi(x)
\end{equation}

但接下来的分析将首先回避这一微分方程,而直面纯粹的本征值问题 $H\ket{\psi_\nu^i}=E_\nu\ket{\psi_\nu^i}$。其中 $\nu$ 用于区分不同的本征值,$i$ 用于区分简并本征值的本征空间内相互正交的本征矢。

\subsubsection{升降算符的构造}
为了方便,我们重新定义约去量纲的位置、动量和哈密顿算符:
\begin{equation*}
    \td{X} =\sqrt{\frac{m\omega}{\hbar}}X,\quad \td P = \frac{1}{\sqrt{m\hbar\omega}}P,\quad \td H=\frac 1 2(\td X^2+\td P^2)
\end{equation*}
可得以下关系
\begin{equation*}
    [\td X,\td P] = \i,\quad H=\hbar\omega\td H
\end{equation*}

接下来引入一对互为伴随的无量纲算符:
\begin{equation}
    a = \frac 1{\sqrt 2}(\td X+\i\td P),\quad a^*=\frac 1{\sqrt 2}(\td X-\i\td P)
\end{equation}
可以用这两个算符表示原来的约化位置和约化动量
\begin{equation}
    \td X = \frac 1{\sqrt 2}(a+a^*),\quad\td P=\frac 1{\sqrt 2\i}(a-a^*)
\end{equation}
令 $N=a^*a$,从正则对易关系~\ref{thm:commuterp},可以得到以下对易关系式
\begin{itemize}
    \item \begin{equation}
        [a,a^*] = 1
    \end{equation}
    \item \begin{equation}
        \td H = N+\frac 1 2
    \end{equation}
    \item \begin{equation}\label{Nacommute}
        [N,a] = -a,\quad [N,a^*] = a^*
    \end{equation}
\end{itemize}
此时转而考虑 $N$ 的本征值方程 $N\ket{\psi_\nu^i} = \nu\ket{\psi_\nu^i}$,则它等价于
\begin{equation*}
    H\ket{\psi_\nu^i} = \left(\nu+\frac 1 2\right)\hbar\omega\ket{\psi_\nu^i}
\end{equation*}
也就是说求解 $N$ 的本征值问题,等同于求解 $H$ 的本征值问题。本征值之间的关系是 $E_\nu=\left(\nu+\frac 1 2\right)\hbar\omega$。

接下来提出几条引理来说明这些新构造的算符的特殊性质:
\begin{lemma}[$N$ 的半正定性]\label{lem:nonnegegval}
    $N$ 的本征值非负。
\end{lemma}
\begin{proof}
    任取 $N$ 的一个本征矢 $\ket{\psi_\nu^i}$,根据模方的正定性,有
    \begin{equation}
        \label{eq:nonnegegval}
        \|a\ket{\psi_\nu^i}\|^2=\braket
        {\psi_\nu^i|a^*a|\psi_\nu^i}=\braket{\psi_\nu^i|N|\psi_\nu^i}=\nu\braket{\psi_\nu^i|\psi_\nu^i}\ge 0
    \end{equation}
    由于有物理意义的态矢满足 $\|\psi_\nu^i\|> 0$,所以必然有 $\nu\ge 0$。
\end{proof}

\begin{lemma}[降算符的零空间]\label{lem:zeroegval}
    $a\ket{\psi} = 0$ 当且仅当 $\ket{\psi}$ 是 $N$ 的对应于本征值 $\nu=0$ 的本征矢。
\end{lemma}
\begin{proof}
    先证必要性:将 $\nu=0$ 对应的任意本征矢记作 $\ket{\psi_\nu^0}$,根据推导过程 \ref{eq:nonnegegval},$a\ket{\psi_0^i}$ 的模方为 0,根据模方的正定性,这当且仅当 $a\ket{\psi_0^i} = 0$。
    
    再证充分性:对于一个满足 $a\ket{\psi} = 0$ 的右矢,用 $a^*$ 作用于两端就得到 $a^*a\ket{\psi} = N\ket{\psi} = 0$,于是 $\ket{\psi}$ 确实是 $N$ 的对应于本征值 $\nu=0$ 的本征矢。
\end{proof}

\begin{lemma}[降算符的作用]\label{lem:nonzeroegval}
若 $\nu > 0$,则 $a\ket{\psi_\nu^i}$ 是 $N$ 对应于本征值 $\nu-1$ 的本征矢。
\end{lemma}
\begin{proof}
    根据推导过程 \ref{eq:nonnegegval},$a\ket{\psi_\nu^i}$ 的模方是整数,因此 $a\ket{\psi_\nu^i} \ne 0$。然后使用对易关系 \ref{Nacommute},有
    \begin{equation}
        [N,a]\ket{\psi_\nu^i} = (Na-aN)\ket{\psi_\nu^i} = (Na-\nu a)\ket{\psi_\nu^i}= -a\ket{\psi_\nu^i} 
    \end{equation}
    这也就得到
    \begin{equation}
        N(a\ket{\psi_\nu^i})=(\nu-1)(a\ket{\psi_\nu^i})
    \end{equation}
    于是 $a\ket{\psi_\nu^i}$ 确实是 $N$ 的对应于本征值 $\nu-1$ 的本征矢。
\end{proof}

\begin{lemma}[升算符的零空间]\label{lem:zeroegval2}
    $a^*\ket{\psi_\nu^i}$ 永远不会为 0。
\end{lemma}
\begin{proof}
任取 $N$ 的一个本征矢 $\ket{\psi_\nu^i}$,有
    \begin{equation}
        \|a^*\ket{\psi_\nu^i}\|^2=\braket{\psi_\nu^i|aa^*|\psi_\nu^i}=\braket{\psi_\nu^i|(N+1)|\psi_\nu^i}=(\nu+1)\braket{\psi_\nu^i|\psi_\nu^i}\ge 0
    \end{equation}
    根据 $\|\psi_\nu^i\|^2>0$,以及前面引理 \ref{lem:nonnegegval} 证明的 $\nu\ge 0$,就推出了 $\|a^*\psi_\nu^i\|^2> 0$,因而 $a^*\ket{\psi_\nu^i}\ne 0$。
\end{proof}

\begin{lemma}[升算符的作用]\label{lem:nonzeroegval2}
    若 $\nu > 0$,则 $a^*\ket{\psi_\nu^i}$ 是 $N$ 对应于本征值 $\nu+1$ 的本征矢。
\end{lemma}
\begin{proof}
使用对易关系 \ref{Nacommute},有
    \begin{equation}
        [N,a^*]\ket{\psi_\nu^i} = (Na^*-a^*N)\ket{\psi_\nu^i} = (Na^*-\nu a^*)\ket{\psi_\nu^i}= a^*\ket{\psi_\nu^i} 
    \end{equation}
    这也就得到
    \begin{equation}
        N(a^*\ket{\psi_\nu^i})=(\nu+1)(a^*\ket{\psi_\nu^i})
    \end{equation}
    于是 $a^*\ket{\psi_\nu^i}$ 确实是 $N$ 的对应于本征值 $\nu+1$ 的本征矢。
\end{proof}

\subsubsection{本征系统的性质}

结合以上引理,能够对 $N$ 的谱作出预言:
\begin{theorem}[$N$ 的本征值的取值范围]
   $N$ 的本征值 $\nu$ 遍及所有自然数。
\end{theorem}
\begin{proof}
    引理 \ref{lem:nonnegegval} 封堵了所有本征值为负的可能。
    
    于是,假设 $\nu$ 不是整数,那么一定能找到自然数 $n$ 使得 $n<\nu<n+1$,则根据引理 \ref{lem:nonzeroegval},$a^p\ket{\psi_\nu^i}$ 在 $0\le p\le n$ 时都是 $N$ 对应本征值 $\nu-p$ 的本征矢。而后继续将 $a$ 作用于 $a^n\ket{\psi_\nu^i}$,仍得到一个 $N$ 的本征矢,对应于本征值 $\nu-p-1<0$,从而与引理 \ref{lem:nonnegegval} 相矛盾。
    
    而 $\nu$ 是整数的情况则不会出现此情况。假设 $\nu = n$,则 $a^n\ket{\psi_\nu^i}$ 将得到对应本征值 0 的本征矢,而根据引理 \ref{lem:zeroegval},$a^{n+1}\ket{\psi_\nu^i}=0$,不再是 $N$ 的本征矢。
    
    于是只要找到一个本征值 $k$,它就一定是自然数,在对应的本征矢上施以 $k$ 次 $a$ 就得到对应本征值 0 的本征矢。根据引理 \ref{lem:nonzeroegval2},由数学归纳法可知,再施以 $n$ 次 $a^*$ 将得到对应于任意自然数 $n$ 的本征矢。找到了这些本征矢,就表明任何自然数都是 $N$ 的本征值。
\end{proof}

根据这条引理,$N$ 具有某种计数的性质,我们又称其为\textbf{量子数算符},它直接以本征值的形式给出一个本征态的量子数 $\nu$。

回到算符 $H$ 的视角,它的谱也就表现为
\begin{equation}
    E_n=\left(n+\frac 1 2\right)\hbar\omega,\quad n\in\N
\end{equation}
作为束缚态(经典力学中的弹簧振子具有最大振幅,故位置的范围受限),它与一维无限深势阱的结论相似:其能量是量子化的,且有零点能 $\hbar\omega/2$。但其相邻的容许能量之间有着固定的间距 $\hbar\omega$,算符 $a$ 每作用一次,所得态的能量本征值降低 $\hbar\omega$ 直至零点能为止;而算符 $a^*$ 每作用一次,所得态的能量本征值升高 $\hbar\omega$。因此 $a$ 和 $a^*$ 称为该体系下的\textbf{降算符}和\textbf{升算符}。 

谱的完整信息还包括各本征值的简并度,此时提出以下结论
\begin{theorem}[$N$ 的本征值的简并度]\label{thm:1d_hrmnc_osclt_nndgnrtn}
   $N$ 的所有本征值都是非简并的。
\end{theorem}
\begin{proof}
    首先证明基态能级是非简并的。作为基态 $\ket{\psi_0^i}$,引理 $\ref{lem:zeroegval}$ 要求满足方程 $a\ket{\psi_0^i}$,展开 $a$ 算符,得到
    \begin{equation}
        \frac 1 {\sqrt 2}\left[\sqrt{\frac{m\omega}{\hbar}X}+\frac{\i}{\sqrt{m\hbar\omega}}P\right]\ket{\psi_0^i}=0
    \end{equation}
    取位置表象,得到
    \begin{equation}
        \left(\frac{m\omega}{\hbar}x+\frac{\d}{\d x}\right)\psi_0^{(i)}(x)=0
    \end{equation}
    对于这样的一阶线性常系数齐次常微分方程,有非平凡的通解
    \begin{equation}\label{eq:1d_hrmnc_osclt_grnd_0}
        \psi_0^{(i)}(x)=C\e^{-\frac 1 2 \frac{m\omega}{\hbar}x^2}
    \end{equation}
    无论是从该通解的具体形式,还是从一阶微分方程只有一个线性无关解的结论,可知该通解只张成一维的本征空间,因此基态能级 $E_0=\hbar\omega/2$ 是非简并的。
    
    而假设已经知道:能级 $E_n=(n+1/2)\hbar\omega$ 非简并,即满足本征方程 $N\ket{\psi_n}=n\ket{\psi_n}$ 的本征矢 $\ket{\psi_n}$ 张成一维空间(这也是为什么上标 $i$ 被略去);然后任取一个对应于本征值 $n+1$ 的本征矢 $\ket{\psi_{n+1}^i}$,用降算符作用于其上,则必然存在一个数 $c$ 使得
    $a\ket{\psi_{n+1}^i}=c\ket{\psi_n}$,于是再将升算符作用于其上,就得到了
    \begin{equation}
        a^*a\ket{\psi_{n+1}^i}=ca^*\ket{\psi_n}=N\ket{\psi_{n+1}^i}\implies\ket{\psi_{n+1}^i}=\frac{c}{n+1}a^*\ket{\psi_n}
    \end{equation}
    而根据升算符的性质,$a^*\ket{\psi_n}$ 必然是一个属于本征值 $n+1$ 的本征矢,这就说明任意属于本征值 $n+1$ 的本征矢都与同一个右矢 $a^*\ket{\psi_n}$ 只相差一个系数,因而张成一维空间,因此本征值 $n+1$ 也是非简并的。
    
    由数学归纳法,所有的能级都是非简并的,对应本征值 $E_n$ 的本征矢分别都简记为 $\ket{\psi_n}$。
\end{proof}

\subsubsection{波函数的行为}

从定理 \ref{thm:1d_hrmnc_osclt_nndgnrtn} 的证明过程也能看出,假设给定了归一化的右矢 $\ket{\psi_0}$,就能通过升算符将所有的归一化右矢 $\ket{\psi_n}$ 表示出来,这是一个递推的过程。假设 $\ket{\psi_{n-1}}$ 已经是归一化的,则有 $\ket{\psi_n}=c_n a^*\ket{\psi_{n-1}}$。于是对 $\ket{\psi_n}$ 归一化,有
\begin{equation}
    \braket{\psi_n|\psi_n}=|c_n|^2\braket{\psi_{n-1}|aa^*|\psi_{n-1}}=n|c_n|^2=1
\end{equation}
如果此时规定相位,使得系数为正实数,则应该取 $c_n=1/\sqrt{n}$,于是递推地得到
\begin{equation}
    \ket{\psi_n}=\frac 1{\sqrt n}a^*\ket{\psi_{n-1}}=\cdots=\frac 1{\sqrt n}\frac 1{\sqrt{n-1}}\cdots\frac 1{\sqrt 2}(a^*)^n\ket{\psi_0}=\frac 1{\sqrt{n!}}(a^*)^n\ket{\psi_0}
\end{equation}

此时如果采用位置表象,就应首先从归一化的 $\psi_0(x)=\braket{x|\psi_0}$ 开始。通过归一化条件和相位约定,确定 \ref{eq:1d_hrmnc_osclt_grnd_0} 中的未定系数,就得到了
\begin{equation}\label{eq:1d_hrmnc_osclt_grnd}
        \psi_0^{(i)}(x)=\left(\frac{m\omega}{\pi\hbar}\right)^{1/4}\e^{-\frac 1 2 \frac{m\omega}{\hbar}x^2}
\end{equation}
考虑位置表象下,波函数空间内算符 $\hat X$ 和 $\hat P$ 的形式,就能得到升算符
\begin{equation}
    \hat a^*=\frac 1{\sqrt 2}\left[\sqrt{\frac{m\omega}{\hbar}}x-\sqrt{\frac{\hbar}{m\omega}}\frac\d{\d x}\right]
\end{equation}
于是计算出递推式
\begin{align}
    \psi_n(x)&=\braket{x|\psi_n}=\frac 1{\sqrt n!}\braket{x|(a^*)^n|\psi_0}\nonumber\\
    &=\frac{1}{\sqrt{n!}}\frac{1}{\sqrt{2^n}}\left[\sqrt{\frac{m\omega}{\hbar}}x-\sqrt{\frac{\hbar}{m\omega}}\frac\d{\d x}\right]^n\psi_0(x)\nonumber\\
    &=\left[\frac 1{2^nn!}\left(\frac{\hbar}{m\omega}\right)^n\right]^{1/2}\left(\frac{m\omega}{\pi\hbar}\right)^{1/4}\left[\frac{m\omega}{\hbar}x-\frac \d{\d x}\right]^n\e^{-\frac 1 2 \frac{m\omega}{\hbar}x^2}
\end{align}
从中可以观察到 $\psi_n(x)$ 一直是一个指数因子和一个 $n$ 次多项式的乘积,而该多项式称为\textbf{厄米多项式}。从无限深势阱的分析中就已经提到,束缚态问题所得的离散态矢构成了新的束缚态表象,谐振子正属于此类情况。而厄米多项式与指数因子的乘积就是该体系对称性在位置表象下的反映,所有可能的谐振子波函数都是这些基本波函数的线性组合。

一维谐振子模型是双原子分子振动的基础近似,在多原子分子、晶格(乃至真空场)的集体振动中,也可以利用经典力学的小振动普遍方法分解为一系列简正振动模式,分别按照简谐振子进行处理。分子的红外光谱和固体的多种物理特性都与这种振动模式有关。

\subsection{角动量}

角动量矢量是诞生于经典力学中的重要概念,它与旋转运动和中心力场问题密切相关。在经典力学中,如果建立直角坐标系,角动量满足的关系是 $\bm{\mc L}=\bm r\times \bm p$,从一个分量来看,就意味着 $\mc L_x=yp_z-zp_y$。于是通过量子化规则 \ref{eq:quantization},根据对易关系 \ref{thm:commuterp},甚至不必对称化就可以得到量子力学的角动量分量算符 $L_x=YP_z-ZP_y$,综合三个坐标分量,就得到了 $\bm L=\bm R\times \bm P$。

通过经典力学类比引入的角动量算符可以称之为\textbf{轨道角动量算符}。其不同分量之间具有独特的对易关系,考察一个特殊的方向为例:利用对易子的性质 \ref{thm:cmmttr_algbr_prprt} 和基本对易关系 \ref{thm:commuterp} 进行化简,可得
\begin{align}
    [L_x,L_y] & = [YP_z-ZP_y, ZP_x-XP_z]\nonumber\\
    & = [YP_z,ZP_x] + [ZP_y,XP_z]\nonumber\\
    & = Y[P_z,Z]P_x+X[Z,P_z]P_y\nonumber\\
    & = -\i\hbar YP_x+\i\hbar XP_y\nonumber\\
    & = \i\hbar L_z
\end{align}
其中第二个等式用到了 $[YP_z,XP_z]=[ZP_y,ZP_x]=0$,这容易从基本对易关系 推出。推广到三个分量,得到的一般关系是
\begin{equation}
    [L_i,L_j]=\epsilon_{ijk}\i\hbar L_k,\quad i,j,k\in\{x,y,z\}
\end{equation}
其中\textbf{列维-奇维塔(Levi-Civita)符号} $\epsilon_{ijk}$ 在 $(ijk)$ 取 $(xyz)$ 的奇排列时为 $-1$,偶排列时为 $1$,其余情况为 $0$。

\subsubsection{角动量的一般定义}

为了进一步推广,实际上能够抛弃位置和动量算符,从更一般的角度定义角动量:以下的定义抓住了分量之间的对易关系,将其视作角动量最重要的特征。

\begin{definition}[广义角动量算符]\label{def:gnrl_anglr_mmntm}
   可以将具有三个分量 $J_x,J_y,J_z$ 的矢量观测算符 $\bm J$ 称为广义角动量算符,只要其满足对易关系:
\begin{equation}
    [J_i,J_j]=\epsilon_{ijk}\i\hbar J_k,\quad i,j,k\in\{x,y,z\}
\end{equation}
\end{definition}

显然轨道角动量属于广义角动量的一种,接下来进一步剖析角动量中的代数关系。

\begin{theorem}[角动量平方算符的对易关系]\label{thm:anglr_mmntm_sqr_cmmt}
   称 $\bm J^2 = J_x^2+J_y^2+J_z^2$ 为角动量平方算符,它满足以下对易关系:
   \begin{equation}
       [\bm J^2,\bm J]=\bm 0
   \end{equation}
\end{theorem}
\begin{proof}
    对于矢量算符 $\bm J$,只需用一个分量为例。
    \begin{equation}
        [\bm J^2,J_x] = [J_x^2+J_y^2+J_z^2,J_x] = [J_y^2,J_x]+[J_z^2,J_x]
    \end{equation}
    这里使用了 $J_x^2$ 与 $J_x$ 显然的对易性。然后分别处理两项
    \begin{align}
        [J_y^2,J_x] & = J_y[J_y,J_x] + [J_y,J_x]J_y=-\i\hbar J_yJ_z-\i\hbar J_zJ_y\\
        [J_z^2,J_x] & = J_z[J_z,J_x] + [J_z,J_x]J_z=\i\hbar J_zJ_y+\i\hbar J_yJ_z\\
    \end{align}
    合并后得 0。
\end{proof}
    
根据以上关系,角动量平方与角动量各分量对易,角动量任意两个不同分量之间不对易,因此 $\bm J^2$ 和 $J_z$($z$ 方向并没有任何特殊性,它可以代表任何一个方向)可以融入某个 C.S.C.O. 中。而在为了求解这个 C.S.C.O. 共同本征矢,就必须了解 $J^2$ 和 $J_z$ 的谱。

首先简单分析 $\bm J^2$ 的本征值应有的性质,如果 $\bm J^2$ 对应某个本征值 $\lambda\hbar^2$(能够验证 $\hbar^2$ 的出现使其与角动量平方的量纲吻合),就会有
\begin{align}
    \braket{\psi|\bm J^2|\psi} & = \lambda\|\psi\|^2\nonumber\\&
    = \braket{\psi|\bm J^2_x|\psi}+\braket{\psi|\bm J^2_y|\psi}+\braket{\psi|\bm J^2_z|\psi} = \|J_x\psi\|^2+\|J_y\psi\|^2+\|J_z\psi\|^2\ge 0
\end{align}
由于 $\|\psi\|^2 \ge 0$,所以必有 $\lambda\ge 0$。因此可以记
\begin{equation}
    \lambda = j(j+1),\quad j\ge 0
\end{equation}
对于每一个非负的 $\lambda$,都有唯一非负的 $j$ 以这种方式与之对应。于是 $\bm J^2$ 的本征值记为 $j(j+1)\hbar^2\ (j\ge 0)$。

然后记 $J_z$ 的本征值为 $m\hbar$,在详尽分析之前,暂且无法得知 $m$ 的性质。

回顾最终目的:求解含有 $\bm J^2$ 和 $J_z$ 的 C.S.C.O. 的共同本征矢。每个本征矢自然有 $j,m$ 来标记 $\bm J^2$ 和 $J_z$ 的本征值,但是还可能有其它算符与其共同组成 C.S.C.O.。于是这些算符的本征值问题也要处理,才能确定唯一共同本征矢,它们的本征值可以用一个离散或连续、单个或成组的指标 $k$ 来概括。于是这些本征矢都标记为 $\ket{k,j,m}$,与角动量相关的本征值问题即为
\begin{align}
    \bm J^2\ket{k,j,m} & = j(j+1)\hbar^2\ket{k,j,m}\\
    J_z\ket{k,j,m} & = m\hbar \ket{k,j,m}
\end{align}

\subsubsection{升降算符的构造}


接下来将采取与谐振子问题类似的方法来解决问题(这自然是有其原因的)。定义角动量升降算符如下:
\begin{equation}
    J_+=J_x+\i J_y,\quad J_-=J_x-\i J_y
\end{equation}

利用角动量的对易关系,可通过计算证明以下重要性质。
\begin{itemize}
    \item
    \begin{equation}\label{eq:cmmt_Jz_J+-}
        [J_z,J_\pm] = \pm\hbar J_\pm
    \end{equation}
    \item
    \begin{equation}
        [J_+,J_-] = 2\hbar J_z
    \end{equation}
    \item
    \begin{equation}\label{eq:cmmt_J2_J+-}
        [\bm J^2,J_+] = [\bm J^2,J_-] = 0
    \end{equation}
    \item 
    \begin{equation}\label{eq:J_+-J_-+}
        J_\pm J_\mp = J_x^2+J_y^2\pm \hbar J_z = \bm J^2-J_z^2\pm\hbar J_z
    \end{equation}
\end{itemize}

以下几条引理能够揭示两个算符的用途,它比谐振子升降算符更加复杂。
\begin{lemma}[$\bm J^2$ 和 $J_z$ 本征值之间的约束关系]\label{lem:j_m_cnstrt}
    如果 $j(j+1)\hbar^2$ 和 $m\hbar$ 分别是 $\bm J^2,\bm J_z$ 对应同一个本征矢 $\ket{k,j,m}$ 的本征值,则不等式 $-j\le m\le j$ 一定成立。
\end{lemma}
\begin{proof}
    根据模方的正定性,给定归一化的右矢 $\ket{k,j,m}$,则有
    \begin{align}
        \|J_+\ket{k,j,m}\|^2 &= \braket{k,j,m|J_-J_+|k,j,m} = \braket{k,j,m|J^2-J_z^2-\hbar J_z|k,j,m} = [j(j+1)-m^2-m]\hbar^2\ge 0\\
        \|J_-\ket{k,j,m}\|^2 &= \braket{k,j,m|J_+J_-|k,j,m} = \braket{k,j,m|J^2-J_z^2+\hbar J_z|k,j,m} = [j(j+1)-m^2+m]\hbar^2\ge 0
    \end{align}
    从以上二式提炼出
    \begin{equation}
        \begin{cases}
        j(j+1)-m(m+1)=(j-m)(j+m+1)\ge 0\\
        j(j+1)-m(m-1)=(j-m+1)(j+m)\ge 0
    \end{cases}
    \end{equation}
    再考虑到 $j\ge 0$,就能推出
    \begin{equation}
        \begin{cases}
            -(j+1) \le m \le j\\
            -j\le m \le j+1
        \end{cases}\implies -j\le m\le j
    \end{equation}
\end{proof}

\begin{lemma}[降算符的零空间]\label{lem:anglr_lwr_oprtr_nll_spc}
    如果 $j(j+1)\hbar^2$ 和 $m\hbar$ 分别是 $\bm J^2,\bm J_z$ 对应同一个本征矢 $\ket{k,j,m}$ 的本征值,则 $J_-\ket{k,j,-j} = 0$ 当且仅当 $m=-j$。
\end{lemma}
\begin{proof}
    先证充分性,从引理 \ref{lem:j_m_cnstrt} 的证明过程可知,$\|J_-\ket{k,j,m}\|^2 = [j(j+1)-m(m-1)]\hbar^2$,代入 $m=-j$ 得 0,因此 $J_-\ket{k,j,m} = 0$。
    
    再证必要性,假设 $J_-\ket{k,j,m}=0$,左乘 $J_+$ 后就得到
    \begin{equation}
        \hbar^2(j+m)(j-m+1)\ket{k,j,m}=0,
    \end{equation}
    考虑到 $m$ 与 $j$ 在引理 \ref{lem:j_m_cnstrt} 中的约束,唯一解就是 $j=-m$。
\end{proof}


\begin{lemma}[降算符的作用]\label{lem:anglr_lwr_oprtr_nll_effct}
    如果 $j(j+1)\hbar^2$ 和 $m\hbar$ 分别是 $\bm J^2,\bm J_z$ 对应同一个本征矢 $\ket{k,j,m}$ 的本征值,则 $m>-j$ 时 $J_-\ket{k,j,m}$ 是 $\bm J^2$ 和 $J_z$ 的共同本征矢,分别属于本征值 $j(j+1)\hbar^2$ 和 $(m-1)\hbar$。
\end{lemma}
\begin{proof}
    已经证明了 $m\ge j$ 时,$J_-\ket{k,j,m}\ge 0$。根据对易关系 \ref{eq:cmmt_J2_J+-},又能得到 $[\bm J^2,j_-]\ket{k,j,m}=0$,拆开对易子,又可以写成
    \begin{equation}
        \bm J^2 J_-\ket{k,j,m} = J_-\bm J^2\ket{k,j,m} = j(j+1)\hbar^2 J_-\ket{k,j,m}
    \end{equation}
    此式说明了 $J_-\ket{k,j,m}$ 是 $\bm J^2$ 的本征矢,属于本征值 $j(j+1)\hbar^2$。
    
    再利用对易关系 \ref{eq:cmmt_Jz_J+-} 得到 $[J_z,J_-]\ket{k,j,m} = -\hbar J_-\ket{k,j,m}$,拆开后也即
    \begin{equation}
        J_zJ_-\ket{k,j,m} = J_-J_z\ket{k,j,m} - \hbar J_-\ket{k,j,m} = m\hbar J_-\ket{k,j,m} -\hbar J_-\ket{k,j,m} = (m-1)\hbar J_-\ket{k,j,m}
    \end{equation}
    此式说明了 $J_-\ket{k,j,m}$ 是 $\bm J^z$ 的本征矢,属于本征值 $j(j+1)\hbar^2$。
\end{proof}

\begin{lemma}[升算符的零空间及其作用]
    如果 $j(j+1)\hbar^2$ 和 $m\hbar$ 分别是 $\bm J^2,\bm J_z$ 对应同一个本征矢 $\ket{k,j,m}$ 的本征值,则 $J_+\ket{k,j,-j} = 0$ 当且仅当 $m=-j$,$m>-j$ 时 $J_+\ket{k,j,m}$ 是 $\bm J^2$ 和 $J_z$ 的共同本征矢,分别属于本征值 $j(j+1)\hbar^2$ 和 $(m+1)\hbar$。
\end{lemma}
\begin{proof}
    证明完全类似于降算符的两个引理 \ref{lem:anglr_lwr_oprtr_nll_spc} 和 \ref{lem:anglr_lwr_oprtr_nll_effct}。
\end{proof}

综合以上引理,可以推得 $\bm J^2$ 和 $J_z$ 的谱的完整性质。
\begin{theorem}[角动量算符的谱]\label{thm:anglr_mmntm_spctrm}
   $\bm J$ 是任意的广义角动量算符。如果 $j(j+1)\hbar^2$ 和 $m\hbar$ 分别是 $\bm J^2,\bm J_z$ 对应同一个本征矢 $\ket{k,j,m}$ 的本征值,则 $2j$ 只有可能是自然数;而对于给定的 $j$,只要有 $m$ 的取值存在,则 $m$ 就一定能遍及 $-j,-j+1,\cdots,j$ 共 $2j+1$ 种取值。
\end{theorem}
\begin{proof}
    引理 \ref{lem:j_m_cnstrt} 使得一定存在一个自然数 $p$,满足 $-j\le m-p< -j+1$。于是在 $\ket{k,j,m}$ 上让 $J_-$ 逐次作用直至 $p$ 次。根据引理 \ref{lem:anglr_lwr_oprtr_nll_effct},得到的 $(J_-)^n\ket{k,j,m}\ (n=0,1,\cdots,p)$ 是 $\bm J^2$ 和 $J_z$ 的共同本征矢,分别属于本征值 $j(j+1)\hbar^2$ 和 $(m-n)\hbar$。
    
    此时假设 $m-p>-j$,则可以再让 $J_-$ 作用一次,将得到 $(J_-)^{p+1}\ket{k,j,m}$,它也是 $\bm J^2$ 和 $J_z$ 的共同本征矢,分别属于本征值 $j(j+1)\hbar^2$ 和 $(m-p-1)\hbar$。但根据对 $p$ 的设定,必然有 $m-p-1<-j$,这就与引理  \ref{lem:j_m_cnstrt} 相矛盾。
    
    但如果 $m-p=-j$,根据引理 \ref{lem:anglr_lwr_oprtr_nll_spc} 就会使得 $(J_-)^{p+1}\ket{k,j,m}=0$,不再是本征矢,从而不会引发矛盾,所以这是唯一可能的情况。对称地,我们也可以从 $J_+$ 的角度证明,一定存在一个自然数 $q$ 使得 $m+q=j$。结合两边的情况就得到 $p+q=2j$,因此 $2j$ 一定是自然数。
    
    在此基础上,假设对于给定的 $j$,某个 $\ket{k,j,m}$ 确实存在,那么可以立即通过升降算符逐级下探到 $m-p=-j$,上寻到 $m+q=j$,中间一路得到的本征值就是
    \begin{equation}
        -j\hbar,(-j+1)\hbar,\cdots,(j-1)\hbar,j\hbar
    \end{equation}
\end{proof}

\subsubsection{角动量表象}
    由于上述衍生自 $\bm J$ 的算符都设定为观测算符,且用本征值 $k$ 代表的那些观测算符与对应于 $j$ 的 $\bm L^2$,对应于 $m$ 的 $L_z$ 共同构成了 C.S.C.O。所以对于整个态空间 $\ms E$ 而言,应该能找到一组形如 $\{\ket{k,j,m}\}$ 的正交归一基,但是这组基如何构造,以及指标 $k$ 与 $l,m$ 的联系目前是不明确的。
    
    首先,根据所要研究的体系,$l,m$ 的取值范围不同。我们先取一对可能出现的 $l,m$,则对应的所有本征矢张成了 $\ms E$ 的一个子空间 $\ms E(j,m)$,它的维数将由未定的 $k$ 来确定。为了方便讨论,只详细分析有限维的情况,假设 $\dim \ms E(j,m)=g(j,m)$,因此可以在 $\ms E(j,m)$ 中找到形如 $\{\ket{k,j,m}:k=1,\cdots,g(j,m)\}$ 的正交归一基。
    
    而在 $m\ne \pm j$ 时,就可以想象 $\ms E(j,m\pm 1)$ 的存在。接下来将从选定的 $\{\ket{k,j,m}\}_k$ 出发构造 $\ms E(j,m\pm 1)$ 的基。
    
    \begin{theorem}[从 $\ms E(j,m)$ 的基构造 $\ms E(j,m\pm 1)$ 的基]
       $m\ne \pm j$ 时,若 $\{\ket{k,j,m}\}_k$ 是 $\ms E(j,m)$ 中的正交归一基,则 $\{J_\pm\ket{k,j,m}\}_k$ 经过归一化后能得到 $\ms E(j,m\pm 1)$ 的正交归一基。
    \end{theorem}
    \begin{proof}
    先验证正交归一性:由于 $J_\pm\ket{k,j,m}$ 身在 $\ms E(j,m\pm 1)$,有
    \begin{align}
        \braket{k_2,j,m|J_\mp J_\pm|k_1,j,m} &= \braket{k_2,j,m|(\bm J^2-J_z^2\mp \hbar J_z)|k_1,j,m}
        \\&=[j(j+1)-m(m\pm 1)]\hbar^2\braket{k_2,j,m|k_1,j,m}=[j(j+1)-m(m\pm 1)]\hbar^2\delta_{k_1k_2}
    \end{align}
    这里用到了恒等式 \ref{eq:J_+-J_-+} 和 $\ms E(j,m\pm 1)$ 之内的正交归一关系。此式说明了 $J_\pm\ket{k,j,m}$ 在 $\ms E(j,m\pm 1)$ 中同样是正交的,且模方为 $[j(j+1)-m(m\pm 1)]\hbar^2$。因此我们只需按惯例设定正实数归一化因子,就得到递推式
    \begin{equation}\label{eq:k,j,m+-1}
        \ket{k,j,m\pm1} = \frac{1}{\hbar\sqrt{j(j+1)-m(m\pm1)}}J_\pm\ket{k,j,m}
    \end{equation}
    就从 $\{\ket{k,j,m}\}_k$ 得到了形如 $\{\ket{k,j,m\pm1}\}_{k}$ 正交归一右矢。
    
    为了验证完备性,假设 $\ms E(j,m\pm 1)$ 中还存在一个未曾考虑的 $k_0$ 使得右矢 $\ket{k_0,j,m\pm1}$ 与以上得到的所有 $\ket{k,j,m\pm1}$ 都正交,可以作出以下推理:
    \begin{enumerate}
        \item 根据 $m,j$ 的约束关系,一定会有 $m\pm1\ne \mp j$,所以根据 $J_\pm$ 的作用方式与 $m$ 的关系,有 $J_\mp\ket{k_0,j,m\pm1}\ne 0$。
        \item 根据 $J_\pm$ 作用后的结果,$J_\mp\ket{k_0,j,m\pm1}$ 是$\ms E(j,m)$ 中的成员。
        \item 根据前面验证的正交归一性,$J_\mp\ket{k_0,j,m\pm1}$ 应正交于所有的 $J_\mp\ket{k,j,m\pm1}$。
        \item 但是根据递推式~\ref{eq:k,j,m+-1} 和恒等式 \ref{eq:J_+-J_-+},$J_\mp\ket{k,j,m\pm1}\propto J_\mp J_\pm\ket{k,j,m}\propto\ket{k,j,m}$ 对任何 $k$ 都成立。
        \item 于是对于已经考虑的所有 $k$,$J_\mp\ket{k_0,j,m\pm1}$ 在 $\ms E(j,m)$ 中正交于所有 $\ket{k,j,m}$。
        \item 这说明 $\{J_\pm\ket{k,j,m}\}_k$ 还不足以称为 $\ms E(j,m)$ 的基,至少要加入 $J_\mp\ket{k_0,j,m\pm1}$,因而产生了矛盾。
    \end{enumerate}
    
    因此这样的 $\ket{k_0,j,m\pm1}$ 不存在,由递推式得到的正交归一右矢组的完备性得以保证,前面得到的 $\{\ket{k,j,m\pm1}\}_k$ 就是 $\ms E(j,m\pm 1)$ 的基。
        
    \end{proof}
    
    递推式 \ref{eq:k,j,m+-1} 实际上建立了 $\ms E(j,m\pm 1)$ 和 $\ms E(j,m)$ 的基之间的一一对应关系,从而使得 $g(j,m)\equiv g(j)$ 与 $m$ 无关;普遍意义下则说所有的 $\{\ms E(j,m)\}_m$ 之间因维数相等而同构,而算符 $J_z,J_\pm$ 只是在显露 $m$ 的值或操纵其升降,与 $k$ 完全无关。
    
    而另一方面考虑,假设 $k$ 相关的唯一算符是 $A$,它与 $\bm J^2,J_z$ 都应对易,所以容易证明 $\ms E(j,m)$ 在 $A$ 下不变。$A$ 又是一个观测算符,因此对于每个可能取到的 $j$,$A$ 都可以先在 $\ms E(j,m=j)$ 中对角化,得到一系列本征值 $a_{k,j}$。由于 $A,\bm J^2,J_z$ 组成 C.S.C.O.,因此与每个 $a_{k,j}$ 相关的矢量 $\ket{k,j,m=j}\in\ms E(j,j)$ 是唯一的,这就选定了 $E(j,j)$ 中的标准基 $\{\ket{k,j,m=j}\}_k$。
    
    对于体系中每一个可能出现的 $j$,选定一个任意一个 $\ms E(j,m=j)$ 中的标准基,都可以通过归纳将这种关系延续到所有可能取到的 $2j+1$ 个 $m$ 上。以这种方式得到的基 $\{\ket{k,j,m}\}$ 就是 $\ms E$ 的角动量表象下的标准基。根据已证的结论,$k$ 相关的观测算符性质和各个子空间 $\ms E(j,m)$ 中原本的性质,有正交、归一、完备关系:
    \begin{align}
        &\braket{k,j,m|k',j',m'} = \delta_{kk'}\delta_{jj'}\delta_{mm'}\\
        &\sum_j\sum_{m=-j}^{j}\sum_{k=1}^{g(j)}\ket{k,j,m}\bra{k,j,m} = I
    \end{align}
    
    这同时说明来自不同的 $\ms E(j,m)$ 之间的矢量是正交的,因此可以写出直和关系,即
    \begin{equation}
        \ms E=\bigoplus_{j,m} \ms E(j,m)
    \end{equation}
    于是在构造以 $k,j,m$ 为指标的标准基同时提供了态空间 $\ms E$ 的一种直和分解。尽管这种基的构造是方便的,但这种直和分解并不利于讨论问题,因为讨论 $g(j)$ 就要分析 $A$ 的具体性质有关,且这些子空间 $\ms E(j,m)$ 并不是 $\bm J_z$ 相关的算符的不变子空间——这为该分解下 $\bm J^2,J_z$ 的矩阵表示带来了更多麻烦。
    
    但是,注意到每个 $j$ 伴随着相同的 $k$ 集合,因此可以把标准基 $\{\ket{k,j,m}\}$ 中具有相同 $k,j$ 的向量组合起来,张成子空间 $\ms E(k,j)$,其中的标准基自然地定为 $\{\ket{k,j,m}\}_m$。不同子空间中的态矢显然也是正交的,直和分解写作
    \begin{equation}
        \ms E=\bigoplus_{k,j} \ms E(k,j)
    \end{equation}
    
    该子空间有两条优良性质:
    \begin{enumerate}
        \item $\dim\ms E(k,j)\equiv2j+1$,这来自于 $m$ 的 $2j+1$ 个取值。
        \item $\ms E(k,j)$ 在 $\bm J$ 下不变,因为 $\bm J$ 相关的算符只对 $m$ 起作用。
    \end{enumerate}
    
    
    第一条性质规定了 $\ms E(k,j)$ 的维数,第二条性质使得角动量表象下 $\bm J$ 相关的矩阵都可以写成分块对角矩阵——只需按子空间 $\ms E(k,j)$ 对基矢进行分组,则每个 $\ms E(k,j)$ 都对应一个对角块,其他部分的矩阵元都为 0,常用的情况根据 \ref{eq:k,j,m+-1} 和算符的各定义求出:
    \begin{align}\label{eq:anglr_mmntm_mtrx_elmnt}
        & \braket{k,j,m|J_z|k',j',m'} = m\hbar\delta_{kk'}\delta_{jj'}\delta_{mm'}\\
        & \braket{k,j,m|J_\pm|k',j',m'} = \hbar\sqrt{j(j+1)-m(m\pm1)}\delta_{kk'}\delta_{jj'}\delta_{m,m'\pm 1}\\
        & \braket{k,j,m|J^2|k',j',m'} =j(j+1)\hbar^2\delta_{kk'}\delta_{jj'}\delta_{mm'}\\
        & \braket{k,j,m|J_x|k',j',m'} = \frac{\hbar} 2\delta_{kk'}\delta_{jj'}\left[\sqrt{j(j+1)-m'(m'+1)}\delta_{m,m'+1}+\sqrt{j(j+1)-m'(m'-1)}\delta_{m,m'-1}\right]\\
        & \braket{k,j,m|J_y|k',j',m'} = \frac{\hbar} 2\delta_{kk'}\delta_{jj'}\left[\sqrt{j(j+1)-m'(m'+1)}\delta_{m,m'+1}-\sqrt{j(j+1)-m'(m'-1)}\delta_{m,m'-1}\right]
    \end{align}
    
    由此看出这些矩阵元的数值只依赖于 $j,m$,而不依赖于 $k$。也就是说无论选定了怎样的额外算符 $A$,都可以用相同的方式来分析给定 $j,m$ 的矩阵元,具有相当的普适性。

\subsubsection{位置表象与轨道角动量}

轨道角动量是与经典力学联系最紧密的角动量。在位置表象下,在 $\bm L=\bm R\times\bm P$ 中代入位置和动量算符的具体形式,就得到了函数空间内的角动量分量算符的表达式
\begin{align}
        \hat L_x=\frac\hbar\i \left(y\frac{\partial}{\partial z}-z\frac{\partial}{\partial y}\right)\\
        \hat L_y=\frac\hbar\i \left(z\frac{\partial}{\partial x}-x\frac{\partial}{\partial z}\right)\\
        \hat L_z=\frac\hbar\i \left(x\frac{\partial}{\partial y}-y\frac{\partial}{\partial x}\right)
\end{align}

有时极坐标更容易反映问题本身的对称性,即这样的坐标变换:
\begin{equation}
    \begin{cases}
        x = r\sin\theta\cos\vphi\\
        y = r\sin\theta\sin\vphi\\
        z = r\cos\theta
    \end{cases},\quad \begin{cases}
        r\ge 0\\
        0\le \theta\le \pi\\
        0\le\vphi<2\pi
    \end{cases}
\end{equation}

代入后得到极坐标下算符的形式
\begin{align}
        \hat L_x &=\i\hbar\left(\sin\vphi\frac{\partial}{\partial\theta}+\frac{\cos\vphi}{\tan\theta}\frac{\partial}{\partial\vphi}\right)\\
        \hat L_y&=\i\hbar\left(-\cos\vphi\frac{\partial}{\partial\theta}+\frac{\sin\vphi}{\tan\theta}\frac{\partial}{\partial\vphi}\right)\\
        \hat L_z&=\frac\hbar\i \frac{\partial}{\partial\vphi}\\
        \hat{\bm L}^2 &= -\hbar^2\left(\frac{\partial^2}{\partial\theta^2}+\frac 1{\tan\theta}\frac{\partial}{\partial\theta}+\frac 1{\sin^2\theta}\frac{\partial^2}{\partial\vphi^2}\right)\label{eq:plr_crdnt_L^2}\\
        \hat L_+ &= \hbar\e^{\i\vphi}\left(\frac{\partial}{\partial\theta}+\i\cot\theta\frac{\partial}{\partial\vphi}\right)\\
        \hat L_- &= \hbar\e^{-\i\vphi}\left(-\frac{\partial}{\partial\theta}+\i\cot\theta\frac{\partial}{\partial\vphi}\right)
\end{align}

值得注意的是,角动量算符各分量都不含径向坐标 $r$,因此函数空间中的角动量本征值问题就是关于角分量 $\theta,\vphi$ 的偏微分方程组,其解是本征矢在位置表象下得到的本征函数,记为 $\mr Y_l^m(\theta,\vphi)$,满足的方程简写为
\begin{equation}
    \begin{cases}
        \hat{\bm L}^2 \mr Y_l^m(\theta,\vphi)=l(l+1)\hbar^2 \mr Y_l^m(\theta,\vphi)\\
        \hat L_z\mr Y_l^m(\theta,\vphi)=m\hbar \mr Y_l^m(\theta,\vphi)
    \end{cases}
\end{equation}

这种从 $\bm R$ 和 $\bm P$ 导出的角动量比仅通过对易关系定义的广义角动量多了一层特殊性,详见以下结论:

\begin{theorem}[$L_z$ 的本征值]
   当 $\bm L$ 是轨道角动量算符时,若 $m\hbar$ 是 $L_z$ 的本征值,则 $m$ 只能为整数
\end{theorem}
\begin{proof}
具体考虑 $L_z$ 的本征方程,即
\begin{equation}
    \frac\hbar\i \frac{\partial}{\partial\vphi}\mr Y_l^m(\theta,\vphi) = m\hbar \mr Y_l^m(\theta,\vphi)
\end{equation}
单独对 $\vphi$ 变量求解这个方程,能够得到
\begin{equation}\label{eq:sphrcl_hrmnc_vrbl_sprtn}
    \mr Y_l^m(\theta,\vphi) = F_l^m(\theta)\e^{\i m\vphi}
\end{equation}
的形式。而为了保证角动量算符中的偏导数总有意义,波函数在空间各点必然是连续的,这要求周期边界条件:
\begin{equation}\label{eq:prdc_bndry_cndtn}
    \mr Y_l^m(\theta,\vphi=0)=\mr Y_l^m(\theta,\vphi=2\pi)
\end{equation}
定理 \ref{thm:anglr_mmntm_spctrm} 表明 $l$ 只能是半整数(奇数的一半)或整数,而 $m$ 与 $l$ 的差恒为整数,因此 $m$ 也是半整数或整数。而要满足上式,就必须有 $\e^{\i m\pi}=1$,也就是说 $m$ 只能是整数。
\end{proof}

受到 $m$ 的规约,$l$ 亦只能是整数,这就是轨道角动量的特殊之处,这是由周期性边界条件决定的:我们要求变量 $\vphi$ 环绕一周,本征函数必须回到原位。而周期性边界条件来自于物理上的自然考量,一个建立在真实空间坐标系中的\textbf{角向函数}必须满足的性质。这进一步暗示了半整数的 $m$ 所代表的角动量并不存在与空间坐标所对应的物理实在。

\begin{assumption}
在研究角动量算符的本征值问题时,我们可以冻结无关算符 $A$ 对应的指标 $k$,实际上讨论的就是单个子空间 $\ms E(k)$,它对应于与径向变量无关的角向函数空间 $\ms E_\Omega$。
\end{assumption}

于是就可以用上节得到的一般化结论,从函数空间和态空间两端处理问题。

\begin{theorem}[$\bm L^2$ 的本征值和本征函数]\label{thm:orbtl_anglr_mmntm_spctrm}
   当 $\bm L$ 是轨道角动量算符时,若 $l(l+1)\hbar^2$ 是 $\bm L^2$ 的本征值,则 $l$ 将取到全体自然数;且在与 $\bm L$ 无关的观测算符有关的 $k$ 固定的情况下,对每个 $-l\le m\le l$ 都存在非简并的态 $\ket{k,l,m}\in\ms E(k)$,也即存在唯一的(相差一个常数因子)$\mr Y_l^m(\theta,\vphi)$。
\end{theorem}
\begin{proof}
    升算符的性质使得在函数空间中要求 $\hat L_+\mr Y_l^l(\theta,\vphi)=0$,代入分离变量结果 \ref{eq:sphrcl_hrmnc_vrbl_sprtn} 中,得到
    \begin{equation}
        \left(\frac\d{\d\theta}-l\cot\theta\right)F_l^l(\theta)=0
    \end{equation}
    通解为
    \begin{equation}
        F_l^l(\theta)=c_l(\sin\theta)^l
    \end{equation}
    这里只有一个待定因子 $c_l$,于是对于任意给定的自然数 $l$,都存在唯一的函数
    \begin{equation}
        \mr Y_l^l(\theta,\vphi)=c_l(\sin\theta)^l\e^{\i l\vphi}
    \end{equation}
    也即 $\ket{k,l,l}$ 是唯一的。无论从微分方程定解的角度,还是在考虑 $k$ 后对 C.S.C.O. 的性质要求,这种唯一性都不出人意料。再用 $L_-$ 作用于其上,(如果 $l\le 0$)就会得到 $\ket{k,l,m=l-1},\cdots,\ket{k,l,m=-l}$ 的一系列态矢,它们作为 $\ms E(k,l)$ 的基自然也是唯一的。在位置表象下采用球坐标系,就得到了这些 $\mr Y_l^m(\theta,\vphi)$。
\end{proof}

这一族被指标 $l,m$ 确定的函数 $\mr Y_l^m(\theta,\vphi)$ 称为\textbf{球谐函数}。而由于 $\ms E(k)$ 显然拥有基 $\{\ket{k,l,m}\}_{l,m}$,所以态矢的正交归一性对应波函数的正交归一性,只需选定常数因子就能使得
\begin{equation}
    \int_0^{2\pi}\d\vphi\int_0^\pi\sin\theta\d\theta \mr Y_{l'}^{m'*}(\theta,\vphi)Y_l^m(\theta,\vphi)=\delta_{ll'}\delta_{mm'}
\end{equation}

同时态矢 $\ket{\psi_k}\in\ms E(k)$ 在基 $\{\ket{k,l,m}\}_{l,m}$ 的展开表现为任何可能的角向波函数 $f(\theta,\vphi)\in\ms E_\Omega$ 都可以用球谐函数展开,注意到已证的 $l,m$ 的取值完全性,就有
\begin{equation}
    f(\theta,\vphi)=\sum_{l=0}^\infty\sum_{m=-l}^l c_{l,m}\mr Y_l^m(\theta,\vphi)
\end{equation}
其中
\begin{equation}
    c_{l,m} = \braket{\psi_k}{k,l,m} = \int_0^{2\pi}\d\vphi\int_0^\pi\sin\theta\d\theta\mr Y_l^{m*}(\theta,\vphi)f(\theta,\vphi)
\end{equation}
因此可以看出,角动量的本征值问题与束缚态问题类似,被周期边界条件 \ref{eq:prdc_bndry_cndtn} 限定后,所得离散的球谐函数解反映体系的对称性。

利用矩阵元关系 \ref{eq:anglr_mmntm_mtrx_elmnt},在函数空间中可以写出球谐函数的递推式,从而求出所有球谐函数
\begin{equation}\label{eq:sphrcl_hrmnc_rcrsn_frml}
    \e^{\pm\i\vphi}\left(\pm\frac{\partial}{\partial\theta}-m\cot\theta\right)\mr Y_l^m(\theta,\vphi)=\sqrt{l(l+1)-m(m\pm 1)}\mr Y_l^{m\pm1}(\theta,\vphi)
\end{equation}

接下来将 $k$ 纳入讨论,考虑整个 $\ms E$ 空间及其上的 C.S.C.O. 的本征矢构成的角动量标准表象 $\{\ket{k,l,m}\}$,放入位置表象之中就是讨论波函数空间 $\ms E_{\bm r}$ 和波函数标准基 $\{\psi_{k,l,m}(\bm r)\}$。于是根据上一节的讨论:从 $\ms E(l,m=l)$ 开始构筑标准基,对应于波函数 $\{\psi_{k,l,l}(\bm r)\}_k$,然后再用 $\hat L_-$ 逐次作用于其上,就构成了各 $\psi_{k,l,m}(\bm r)$。它们满足的方程是
\begin{align}
    &\hat{\bm L}^2\psi_{k,l,m}(\bm r) = l(l+1)\hbar^2\psi_{k,l,m}(\bm r)\\
    &\hat L_z\psi_{k,l,m}(\bm r) = m\hbar \psi_{k,l,m}(\bm r)\\
    &L_\pm \psi_{k,l,m}(\bm r) = \hbar\sqrt{l(l+1)-m(m\pm 1)}\psi_{k,l,m\pm 1}(\bm r)\label{eq:orbtl_anglr_mmntm_egnfnctn_3}
\end{align}
然而基于两条事实:
\begin{enumerate}
    \item 算符 $\hat{\bm L}^2,\hat L_z$ 都与径向变量 $r$ 无关
    \item 仅含角向变量的 $\mr Y_l^m(\theta,\vphi)$ 是 $\hat{\bm L}^2, \hat L_z$ 本征值问题的解,且已经证明了这一仅对角向变量其作用的偏微分方程的解只能是 $\mr Y_l^m(\theta,\vphi)$。
\end{enumerate}
这说明标准基在位置表象下具有分离变量的形式:
\begin{equation}
    \psi_{k,l,m}(\bm r)=R_{k,l,m}(r)\mr Y_l^m(\theta,\vphi)
\end{equation}

此时重新代回 \ref{eq:orbtl_anglr_mmntm_egnfnctn_3},由于该算符对 $r$ 无效,将会发现
\begin{equation}
    L_\pm\psi_{k,l,m}(\bm r)=R_{k,l,m}(r)L_\pm Y_l^m(\theta,\vphi)=\hbar\sqrt{l(l+1)-m(m\pm 1)}R_{k,l,m}(r)L_\pm Y_l^{m\pm 1}(\theta,\vphi)
\end{equation}
比较可得 $R_{k,l,m\pm 1}(r)=R_{k,l,m}(r)$,这就意味着径向函数与 $m$ 无关。因此,轨道角动量标准基 $\ket{k,l,m}$ 对应的位置表象波函数的形式进一步简化为

\begin{equation}\label{eq:orbtl_anglr_mmntm_wv_fnctn}
    \psi_{k,l,m}(\bm r)=R_{k,l}(r)\mr Y_l^m(\theta,\vphi)
\end{equation}
由于球谐函数已经被要求归一化,为了整体的归一化,还要附加径向函数归一化关系式的要求:

\begin{equation}
    \int_0^{\infty}r^2\d r R_{k,l}^*(r)R_{k',l}(r)=\delta_{kk'}
\end{equation}

\subsection{氢原子}
以上的抽象模型已经为我们奠定了处理实际问题的基础,而氢原子正是为数不多的能在量子力学水平下得到精确求解的实际体系,它是经典力学中以有心力形式相互作用的二体模型的代表。氢原子核(通常为一个质子)与电子之间存在静电吸引力,由于质子的质量远大于电子,通常可以近似为电子围绕固定的质子,即电子在质子产生的中心力场中运动。

因此,首先需要对一般的中心力场问题进行分析:它的哈密顿量的形式是什么?哪些算符构成 C.S.C.O.?本征系统的结构怎样?与之前讨论过的模型有哪些联系?由于势场本身依赖于位置,所以必须充分利用位置表象来研究中心力场问题,而经典力学的结论表明,中心力场问题中角动量守恒,这可能意味着角动量的本征值问题将起到重要作用。

\subsubsection{中心力场中的分离变量}
球坐标系是更简便的选择,因为所涉及的势能 $V(\bm r)= V(r)$。球坐标系中的 Laplace 算符形如:
\begin{equation}\label{eq:sphrcl_crdnt_lplc}
    \Delta = \frac 1 r\frac{\partial^2}{\partial r^2}r+\frac 1{r^2}\left(\frac{\partial^2}{\partial\theta^2}+\frac 1{\tan\theta}\frac{\partial}{\partial\theta}+\frac 1{\sin^2\theta}\frac{\partial^2}{\partial\vphi^2}\right)
\end{equation}

代入定态薛定谔方程 \ref{eq:sttnry_stt_schrdngr_eqtn},比照位置表象下角动量算符 $\hat{\bm L}^2$ 的表达式 \ref{eq:plr_crdnt_L^2},对于一个质量为 $\mu$ 的粒子,哈密顿算符的形式为
\begin{equation}
    \hat H = -\frac{\hbar^2}{2\mu}\frac 1 r\frac{\partial^2}{\partial r^2}r+\frac 1{2\mu r^2}\hat{\bm L}^2+V(r)
\end{equation}
可见角向变量包含在 $\hat{\bm L}^2$ 项中,而径向变量单独成项,它们的明确物理意义分别为径向运动对动能的贡献和角向运动对动能的贡献。在函数空间一侧,我们知道 $\hat{\bm L}$ 的三个分量都只作用于 $\theta,\vphi$,从而与只含 $r$ 的项对易;另外这些分量也与 $\hat{\bm L}^2$ 对易,因此总体而言与 $H$ 对易。

于是返回态空间一侧,从上述的 $[H,\bm L]=0$ 可知,$H,\bm L^2,L_z$ 两两对易。于是 $H$ 担当了角动量的讨论中额外算符 $A$ 的角色,它可能与角动量的两个算符构成体系的 C.S.C.O.——至少,我们可以期待上述本征方程的解 $\psi(r,\theta,\vphi)$ 同时是 $\bm L^2,L_z$ 的本征函数,而这种本征函数的形式是已知的,根据 \ref{eq:orbtl_anglr_mmntm_wv_fnctn} 处的结论,它一定形如
\begin{equation}
    \psi(\bm r)=R(r)\mr Y_l^m(\theta,\vphi)
\end{equation}
有待解决的问题只剩下求解 $R(r)$ 使得 $\psi(\bm r)$ 是 $\hat H$ 的本征函数。之前曾在存在某额外算符 $A$,其本征值用指标 $k$ 标记的情境下进一步讨论此问题,并得出结论 $R(r)=R_{k,l}(r)$。

\begin{remark}
但对此处而言,尚未证明 $H$ 的加入已经组成 C.S.C.O.,这意味着即使用 $k$ 标记了 $H$ 的诸本征值,也不能确保这样写出的 $R_{k,l}(r)$ 是唯一的:这需要进一步的讨论和分析,但这个记号可以暂时得到认可(即使它代表的是一族线性无关的函数)。
\end{remark}

于是我们将上式代入本征值方程,从两边消去球谐函数,整理得到径向方程:
\begin{equation}\label{eq:rdl_eqtn_cntl_fld}
    \left[-\frac{\hbar^2}{2\mu}\frac 1 r\frac{\d^2}{\d r^2}r+\frac{l(l+1)\hbar^2}{2\mu r^2}+V(r)\right]R_{k,l}(r)=E_{k,l}R_{k,l}(r)
\end{equation}
从方程的形式也可以看出,其解一定依赖于参数 $l$ 而不依赖于 $m$,该方程统领了给定 $l$ 时 $2l+1$ 个子空间 $\ms E(l,m)$ 中位置表象下的本征值问题。因此,其中 $E_{k,l}$ 代表 $H$ 在子空间 $\ms E(l,m)$ 中的第 $k$ 个本征值,而 $R_{k,l}$ 代表与这个本征值联系的、$\ms E(l,m)$ 的位置表象下所有可能的本征函数。

令 $u(r)=R(r)/r$,得到化简后的微分方程,为
\begin{equation}\label{eq:rdcd_rdl_eqtn_cntl_fld}
    \left[-\frac{\hbar^2}{2\mu}\frac{\d^2}{\d r^2}+\frac{l(l+1)\hbar^2}{2\mu r^2}+V(r)\right]u_{k,l}(r)=E_{k,l}u_{k,l}(r)
\end{equation}
因为这个方程是二阶的,在不加任何额外限制时它应该有两个线性无关解。

单纯考虑径向坐标 $r$,可以将方程左侧分为仅含二阶导的径向动能项和一个有效势能项
\begin{equation}\label{eq:effctv_ptntl_rdl_cntrl_fld}
    V_\text{eff}(r)=V(r)+\frac{l(l+1)\hbar^2}{2\mu r^2}
\end{equation}
有效势能的第一项是中心力场本身带来的,可能是吸引性的($V(r)<0$)或排斥性的($V(r)>0$);第二项称为离心项,具有排斥作用,分母上的 $r^2$ 意味着越靠近力心,在 $V_\text{eff}$ 中排斥作用越占主导地位。

\begin{assumption}
在实际的物理体系中,我们可以期望 $V(r)$ 在趋近力心时保持有限值,或至少比 $1/r$ 趋近无穷大更慢。
\end{assumption}

利用这一假设可以分析径向方程的渐近行为:我们假设 $R(r)$ 的解经幂级数展开后最低次项为 $s$ 次,这样更高次的项都是 $r\ra 0$ 时的小量,也就是说原点附近有
\begin{equation}
    R_{k,l}(r)\sim Cr^s,\quad r\ra 0
\end{equation}
代入径向方程 \ref{eq:rdl_eqtn_cntl_fld},得到 $-s(s+1)+l(l+1)=0$,从而可知 $s=l$ 或 $s=-l-1$。这刻画了给定 $E_{k,l}$ 时方程的两个可能的线性无关解在原点处的行为。但回顾方程形式,以及最初的本征值问题中 Laplace 算符的形式,方程系数分母上的 $r$ 意味着力心处是方程的奇点,为了使含有微分运算的算符有意义,必须舍弃 $s=-l-1$ 的解,只保留 $s=l$ 的解,此时有
\begin{equation}
    u_{k,l}(r)\sim Cr^{l+1},\quad r\ra 0
\end{equation}
因此 $u(r)$ 的幂级数展开的最低阶项的次数是正数,这意味着它的函数形式一定满足 $u_{k,l}(0)=0$。这同时也使得 $R_{k,l}(r)$ 代表唯一的线性无关解,$k$ 指标的使用变得名正言顺。

\begin{remark}
从另一种角度来理解,化简后 $u$ 的方程  \ref{eq:rdcd_rdl_eqtn_cntl_fld} 实际上是一个一维问题,其中有效势在正半轴体现为 \ref{eq:effctv_ptntl_rdl_cntrl_fld},而在非正半轴接续为正无穷。因此对应的波函数 $u_{k,l}(r)$ 在负半轴上取值一定恒为 0,根据衔接条件中的连续性条件,在原点处的取值也一定为 0。
\end{remark}

\begin{conclusion}
至此,一般的中心力场 $V(r)$ 中粒子的波函数是带有三个指标的函数 $\psi_{k,l,m}=R_{k,l}(r)\mr Y_l^m(\theta,\vphi)$,它同时是 C.S.C.O. $H,\bm L^2,L_z$ 的本征函数,对应的本征值分别为与 $V(r)$ 具体有关的 $E_{k,l}$ 和已经确定的 $l(l+1)\hbar^2,m\hbar$。此时 $k$ 称为\textbf{径向量子数},$l$ 称为\textbf{角量子数},$m$ 称为\textbf{磁量子数}。前两者与波函数径向部分的形式相关,后两者决定了角向部分的形式。
\end{conclusion}

\subsubsection{玻尔模型}

早在量子力学的现代理论建立之前,玻尔(Bohr)就已经提出了一套半经典的经验理论来解释氢原子的电子结构。

\begin{assumption}\textbf{定态运动}

玻尔认为电子在质子的作用下在以 $r$ 为半径的圆形轨道上作匀速圆周运动。

如果电子在系统中呈现的质量为 $\mu$,其运动速率为 $v$,电荷量为 $q$,则其遵守以下两个经典力学和经典电磁学推出的方程。

库仑力充当向心力:
\begin{equation}
    \frac{\mu v^2} r=\frac{e^2}{r^2},\quad e^2=\frac{q^2}{4\pi\ve_0}
\end{equation}

总能量是动能和电势能(以无穷远处为势能零点)之和:
\begin{equation}
    E=\frac 1 2{\mu v^2}-\frac{e^2} r
\end{equation}
\end{assumption}

\begin{assumption}\textbf{角动量量子化}

   玻尔认为电子的轨道角动量满足量子化条件,具体体现为
    \begin{equation}
        \mu vr=n\hbar,\quad n\in\N_+
    \end{equation}
\end{assumption}

这些假设能够揭示里德堡(Ridburg)观察到的氢原子光谱规律。利用以上三式可以计算出一些重要的表达式。
\begin{conclusion}氢原子玻尔模型的重要物理量

轨道能量:
\begin{equation}
    E_n=-\frac 1{n^2}E_I,\quad E_I=\frac{\mu e^4}{2\hbar^2}
\end{equation}

轨道半径:
\begin{equation}
    r_n = n^2 a_0,\quad a_0=\frac{\hbar^2}{\mu e^2}
\end{equation}
其中 $n=1$ 对应的轨道能量 $E_I$ 又称作氢原子的\textbf{电离能},即剥夺一个电子到无穷远处所需的能量;而轨道半径 $a_0$ 反映了氢原子的大小,称为\textbf{玻尔半径}。
\end{conclusion}

但是角动量量子化缺乏坚实的理论依据,定态假设也与经典电动力学中“加速运动的带电粒子产生电磁波,损失的能量将使得核外电子坠入原子核”的结论不符,因此势必是一个过渡理论。从量子力学的观点重新审视,认为电子沿着某个确定的轨道作匀速圆周运动也与不确定性原理相违背,但是从这一模型中计算得到的 $E_I,a_0$ 的数值在一定程度上是精确的,在接下来的量子力学讨论中也会沿用。

\subsubsection{径向波函数}

氢原子问题是特殊的中心力场问题,具体体现在库仑势的形式 $V(r)\propto -1/r$,它符合前面讨论对 $V(r)$ 的假想,同时也给中心力场的普遍结果带来了更多的特殊性。

在具体问题中,首先要让方程中所涉及的物理量:径向坐标 $r$ 和能量 $E_{k,l}$ 无量纲化,而玻尔模型提供的玻尔半径和电离能正是一个恰当的参照物。因此引入无量纲量
\begin{equation*}
    \rho = \frac r {a_0},\quad \lambda_{k,l} = \sqrt{\frac{-E_{k,l}}{E_I}}
\end{equation*}
\begin{remark}
由于氢原子是束缚态,无穷远处静止粒子对应能量零点,$E_I$ 是一个正值,可以预料到总能 $-E_{k,l}$ 一定是负的,所以 $\lambda_{k,l}$ 的设定能保证是实数。
\end{remark}

代入约化后的径向方程 \ref{eq:rdcd_rdl_eqtn_cntl_fld},得到
\begin{equation}
    \left[\frac{\d^2}{\d\rho^2}-\frac{l(l+1)}{\rho^2}+\frac 2\rho-\lambda_{k,l}^2\right]u_{k,l}(\rho)=0
\end{equation}

在中心力场的普遍讨论中,我们已经对 $\rho\ra0$ 处的渐近行为进行了分析,并最终得到了 $u_{k,l}(0)=0$ 的重要结论。而在氢原子问题中,还可以对 $\rho\ra\infty$ 处的渐近行为进行分析:此时分母上含 $\rho$ 的项可以忽略,方程变为
\begin{equation}
    \left(\frac{\d^2}{\d\rho^2}-\lambda_{k,l}^2\right)u_{k,l}(\rho)=0
\end{equation}
一组线性无关的解是 $u_{k,l}(\rho)=\e^{\pm\rho\lambda_{k,l}}$。然而为了使波函数能容纳于平方可积函数空间,无穷远处必然是有界的,因此只能保留 $\e^{-\rho\lambda_{k,l}}$ 的项。然后引入一个额外的因子,并期待对这个解在有限远处的行为进行修正,也即尝试以下函数变换:
\begin{equation}
    u_{k,l}(\rho)=\e^{-\rho\lambda_{k,l}}y_{k,l}(\rho)
\end{equation}
仍然继承力心处的条件 $y_{k,l}(0)=0$,再将其代回方程,得到
\begin{equation}
    \left\{\frac{\d^2}{\d\rho^2}-2\lambda_{k,l}\frac{\d}{\d\rho}+\left[\frac 2\rho-\frac{l(l+1)}{\rho^2}\right]\right\}y_{k,l}(\rho)=0
\end{equation}

然后用幂级数解法处理该方程,设展式:
\begin{equation}
    y_{k,l}(\rho) = \rho^s\sum_{q=0}^{\infty}c_q\rho^q
\end{equation}
其中 $c_0\ne 0$,力心处的条件推知 $s>0$。将这一幂级数代入方程左端并交换微分和求和顺序,整理后与右端比较各项系数,应知全体系数都为 0。

首先是最低次的 $\rho^{s-2}$ 项,其系数为零得到指标方程:
\begin{equation}
    [-l(l+1)+s(s-1)]c_0=0
\end{equation}
由于 $c_0\ne 0$,得到 $s=l+1$ 或 $s=-l$(舍)。然后考虑任意项,代入 $s=l+1$,就得到递推关系:
\begin{equation}
    q(q+2l+1)c_q = 2[(q+1)\lambda_{k,l}-1]c_{q-1}
\end{equation}
只要取定一个非零的 $c_0$,就可以递推出全体系数,而且会发现 $q\ra\infty$ 时
\begin{equation}\label{eq:hydrgn_atm_rdl_cffcnt_rcrrnc_rltn}
    \frac{c_q}{c_{q-1}}\sim\frac{2\lambda_{k,l}} q\ra 0,\quad q\ra\infty
\end{equation}
由幂级数的性质可得其收敛半径为 $\infty$,这意味着此幂级数在全复平面都是方程的解的一部分。

但是此时如果反观 $u_{k,l}(\rho)$ 和 $y_{k,l}(\rho)$ 之间的关系:如果 $y_{k,l}(\rho)$ 在无穷远处的渐近行为类似于 $\e^{2\rho\lambda_{k,l}}$,则前面对无穷远处渐近行为的约束将功亏一篑。然而如果对所担心的 $\e^{2\rho\lambda_{k,l}}$ 幂级数展开
\begin{equation}
    \e^{2\rho\lambda_{k,l}} = \sum_{q=0}^\infty d_q\rho^q,\quad d_q = \frac{(2\lambda_{k,l})^q}{q!}\implies\frac{d_q}{d_{q-1}} = \frac{2\lambda_{k,l}} q
\end{equation}
就会观察到它与~\ref{eq:hydrgn_atm_rdl_cffcnt_rcrrnc_rltn} 所解释的渐进性质相同。这就意味着,我们所讨论的 $y_{k,l}(\rho)$ 在无穷远处(此时 $q$ 大的项占主导)不幸地和 $\e^{2\rho\lambda_{k,l}}$ 同流合污,从而可能使 $u_{k,l}(\rho)$ 脱离平方可积的函数空间。因此唯一的出路是 $y_{k,l}(\rho)$ 被截断成多项式而非幂级数,而截断条件能从递推关系 \ref{eq:hydrgn_atm_rdl_cffcnt_rcrrnc_rltn} 中看出:即存在某个整数 $k$ 使得 $q=k$ 时 $(k+1)\lambda_{k,l}-1=0$,由于 $c_0\ne 0$,$k$ 只能从 1 开始起步从 $c_1$ 处截断,也即
\begin{equation}\label{eq:hydrgn_atm_enrgy_egnvl}
    \lambda_{k,l} = \frac 1{k+l},\quad E_{k,l} = \frac{-E_I}{(k+l)^2},\quad k\in \N_+
\end{equation}
从而形成了能量的量子化。

\begin{conclusion}
$y_{k,l}(\rho)$ 是最低次项为 $\rho^{l+1}$ 项,最高次项为 $\rho^{l+k}$ 项的多项式,系数的通项公式为
\begin{equation}
    c_q=(-1)^q\left(\frac 2{k+l}\right)^q\frac{(k-1)!}{(k-q-1)!}\frac{(2l+1)!}{q!(q+2l+1)!}c_0
\end{equation}
然后用归一化条件和相位约定确定唯一的 $c_0$,代回所有的函数变换和变量代换,就能得到径向函数 $R_{k,l}(r)$。
\end{conclusion}

\subsubsection{量子数的物理意义}

从 \ref{eq:hydrgn_atm_enrgy_egnvl} 可以看出,当 $k+l$ 相同而 $k,l$ 不同时能级仍然发生了偶然的简并,这是氢原子特殊的对称性导致的。因此我们定义\textbf{主量子数} $n=k+l$:氢原子各能级表示为
\begin{equation}\label{eq:hydrgn_enrgy_lvl}
    E_n=-\frac 1{n^2}E_I
\end{equation}
可见主量子数决定了能级的能量,$n$ 相同的能级构成一个电子\textbf{壳层}。

由于 $k$ 是正整数,每个 $n$ 相关的 $l=0,1,\cdots,n-1$ 共 $n$ 个值,在 $n$ 给定时各自对应的态称为 $n$ 个\textbf{支壳层}。尽管这些态能量本征值都相同,但\textbf{角量子数} $l$ 确定了 $\bm L^2$ 的本征值,也即角动量的大小。

而在对角动量的讨论中已经知道,每个固定的 $l$ 又关联着 $(2l+1)$ 个\textbf{磁量子数} $m$ 的值,分别对应于不同的\textbf{轨道}。这些值决定了态的 $L_z$ 本征值,即角动量在给定方向上的分量大小。

至此,能级 $E_n$ 的总简并度为
\begin{equation}
    g_n=\sum_{l=0}^{n-1}(2l+1)=n^2
\end{equation}

氢原子的电子波函数也转而用这三个量子数来表示:
\begin{equation}
    \psi_{n,l,m}(\bm r) = R_{n,l}(r)\mr Y_l^m(\theta,\vphi)
\end{equation}
其中 $R_{n,l}(r)$ 称为氢原子的\textbf{径向波函数},$Y_l^m(\theta,\vphi)$ 又称为氢原子的\textbf{角向波函数}。

光谱学的惯例使我们有以下命名规则来指代波函数:
\begin{table}[!hbtp]\label{tab:shell}
    \centering
    \caption{壳层符号}
    \begin{tabular}{c|cccccccc}
        \toprule
        主量子数 $n$ & 1 & 2 & 3 & 4 & 5 & 6 & 7 & $\cdots$ \\
        \midrule
        符号 & K & M & N & O & P & Q &R & $\cdots$\\
        \bottomrule
    \end{tabular}
\end{table}

\begin{table}[!hbtp]\label{tab:subshell}
    \centering
    \caption{支壳层符号}
    \begin{tabular}{c|ccccccc}
    \toprule
        角量子数 $l$ & 0 & 1 & 2 & 3 & 4 & 5 & $\cdots$ \\
        \midrule
        符号 & $s$ & $p$ & $d$ & $f$ & $g$ & $h$ & $\cdots$\\
        \bottomrule
    \end{tabular}
\end{table}

考虑电子在点 $(r,\theta,\vphi)$ 附近的体积元 $\d\bm r=r^2\d r\sin\theta\d\theta\d\vphi$ 中出现的概率
\begin{equation}
    \d\P_{n,l,m}(r,\theta,\vphi) = |\psi_{n,l,m}(r,\theta,\vphi)|^2\d r\sin\theta\d\theta\d\vphi = |R_{n,l}(r)|^2r^2\d r\cdot|Y_l^m(\theta,\vphi)|^2\sin\theta\d\theta\d\vphi
\end{equation}
如果固定 $\theta,\vphi$ 和立体角中的 $\d\theta,\d\vphi$,则在 $r$ 和 $r+\d r$ 之间找到电子的概率正比于 $r^2|R_{n,l}|^2\d r$。$D_{n,l}(r) = r^2|R_{n,l}(r)|^2$ 称为氢原子的\textbf{径向概率分布函数}。

对于 $l = n-1\ (k=n-l=1)$ 的情况,按照对径向波函数的分析,多项式 $y_{k=1,l}(\rho)$ 只有 $\rho^{l+1}$ 一项,从而在 $R_{n,l}(r)$ 中只有 $(r/a_0)^{n-1}$ 一项,因此
\begin{equation}
    D_{n,l}(r) \propto \frac{r^2}{a_0^2}\left[\left(\frac r{a_0}\right)^{n-1}\e^{-\frac{r}{na_0}}\right]^2 = \left(\frac r{a_0}\right)^{2n}\e^{\frac{-2r}{na_0}}
\end{equation}
\begin{remark}
该函数的唯一极大值位于 $r=r_n=n^2 a_0$ 处,即玻尔预言的对应于能量 $E_n$ 的轨道半径。于是,无论是轨道能量和轨道半径,量子力学都得到了与玻尔的半经典预言相称的结果。
\end{remark}

再次观察多项式因子的特点,会发现只有 $l=0$ 时才会在 $y_{k,l=0}(\rho)$ 中出现 $\rho^1$ 项,从而在 $R_{n,l}(r)$ 的多项式因子中出现常数项。这就得到以下重要结论:
\begin{corollary}[$s$ 轨道的特殊性]
只有各 $l=0$ 的态($s$ 态),在原点附近找到电子的概率才不为 0,也即 $D_{n,l}(0)\ne 0$ 当且仅当 $l\ne 0$。
\end{corollary}

化学上经常用主量子数和支壳层符号来共同表示一个能级,如 $1s,2p,3d,4f$ 等。而回忆球谐函数的分离变量 \ref{eq:sphrcl_hrmnc_vrbl_sprtn},球谐函数对 $\vphi$ 的依赖完全体现为因子 $\e^{\i m\vphi}$。因此,对于 $n,l$ 都相同的氢原子轨道,注意到
\begin{equation}
    \e^{\i m\vphi} + \e^{-\i m\vphi}\propto \cos m\vphi;\ \e^{\i m\vphi} - \e^{-\i m\vphi}\propto \sin m\vphi
\end{equation}
据此重新适当选取归一化系数 $c_0$,就能得到实波函数。

\begin{remark}
例如对 $l=1,m=\pm 1$ 的波函数进行叠加,提取球谐函数部分,有
\begin{equation}
    \frac{1}{\sqrt 2}(\mr Y_1^1+\mr Y_1^{-1}) = \frac{1}{\sqrt 2}\sqrt{\frac{3}{8\pi}}\sin\theta\left(\e^{\i\vphi}+\e^{-\i\vphi}\right) = \frac 1 2\sqrt{\frac 3 \pi}\sin\theta\cos\vphi = \frac 1 2\sqrt{\frac 3 \pi}\frac x r
\end{equation}
注意到此时不仅将球谐函数在保持归一性的前提下组合成了实函数,还使角向部分显式地出现了直角坐标系下的 $x$ 分量。在乘上径向波函数,得到的总波函数对应的轨道称为 $p_x$。

同理可证
\begin{equation}
    \frac{1}{\sqrt 2\i}(\mr Y_1^1-\mr Y_1^{-1}) = \frac 1 2\sqrt{\frac 3 \pi}\sin\theta\sin\vphi=\frac 1 2\sqrt{\frac 3 \pi}\frac y r
\end{equation}
得到的总波函数对应的轨道称为 $p_y$。

而 $l = 1,m=0$ 的波函数是无需处理的,因为角向部分形如
\begin{equation}
    \mr Y_1^0 = \sqrt{\frac 3{4\pi}}\cos\theta=\sqrt{\frac 3{4\pi}}\frac z r
\end{equation}
已经是实函数,其总波函数对应的轨道称为 $p_z$。事实上对于所有 $m=0$ 的轨道都如此。
\end{remark}

类似地,可以将各组 $n,l,\pm m$ 的波函数都进行类似处理,得到一系列实波函数,并找到它们与 $x,y,z,r$ 的关系进行命名。如 $l=2$ 的诸轨道被命名为 $d_{z^2},d_{xz},d_{yz},d_{x^2-y^2},d_{xy}$,由于都是两个波函数重新组合为两个波函数,产生的新波函数的数量仍为 $2l+1$ 个。再结合主量子数,就可以简单地用 $1s, 2p_z, 3d_{x^2-y^2}$ 这样的标记唯一确定氢原子的某个电子轨道。

\section{多自由度问题初探}

当体系中出现多个自由度时,这些自由度可能是无法分离考虑的。为了描述整个体系的态和各个自由度的态之间的联系,需要引入一类新的代数工具。

\subsection{张量积}

\subsubsection{张量积空间}

\begin{definition}[张量积]
    $\ms E=\ms E_1\ox \ms E_2$,即态空间 $\ms E_1$ 和 $\ms E_2$ 的\textbf{张量积},是指对于任意的 $\ket{\vphi(1)}\in\ms E_1,\ket{\chi(2)}\in\ms E_2$,都存在一个矢量 $\ket{\vphi(1)}\ox\ket{\chi(2)}\in\ms E$ 与之对应,它也称为 $\ket{\vphi(1)}\in\ms E_1$ 和 $\ket{\chi(2)}\in\ms E_2$ 的张量积。且这种对应关系要满足以下条件
   \begin{enumerate}
       \item 对数乘的线性:
       \begin{align}
           & [\lambda\ket{\vphi(1)}]\ox\ket{\chi(2)} = \lambda[|\ket{\vphi(1)}\ox\ket{\chi(2)}]\\
           & \ket{\vphi(1)}\ox[\mu\ket{\chi(2)}] = \mu[|\ket{\vphi(1)}\ox\ket{\chi(2)}]
       \end{align}
       \item 加法分配律:
       \begin{align}
           & \ket{\vphi(1)}\ox[\ket{\chi_1(2)}+\ket{\chi_2(2)}] = \ket{\vphi(1)}\ox\ket{\chi_1(2)}+\ket{\vphi(1)}\ox\ket{\chi_2(2)}\\
           & [\ket{\vphi_1(1)}+\ket{\vphi_2(1)}]\ox\ket{\chi(2)} = \ket{\vphi_1(1)}\ox\ket{\chi(2)}+\ket{\vphi_2(1)}\ox\ket{\chi(2)}
       \end{align}
   \end{enumerate}
\end{definition}

为了方便,常常省略张量积符号 $\ox$,将 $\ket{\vphi(1)}\ox\ket{\chi(2)}$ 记作 $\ket{\vphi(1)}\ket{\chi(2)}$ 甚至 $\ket{\vphi(1)\chi(2)}$。 基于以上定义,容易验证 $\ms E$ 也是一个向量空间,它的基也可以从 $\ms E_1$ 和 $\ms E_2$ 的基中构造而来:

\begin{theorem}[张量积空间的基]
   如果在空间 $\ms E_1,\ms E_2$ 中分别选定了基 $\{\ket{u_i(1)}\},\{\ket{v_l(2)}\}$,则诸 $\ket{u_i(1)v_l(2)}$ 构成了 $\ms E$ 的一个基。进一步,假设 $\ms E_1,\ms E_2$ 是有限维的,则 $\dim \ms E = \dim \ms E_1\cdot \dim \ms E_2$。
\end{theorem}

\begin{corollary}[张量积矢量在张量积空间中按基展开]\label{col:tnsr_prdct_vctr_expnsn}
    假设 $\ket{\vphi(1)}\in\ms E_1,\ket{\chi(2)}\in\ms E_2$ 在两个空间中分别按基展开的关系式为
    \begin{equation*}
        \ket{\vphi(1)} = \sum_i a_i\ket{u_i(1)},\quad \ket{\chi(2)} = \sum_l b_l\ket{v_l(2)}
    \end{equation*}
    则张量积矢量 $\ket{\vphi(1)\chi(2)}$ 在基 $\{\ket{u_i(1)v_l(2)}\}$ 中的展开式为
    \begin{equation}
        \ket{\vphi(1)}\ox\ket{\chi(2)} = \sum_{i,l} a_i b_l\ket{u_i(1)}\ox\ket{v_l(2)}
    \end{equation}
    
\end{corollary}

\begin{remark}
但这并不意味着对于任意的 $\ket{\psi}\in\ms E_1\ox\ms E_2$,都存在 $\ket{\vphi(1)}\in\ms E_1,\ket{\chi(2)}\in\ms E_2$ 使得 $\ket{\psi} = \ket{\vphi(1)\chi(2)}$。在具体的表象下,这一结论的后果会得到更直观地阐释。
\end{remark}

从原有空间的内积结构可以诱导出张量积空间的结构,原有的算符也可以作用于张量积空间。
\begin{definition}[张量积空间中的内积]
    若 $\ms E_1,\ms E_2$ 都是内积空间,有 $\ket{\vphi(1)},\ket{\vphi'(1)}\in\ms E_1;\ket{\chi(2)},\ket{\chi'(2)}\in\ms E_2$,则 $\ket{\vphi(1)\chi(2)}$ 和 $\ket{\vphi'(1)\chi'(2)}$ 的内积为
    \begin{equation}
        \braket{\vphi'(1)\chi'(2)|\vphi(1)\chi(2)} = \braket{\vphi'(1)|\vphi(1)}\braket{\chi'(2)|\chi(2)}
    \end{equation}
\end{definition}

\begin{corollary}[张量积空间的正交归一基]
    若 $\{\ket{u_i(1)}\},\{\ket{v_l(2)}\}$ 都是正交归一基,则 $\{\ket{u_i(1)v_l(2)}\}$ 也是正交归一基。
\end{corollary}

\begin{definition}[算符的张量积]
    假设算符 $A(1)\in\mc L(\ms E_1),B(2)\in\mc L(\ms E_2)$,则算符的张量积 $A(1)\ox B(2)\in \mc L(\ms E_1\ox \ms E_2)$ 对任意 $\ket{\vphi(1)}\in\ms E_1,\ket{\chi(2)}\in\ms E_2$ 的作用形式为
    \begin{equation}
        [A(1)\ox B(2)][\ket{\vphi(1)}\ox\ket{\chi(2)}] = [A(1)\ket{\vphi(1)}]\ox[B(2)\ket{\chi(2)}]
    \end{equation}
\end{definition}
\begin{definition}[延伸算符]
    假设算符 $A(1)\in\mc L(\ms E_1),B(2)\in\mc L(\ms E_2)$,还有两个空间内的恒等算符 $I(1)\in\mc L(\ms E_1),I(2)\in\mc L(\ms E_2)$ 则可以将 $A(1),B(2)$ 延伸自张量积空间,得到延伸算符 
    \begin{equation}
        \td A(1) = A(1)\ox I(2),\quad \td B(2) = I(1)\ox B(2)
    \end{equation}
    于是容易验证
    \begin{equation}
        A(1)\ox B(2) = \td A(1)\td B(2)
    \end{equation}
    在不致混淆的情况下,常省略延伸算符的标记,用原记号 $A(1),B(2)$ 代表其在张量积空间上的延伸算符。
\end{definition}

关于两个空间张量积的各种定义和结论都很容易推广到更多空间的张量积上,对于延伸算符而言,则需要将一个空间的算符和其他各空间的恒等算符用张量积连接起来。我们能证明从不同的空间延伸而来的算符在张量积空间内有以下重要性质。
\begin{theorem}[延伸算符的对易性]\label{thm:extnd_oprtr_cmmtble}
   延续以上 $\td A(1),\td B(2)\in\mc L(\ms E)$ 的定义,有 $[\td A(1),\td B(2)]=0$
\end{theorem}

\subsubsection{张量积空间中的本征值问题}

在已知 $A(1)\in\mc L(\ms E_1)$ 的本征值问题的解,欲求 $A(1)$(实际是 $\td A(1)$)在 $\ms E = \ms E_1\ox \ms E_2$ 上的本征值问题的解,其结果表述如下:

\begin{lemma}[延伸算符的本征值问题]
   对于一个离散谱的算符 $A(1)\in\mc L(\ms E_1)$,假设它属于本征值 $a_n$ 的本征矢为 $\ket{\vphi_n^i(1)}$,则对于任何 $\ket{\chi(2)}\in\ms E_2$,矢量 $\ket{\vphi_n^i(1)\chi(2)}$ 都是 $A(1)$ 在 $\ms E$ 中属于本征值 $a_n$ 的本征矢 
\end{lemma}

对于观察算符,就可以用本征矢构造出张量积空间的基

\begin{theorem}[延伸观察算符的本征系统]
   进一步地,如果 $A(1)$ 是 $\ms E_1$ 中的观察算符,$\ms E_2$ 有正交归一基 $\{\ket{v_l(2)}\}$,则可以找到 $\ket{\vphi_n^i(1)}$ 构成 $\ms E_1$ 的正交归一基,且 $\ms E$ 的一个正交归一基中的矢量可以表示为
   \begin{equation}
       \ket{\psi_n^{i,l}} = \ket{\vphi_n^i(1)v_l(2)}
   \end{equation}
\end{theorem}

容易验证 $A(1)$ 在延伸后依然保持厄米性,于是可以得到推论:
\begin{corollary}[延伸后保持观察算符的性质]
    如果 $A(1)$ 在 $\ms E_1$ 中是观察算符,则它也是 $\ms E$ 中的观察算符。
\end{corollary}

由于延伸后仍然能得到属于某个原本征值的本征矢,所以有推论:
\begin{corollary}[延伸后保持原有的谱]
    A(1) 在 $\ms E$ 和 $\ms E_1$ 中的谱相同。
\end{corollary}

考虑到 $\ms E_2$ 空间的基在延伸后本征空间中张开的新维数,有推论:
\begin{corollary}[延伸后本征值简并度的扩张]\label{col:extnd_oprtr_dgnrcy}
    假设 $\ms E_1$ 空间中本征值 $a_n$ 的简并度是 $g_n$,又有 $\dim \ms E_2 = N_2$,则在 $\ms E$ 空间中的简并度是 $g_n N_2$
\end{corollary}

在张量积空间中,一种常见的本征值问题是形如 $C = A(1)+B(2)$ 的算符,其中 $A(1)\in\mc L(\ms E_1),B(2)\in\mc L(\ms E_2)$。对于非简并的情况,可以假设已知本征值和本征矢如
\begin{equation}
    A(1)\ket{\vphi_n(1)} = a_n\ket{\vphi_n(1)},\quad B(2)\ket{\chi_p(2)} = b_p\ket{\chi_p(2)}
\end{equation}
则可以推出张量积空间的行为
\begin{equation}
    A(1)\ket{\vphi_n(1)\chi_p(2)} = a_n\ket{\vphi_n(1)\chi_p(2)},\quad B(2)\ket{\vphi_n(1)\chi_p(2)} = b_p\ket{\vphi_n(1)\chi_p(2)}
\end{equation}
于是得到 $C$ 的本征值和本征矢
\begin{equation}
    C\ket{\vphi_n(1)\chi_p(2)} = (a_n+b_p)\ket{\vphi_n(1)\chi_p(2)}
\end{equation}
因此可以得到
\begin{theorem}[不同空间延伸算符之和的本征系统]\label{thm:extnt_oprtr_sm_egnsystm}
   上述 $C$ 的本征值是 $A(1),B(2)$ 各自某个本征值之和。假设 $A(1),B(2)$ 是观测算符,此时 $C$ 也是观测算符,则可以得到 $C$ 的本征矢构成的基,使得其中的矢量都是 $A(1)$ 和 $B(2)$ 本征矢的张量积。
\end{theorem}

最后考虑张量积空间中 C.S.C.O. 的问题,考虑已经在 $\ms E_1$ 中形成 C.S.C.O. 的算符 $A(1)$ 和 $\ms E_2$ 中的 C.S.C.O. $B(2),C(2)$。这意味着 $A(1)$ 的本征值 $a_n$ 在 $\ms E_1$ 中都对应唯一的 $\ket{\vphi_n(1)}$;然而在 $\ms E_2$ 中需要 $B(2),C(2)$ 各自的本征值 $b_p,c_r$ 来共同确定本征矢 $\ket{\chi_{pr}(2)}$。

假设 $\dim E_1 = N_1, \dim E_2=N_2$,则根据推论 \ref{col:extnd_oprtr_dgnrcy},在 $\ms E=\ms E_1\ox\ms E_2$ 中,本征值 $a_n$ 的简并度变为 $N_2$,因此 $A(1)$ 自身不足以构成 C.S.C.O.;同理,属于 $b_p$ 和 $c_r$ 的 $B(2),C(2)$ 共同本征矢也张成了 $N_1$ 维本征空间,因此 $B(2),C(2)$ 也不足以构成 C.S.C.O.。但是根据定理 \ref{thm:extnd_oprtr_cmmtble},$A(1),B(2),C(2)$ 两两对易,且 $\ket{\vphi_n(1)\chi_{pr}(2)}$ 是唯一对应于 $a_n,b_p,c_r$ 的共同本征矢。由于 $\{\ket{\vphi_n(1)}\},\{\ket{\chi_{pr}(2)}\}$ 分别是 $\ms E_1,\ms E_2$ 的基,所以上述张量积矢量也构成 $\ms E$ 的基。此时 $A(1),B(2),C(2)$ 即是 $\ms E$ 中的 C.S.C.O.。根据以上论证过程,能推广到以下定理:
\begin{theorem}[张量积空间中的 C.S.C.O.]
   若一些空间中各有自己的 C.S.C.O.,则合并这些 C.S.C.O. 就得到这些空间的张量积空间中的一个 C.S.C.O.
\end{theorem}

\subsubsection{二维问题:平面方势阱模型}
张量积空间使得透彻分析一维态空间 $\ms E_x$ 和三维态空间 $\ms E_{\bm r}$ 之间的关系成为了可能。

$\ms E_x$ 都是一维空间中一个粒子的态空间,观察算符 $X$ 是其中的 C.S.C.O.,其本征矢 $\ket{x_0}$ 构成了一组基,任意右矢 $\ket{\psi}$ 在基 $\ket{x}$ 上的展开即是位置表象下的波函数 $\psi(x)$。

类似考虑 $\ms E_y,\ms E_z$ 中的情况,并作张量积 $\ms E_{xyz} = \ms E_x\ox\ms E_y\ox\ms E_z$,它的一个基形如 $\{\ket{x,y,z}\}$,其中 $\ket{x,y,z} = \ket x\ket y\ket z$。这些基矢都是 $X,Y,Z$ 延伸后的本征矢,自然满足
\begin{align}
    X\ket{x,y,z} = x\ket{x,y,z},\ Y\ket{x,y,z} = y\ket{x,y,z},\ Z\ket{x,y,z} = z\ket{x,y,z}
\end{align}
于是记 $\ket{x,y,z} = \ket{\bm r}$,其中 $\bm r = (x,y,z)$,则与之前推导的性质完全一致:$\ket{\bm r}$ 是 $\ms E_{\bm r}$ 的基,且 $X,Y,Z$ 构成一组 C.S.C.O.。再令 $\bm R = X\ox Y\ox Z$,就可以说 $\bm R$ 是 $\ms E_{\bm r}$ 的 C.S.C.O.。

假设 $\ket{\vphi}\in\ms E_x,\ket{\chi}\in\ms E_y,\ket{\omega}\in\ms E_z$,则 $\ket{\vphi\chi\omega}\in\ms E_{\bm r}$ 在位置表象下的分量为
\begin{equation}
    \braket{\bm r|\vphi\chi\omega} = \braket{x|\vphi}\braket{y|\chi}\braket{z|\omega}
\end{equation}
这就意味着以单个张量积矢量形式呈现的 $\ket{\vphi\chi\omega}$ 的位置波函数具有分离变量的形式 $\vphi(x)\chi(y)\omega(z)$。然而我们早已在 \ref{col:tnsr_prdct_vctr_expnsn} 中指出这种分离变量并不是普遍的。

接下来把定理 \ref{thm:extnt_oprtr_sm_egnsystm} 的结论应用到此情况中:假设有一个算符 $H=H_x+H_y+H_z$,三项分别为三个一维态空间中的观察算符的延伸,它们各自在一维态空间中有这样的本征系统(假设是非简并的)
\begin{equation}
    H_x\ket{\vphi_n} = E_x^n\ket{\vphi_n},\ H_y\ket{\chi_p} = E_y^p\ket{\chi_p},\ H_z\ket{\omega_r} = E_z^r\ket{\omega_r}
\end{equation}
于是 $H$ 的本征值形如 $E^{n,p,r} = E_x^n+E_y^p+E_z^r$,对应的本征矢就是 $\ket{\vphi_n\chi_p\omega_r}$,在位置表象下具有分离变量的波函数 $\vphi_n(x)\chi_p(y)\omega_r(z)$。

以上从一维态空间经张量积构造出多维态空间的方法是没有限制的。在一个具体的模型中,可以从最简单的二维态空间开始讨论。例如被限制在某个平面上的粒子的态空间形如 $\ms E_x\ox\ms E_y$,假设平面上有一个无限深方势阱:
\begin{equation}
    V(x,y) = \begin{cases}
        0 & 0\le x \le a, 0\le y\le a\\
        \infty & \text{其他情况}
    \end{cases}
\end{equation}
则这个粒子的哈密顿量可以写作
\begin{equation}
    H = \frac 1 {2m}\left(H_x^2+H_y^2\right)+V_\infty(x)+V_\infty(y)
\end{equation}
其中
\begin{equation}
    V_\infty(u)=\begin{cases}
        0 & 0\le u\le a\\
        \infty & u< 0\ \text{或}\ u>a
    \end{cases}
\end{equation}
因此可以分拆为 $H = H_x+H_y$,其中
\begin{equation}
    H_u = \frac{1}{2m}P_u^2+V_\infty(u)
\end{equation}

根据此前的论述,我们可以找到 $H$ 形如 $\ket{\psi} = \ket{\vphi_x\vphi_y}$ 的本征态,其中 $H_u\ket{\vphi_u} = E_u\ket{\vphi_u}$,于是就有 $H\ket{\psi} = E\ket{\psi}$,其中 $E = E_x+E_y$。

一维上的问题已经解过,由 \ref{eq:1d_infnt_ptntl_wll_egnvl} 可知 $H$ 的本征值形如
\begin{equation}
    E_{n_x,n_y} = \frac{1}{2ma^2}(n_x^2+n_y^2)\pi^2\hbar^2,\quad n_x,n_y\in\N_+
\end{equation}
与之对应的本征矢是 $\ket{\psi_{n_x,n_y}} = \ket{\vphi_{n_x}}_x\ket{\vphi_{n_y}}_y$,根据 \ref{eq:1d_infnt_ptntl_wll_wv_fnctn},其位置表象波函数为
\begin{equation}
    \psi_{n_x,n_y}(x,y) = \vphi_{n_x}(x)\vphi_{n_y}(y) = 
    \begin{cases}
        \dfrac 2 a\sin\dfrac{n_x\pi x}a\sin\dfrac{n_y\pi y}a & 0\le x\le a,0\le y\le a\\
        0 & \text{其他情况}
    \end{cases}
\end{equation}

与一维问题类似,粒子在二维方势阱中也“不得安宁”,其基态 $n_x=1,n_y=1$ 也有零点能
    $E_{1,1} = \dfrac{\pi^2\hbar^2}{ma^2}$。
对于众激发态而言,一般对应于 $n_x\ne n_y$ 的能级都至少是二重简并的,因为能量表达式中 $n_x,n_y$ 处于对称的位置,这使得 $E_{n_x,n_y} = E_{n_y,n_x}$,而 $\ket{\psi_{n_x,n_y}}$ 和 $\ket{\psi_{n_y,n_x}}$ 显然是不同的态。这种简并性来源于体系的对称性,因为方势阱关于 $x,y$ 坐标是对称的,在哈密顿量中交换 $X,Y$ 以及 $P_x,P_y$ 都不影响其形式,因而位置表象下对应于 $\psi(x,y)$ 和 $\psi(y,x)$ 的态的相关本征值一定简并。

这种起源于问题本身的简并性称为\textbf{系统的简并},可以通过引入不对称而消除:考虑一个矩形势阱,它在 $Ox,Oy$ 方向的宽度不相等,势函数为
\begin{equation}
    V(x,y) = \begin{cases}
        0 & 0\le x \le a, 0\le y\le b\\
        \infty & \text{其他情况}
    \end{cases}
\end{equation}
则可以类似地解出能量本征值
\begin{equation}
    E_{n_x,n_y} = \frac{\pi^2\hbar^2}{2m}\left(\frac{n_x^2}{a^2}+\frac{n_y^2}{b^2}\right)
\end{equation}
此时就有 $E_{n_x,n_y}=E_{n_y,n_x}$,我们也同时看到势函数在几何上的对称性、哈密顿量代数上的对称性消失了。

回到对称情况,我们还会注意到 $5^2+5^2=1^2+7^2$,这意味着 $E_{5,5} = E_{1,7} = E_{7,1}$ 至少三重简并的出现,这种高于二重的简并性来源于整数平方和的一些巧合,不使用精妙的数论工具,我们很难预言这种简并性的分布。这种与问题对称性无关的简并性称为\textbf{偶然的简并}。容易想到在非对称情况下偶然简并变得更加难以捉摸,它将与 $a,b$ 的具体比例发生关联,且有可能永远不会发生。

三维势阱也有类似的结论,对于一个长宽高分别为 $a,b,c$ 的长方体势阱,容易求出能量本征值为
\begin{equation}
    E_{n_x,n_y,n_z} = \frac{\pi^2\hbar^2}{2m}\left(\frac{n_x^2}{a^2}+\frac{n_y^2}{b^2}+\frac{n_z^2}{c^2}\right)
\end{equation}
在 $a=b=c$ 的情况下,系统的简并体现在 $n_x,n_y,n_z$ 任意对换顺序得到的能量本征值都相同;而偶然简并将更加复杂,例如 $3^2+3^2+3^2 = 5^2+1^2+1^2, 4^2+4^2+1^2 = 5^2+2^2+2^2$ 等。

\subsubsection{二体问题:重审氢原子结构}
考虑两个简单粒子 (1),(2) 组成的物理体系:单独刻画两个粒子时,首先在粒子 (1) 的态空间 $\ms E_{\bm r_1}$ 中确立位置表象 $\{\ket{\bm r_1}\}$,取观察算符 $\bm R_1$;同样在粒子 (2) 的态空间 $\ms E_{\bm r_2}$ 中确立位置表象 $\{\ket{\bm r_2}\}$,取观察算符 $\bm R_2$。

而合并考虑两个粒子时,作张量积 $\ms E_{\bm r_1\bm r_2} = \ms E_{\bm r_1}\ox\ms E_{\bm r_2}$,其基为 $\ket{\bm r_1\bm r_2} = \ket{\bm r_1}\ket{\bm r_2}$,于是其中的右矢 $\ket{\psi}$ 在位置表象下的波函数为 $\psi(\bm r_1,\bm r_2)=\braket{\bm r_1\bm r_2|\psi}$。这种考虑的概率诠释即是:在 $\bm r_1$ 周围的体积元 $\d\bm r_1$ 和 $\bm r_2$ 周围的体积元 $\d\bm r_1$ 中分别找到粒子 (1),(2) 的概率为
\begin{equation}
    \d\P(\bm r_1,\bm r_2) = |\psi(\bm r_1,\bm r_2)|^2\d\bm r_1\d\bm r_2
\end{equation}
其中 $\psi(\bm r_1,\bm r_2)$ 已经过归一化。在这种诠释下,双粒子态对应的波函数具有了物理意义。

\begin{postulate}[多粒子体系的态空间]
多粒子体系的态空间是每个粒子的态空间的张量积,将原有的 $\bm R_i$ 合并起来就得到多粒子态空间的一套 C.S.C.O.。
\end{postulate}

此前讨论氢原子的电子结构时,由于质子的质量远大于电子,我们忽略了质子的运动,将电子置于一个质子产生的、固定的中心力场中,从而将问题简化为电子的单粒子问题。然而,为了更精确地计算,我们必须直面二体系统的实质。

此前的处理只不过是将问题迁移到了经典力学所谓的质心系,这种操作建立于以下论证:

经典力学中,如果两个位矢为 $\bm r_1,\bm r_2$,质量为 $m_1,m_2$ 的粒子之间的保守力总是沿连线方向,且不受外力,则其总势能函数可以写作 $V(\bm r_1-\bm r_2)$,其总机械能,也即哈密顿量可以写作
\begin{equation}
    \mc H = \frac 1 2 m_1\dot{\bm r}_1^2+\frac 1 2 m_1\dot{\bm r}_2^2+V(\bm r_1-\bm r_2)
\end{equation}
自带的正则动量为熟知的 $\bm p_i = m_i\bm r_i,\ i=1,2$。

引入\textbf{质心坐标} $\bm r_G = \dfrac{m_1\bm r_1+m_2\bm r_2}{m_1+m_2}$ 和相对坐标 $\bm r=\bm r_1-\bm r_2$,以及体系总质量 $M = m_1+m_2$ 和\textbf{约化质量} $\mu = \dfrac{m_1m_2}{m_1+m_2}$,可以写出哈密顿量
\begin{equation}
    \mc H = \frac 1 2 M\dot{\bm r}_G^2+\frac 1 2\mu\dot{\bm r}^2+V(\bm r)
\end{equation}

定义体系总动量 $\bm p_G = M\dot{\bm r}_G = \bm p_1+\bm p_2$ 和两粒子相对动量 $\bm p = \mu\dot{\bm r} = \dfrac{m_2\bm p_1-m_1\bm p_2}{m_1+m_2}$,进而化简得到:
\begin{equation}
    \mc H = \frac{\bm p_G^2}{2M}+\frac{\bm p^2}{2\mu}+V(\bm r)
\end{equation}

接下来将以上经典力学的讨论结果全部量子化:用类似的方式定义质心坐标和相对坐标的观察算符
\begin{equation}
    \bm R_G = \frac{m_1\bm R_1+m_2\bm R_2}{m_1+m_2},\quad \bm R = \bm R_1-\bm R_2
\end{equation}
以及体系总动量和相对动量的观察算符
\begin{equation}
    \bm P_G = P_1+P_2,\quad \bm P =\frac{m_2\bm P_1-m_1\bm P_2}{m_1+m_2}
\end{equation}
根据 $\bm R_i,\bm P_j$ 之间的对易关系,和来自不同态空间的延伸算符对易的结论 \ref{thm:extnd_oprtr_cmmtble},容易得到以下结论
\begin{equation}
    [X_G, P_{Gx}] = [X,P_x] = \i\hbar
\end{equation}
对 $y,z$ 轴分量亦是如此。因此 $\bm R,\bm P$ 和 $\bm R_G,\bm P_G$ 都分别满足一般位置和动量的正则对易关系式 \ref{thm:commuterp},且能验证这两组算符在组间互相对易。因此可以将其解释为两个不同假想粒子——固定质心和一个转子的位置和动量算符。于是我们可以用算符 $\bm R_G,\bm R$ 的本征矢构建表象 $\{\ket{\bm r_G,\bm r}\}$,将体系的态空间也看做与 $\bm R_G$ 联系的 $\ms E_{\bm r_G}$ 和与 $\bm R$ 联系的 $\ms E_{\bm r}$ 的张量积。

经过相同的运算过程,由体系原来的哈密顿算符
\begin{equation}
    H = \frac{\bm P_1^2}{2m_1}+\frac{\bm P_2^2}{2m_2}+V(\bm R_1-\bm R_2)
\end{equation}
重构为
\begin{equation}
    H = \frac{\bm P_G^2}{2M} +\frac{\bm P^2}{2\mu} + V(\bm R)
\end{equation}
由此可以看出,这样做变换的好处在于,假想粒子的两项:质心项和转子项发生了分离,即
\begin{equation}
    H = H_G+H_r,\quad H_G  =\frac{\bm P_G^2}{2M},\quad H_r = \frac{\bm P^2}{2\mu} + V(\bm R)
\end{equation}

此时就会发现 $H_G$ 在 $\ms E_{\bm r_G}$ 中起作用,而 $H_r$ 在 $\ms E_{\bm r}$ 中起作用。根据此前讨论过的分项问题,可以解出 $\ket{\psi} = \ket{\vphi_G\chi_r}$,其中 $\ket{\vphi_G}\in\mc\ms E_{\bm r_G}, \ket{\chi_r}\in\ms E_{\bm r}$,使得
\begin{equation}
    H_G\ket{\vphi_G} = E_G\ket{\vphi_G},\quad H_r\ket{\chi_r} = E_r\ket{\chi_r}
\end{equation}
并有 $H\ket{\psi} = E\ket{\psi}$,其中 $E = E_G+E_r$。

在表象 $\ket{\bm r_G}$ 下写出 $\ms E_{\bm r_G}$ 中的方程式
\begin{equation}
    -\frac{\hbar^2}{2m}\nabla^2\vphi_G(\bm r_G) = E_G\vphi_G(\bm r_G)
\end{equation}
这是一个自由粒子的方程,它的解是平面波
\begin{equation}
    \vphi_G(\bm r_G) = \frac 1{(2\pi\hbar)^{3/2}}\e^{\frac \i\hbar\bm p_G\cdot\bm r_G}
\end{equation}
对应于连续的能量本征值 $E_G = \dfrac{\bm p_G^2}{2M}$,可以取任意非负值,它是整个体系的\textbf{平动能}。

而我们此前关心的氢原子定态问题实际上是在表象 $\ket{\bm r}$ 下写出 $\ms E_{\bm r}$ 中的方程式,它描述了体系在质心坐标系下的行为:
\begin{equation}
    \left[-\frac{\hbar^2}{2\mu}\nabla^2+V(\bm r)\right]\chi_r(\bm r) = E_r\chi_r(\bm r)
\end{equation}
由此可见,我们在氢原子的讨论中使用的 $\mu$ 并非电子质量,而是电子与原子核的约化质量,除此之外,对方程的形式没有任何影响。将所得全部结果中的 $\mu$ 代入约化质量,就得到了二体问题修正之后的结果。

\subsection{角动量的耦合}

研究角动量耦合的动机首先来自于经典力学。在无外力矩时,多粒子体系中各粒子的角动量可因内力而互相传递,但各粒子角动量的矢量和,也即总角动量一定守恒。这一思考带来的启示是:研究总角动量能够规避粒子间相互作用使角动量的原有性质发生改变的问题。

\subsubsection{总角动量算符的确立}

在量子力学中,如果两个角动量 $\bm J_1,\bm J_2$ 分属不同的态空间 $\ms E_1,\ms E_2$,每个态空间中默认的 C.S.C.O. 的角动量相关部分都包含 $\bm J_i^2$ 和 $J_{iz}$,其角动量标准基为 $\{\ket{k_i,j_i,m_i}\}$,满足
\begin{equation}
    \bm J_i^2\ket{k_i,j_i,m_i} = j_i(j_i+1)\hbar^2\ket{k_i,j_i,m_i},\quad J_{iz}\ket{k_i,j_i,m_i} = m_i\hbar\ket{k_i,j_i,m_i}
\end{equation}
$k_i$ 是对应于其他无关算符的指标,以上 $i=1,2$,则 $\ms E_1\ox\ms E_2$ 中的 C.S.C.O. 可以取原态空间的 C.S.C.O. 的并。如果只关注角动量部分,此类 C.S.C.O. 将包含分角动量算符 $\bm J_1^2,J_{1z},\bm J_2^2,J_{2z}$,其共同本征矢构成了分角动量表象,基矢形如
\begin{equation}
    \ket{k_1,k_2;j_1,j_2;m_1,m_2} = \ket{k_1,j_1,m_1}\ox\ket{k_2,j_2,m_2}
\end{equation}

如果定义 $\bm J = \bm J_1+\bm J_2$(分量按此式加和),从 $\bm J_1,\bm J_2$ 携带的角动量对易关系,容易推知 $\bm J$ 的对易关系,从而顺理成章地称为总角动量算符。考虑到 $\bm J_1,\bm J_2$ 身在不同态空间带来的对易性,这一角动量的平方算符 $\bm J^2 = \bm J_1^2+\bm J_2^2+2\bm J_1\cdot\bm J_2$,其中未被考虑的标量积
\begin{equation}
    \bm J_1\cdot\bm J_2=J_{1x}J_{2x}+J_{1y}J_{2y}+J_{1z}J_{2z} = \frac 1 2(J_{1+}J_{2-}+J_{1-}J_{2+})+J_{1z}J_{2z}
\end{equation}

总、分角动量算符之间的对易关系是建立新的表象的基础:
\begin{theorem}[总角动量算符与单粒子角动量算符之间的对易关系]
   四个算符 $\bm J_1^2,\bm J_2^2,\bm J^2,J_z$ 互相对易,而 $\bm J^2$ 不能与 $J_{1z},J_{2z}$ 对易。
\end{theorem}
\begin{proof}
    因为来自不同的态空间 $\bm J_1^2,\bm J_2^2$ 之间的对易性显然,作为角动量的一般结论 $\bm J^2$ 和 $J_z$ 的对易性亦显然。而 $\bm J_i$ 和 $\bm J_j^2$ 无论是否 $i\ne j$,都会因为角动量的固有性质和态空间的不同而对易。$\bm J=\bm J_1+\bm J_2$ 在诸分量上的表现又使得 $\bm J$ 的诸分量与 $\bm J_1^2$ 和 $\bm J_2^2$ 对易,特别表现为 $[J_z,\bm J_i^2]=0,\ i=1,2$。而考虑到 $\bm J^2 = (\bm J_1+\bm J_2)^2$ 中每一项都体现出与 $\bm J_i^2,\ i=1,2$ 的对易性,所以也有 $[\bm J^2,\bm J_i^2]=0,\ i=1,2$。
    
    但是,经过计算
    \begin{equation}
        [\bm J^2,J_{1z}] = 2[\bm J_1\cdot\bm J_2,J_{1z}] = 2[J_{1x}J_{2x}+J_{1y}J_{2y},J_{1z}] = 2\i\hbar(-J_{1y}J_{2x}+J_{1x}J_{2y})
    \end{equation}
    另有 $[\bm J^2,J_{2z}] = 2\i\hbar(J_{1y}J_{2x}-J_{1x}J_{2y})$,因此 $\bm J^2$ 和 $J_{iz},\ i=1,2$ 不对易,但反号关系使得 $[\bm J^2,J_z] = [\bm J^2,J_{1z}+J_{2z}] = 0$ 成立。
\end{proof}

此等对易关系的直接后果是:原本的分角动量表象(基由 $J_{iz}$ 的本征矢构成)与 $\bm J^2$ 的本征矢不兼容;换言之,$\bm J_1^2,\bm J_2^2,\bm J^2,J_z$ 存在共同的本征矢,且这些本征矢不可能同时是 $J_{1z},J_{2z}$ 的本征矢,甚至不可能同时是任何其他与 $\bm J_1,\bm J_2,\bm J$ 有关的算符的本征矢。于是这四个算符(与其他可能存在的无关算符一起)构筑了一个新的表象:总角动量表象。

而研究总角动量表象下的本征值问题,以及总、分角动量表象下本征矢之间的变换关系,就是所谓角动量耦合问题。根据此前对角动量表象下态空间分解的讨论,假设 $\ms E_i(k_i,j_i)$ 是 $\ms E_i$ 中固定 $k,j$ 值的标准基矢量张成的 $2j+1$ 维子空间,则
\begin{equation}
    \ms E_i = \bigoplus_{k_i,j_i}\ms E_i(k_i,j_i)
\end{equation}
我们知道这种分解方式的优点是子空间在 $\bm J_i$ 下的不变性,因此定义
\begin{equation}
    \ms E(k_1,k_2;j_1,j_2) = \ms E_1(k_1,j_1)\ox\ms E_2(k_2,j_2)
\end{equation}
则容易证明
\begin{equation}
    \ms E = \bigoplus_{k_1,k_2,j_1,j_2}\ms E(k_1,k_2;j_1,j_2)
\end{equation}
这是一个 $(2j_1+1)(2j_2+1)$ 维空间,仍然在 $\bm J_1,\bm J_2$ 及其衍生物的作用下不变,因此在 $\bm J$ 及其衍生物的作用下仍不变。对 $\bm J^2$ 和 $J_z$ 的本征值问题的研究只需在每个这样的子空间中独立进行,对于整体的态空间而言,就可以做到分块对角化。

另外 $\bm J_i$ 的作用只与 $j_i,m_i$ 相关,所以各子空间内的对角化问题被 $j_1,j_2$ 完全确定,可以略去指标 $k_i$,用 $\ms E(j_1,j_2)$ 代表 $\ms E(k_1,k_2;j_1,j_2)$,用 $\ket{j_1,j_2;m_1,m_2}$ 代表 $\ket{k_1,k_2;j_1,j_2;m_1,m_2}$。

又因为 $\ms E(j_1,j_2)$ 在总角动量算符 $\bm J$ 及其衍生物 $\bm J^2,J_z,J_+,J_-$ 下的不变性,假设 $\bm J^2$ 有本征值 $J(J+1)\hbar^2$,则同样可以依照这个本征值进行直和分解,就像初次讨论角动量问题时,对整个态空间依照 $k,j$ 进行直和分解一样:
\begin{equation}
    \ms E(j_1,j_2) = \bigoplus_{k,J}\ms E(k,J)
\end{equation}
由于 $\ms E(j_1,j_2)$ 中省略的 $k_1,k_2$ 是被固定的,这里的 $k$ 的来源并非 $k_1,k_2$,而是暂时不能排除可能出现一些子空间的 $J$ 相同而其对 $\bm J$ 的不变性各自独立的情况,因此用 $k$ 对此进行标记。因此此时需要关注的问题首先来源于 $k,J$:在 $j_1,j_2$ 给定时 $J$ 的取值如何?每个 $J$ 对应的 $k$ 有多少个?而另一方面还是要回到旧表象和新表象的变换上,即 $\bm J^2,J_z$ 的共同本征矢怎样在分角动量表象下展开?

\subsubsection{总角动量算符的本征值}
选定一个子空间 $\ms E(j_1,j_2)$,不妨设 $j_1\le j_2$,其中的 $\ket{j_1,j_2;m_1,m_2}$ 是 $J_{iz}$ 的本征矢,因此也是 $J_z$ 的本征矢:
\begin{equation}
    J_z\ket{j_1,j_2;m_1,m_2} = (J_{1z}+J_{2z})\ket{j_1,j_2;m_1,m_2} = (m_1+m_2)\hbar\ket{j_1,j_2;m_1,m_2}
\end{equation}


\begin{enumerate}
    \item \textbf{取值范围}:令 $M = m_1+m_2$,则 $J_z$ 的本征值形如 $M\hbar$。从该定义可见,$M$ 的取值范围是两个取值范围为 $j,j-1,\cdots,-j$ 的数之和的取值范围,也即
 $j_1+j_2,j_1+j_2-1,\cdots,-(j_1+j_2)$。
    \item \textbf{简并度}:在子空间内记每个 $M$ 对应的 $J_z$ 的本征值的简并度为 $g_{j_1,j_2}(M)$,该简并度的来源是不同的 $m_1,m_2$ 的值对应于相同的 $m_1+m_2$。
    \begin{enumerate}
        \item 最简单的情况:$M=j_1+j_2$ 是非简并的:只有 $m_1=j_1,m_2=j_2$ 一种情况,即 $g_{j_1,j_2}(j_1+j_2)=1$。
        \item 更复杂的情况:当 $M=j_1+j_2-1$ 时,有 $m_1=j_1,m_2=j_2-1;m_1=j_1-1,m_2=j_2$ 两种情况。
        \item 以此类推:$g_{j_1,j_2}(j_1+j_2-1)=2$。$M$ 每减小 1,都使简并度增加 1,直至 $M=j_1-j_2$ 时有 $m_1=j_1,m_2 = -j_2,\cdots,m_1=j_1-2j_2,m_2=j_2$ 共 $2j_1+1$ 种情况,即 $g_{j_1,j_2}(j_1-j_2)=2j_1+1$。
        \item 停止增长:若 $M=j_1-j_2-1$,则只有 $m_1=j_1-1,m_2 = -j_2,\cdots,m_1=j_1-2j_2-1,m_2=j_2$ 共 $2j_1+1$ 种情况,因为已经取不到 $m_1=j_1,m_2=-j_2-1$ 的值,这种情况一直延续到 $M=-(j_1-j_2)$。即 $|M|\le j_1-j_2$ 时,都有 $g_{j_1,j_2}(M)=2j_1+1$。
        \item 开始减少:而如果 $M=-(j_1-j_2)-1$,就无法维持 $m_1=-(j_1-j_2)-1-j_2 = -j_1-1,m_2=j_2$ 的情况,因此 $g_{j_1,j_2}(-(j_1-j_2)-1)=2j_1$,之后 $M$ 每减小 1,都是简并度减小 1,直至回到 $M=-(j_1+j_2)$ 时 $g_{j_1,j_2}(-(j_1+j_2))=1$。
    \end{enumerate}
\end{enumerate}

 从整个过程还能看出简并度的对称变化趋势:$g_{j_1,j_2}(-M) = g_{j_1,j_2}(M)$。总结起来就是以下结论:
\begin{theorem}[$J_z$ 的本征值及其简并度]
   在子空间 $\ms E(j_1,j_2)\ (j_1\ge j_2)$ 中,$J_z$ 的本征值 $M\hbar$ 中 $M$ 的取值范围为
   \begin{equation*}
       j_1+j_2,j_1+j_2-1,\cdots,-(j_1+j_2)
   \end{equation*}
   这也意味着如果 $j_1,j_2$ 都是整数或半整数,则 $M$ 都是整数;若一个是整数,另一个是半整数,则 $M$ 都是半整数。
   其简并度为
   \begin{equation}
       g_{j_1,j_2}(M) = \begin{cases}
           j_1+j_2-|M|+1 & j_1-j_2\le |M|\le j_1+j_2\\
           2j_1+1 & 0 \le |M| < j_1-j_2
       \end{cases}
   \end{equation}
\end{theorem}

在分角动量表象下给定了 $(k_1,k_2,)j_1,j_2$ 之后,对于角动量算符 $\bm J^2,J_z$ 的共同本征矢而言,如果进一步确定了代表 $\bm J^2$ 的指标 $J$ ,则代表 $J_z$ 的指标 $M$ 的可能取值就已经被确定,相应地记作 $\ket{k,J,M=J},\cdots,\ket{k,J,M=-J}$,这里的 $k$ 与 $k_1,k_2$ 无关而只是用来标记可能出现的额外简并度,$k,J$ 相同的这些态矢共同张成子空间 $\ms E(k,J)$。因此如果依照初次讨论角动量的思路,通常先考察 $J$,再考察 $M$ 的 $2J+1$ 个取值。

然而此处首先通过简单的加法关系确定了 $M$ 的范围及其简并度:为了从此推出 $J$ 的特征需要作以下分析。
\begin{enumerate}
    \item 所有 $J>j_1+j_2$ 都不存在,而 $J=j_1+j_2$ 的取值因 $M=j_1+j_2$ 的存在而一定存在,它是 $J$ 的最大值,也因为 $g_{j_1,j_2}(j_1+j_2)=1$ 而一定非简并。
\end{enumerate}

因此 $\ms E(k,J=j_1+j_2)$ 中的 $k$ 只有唯一的取值,可以丢弃,子空间记为 $\ms E(J=j_1+j_2)$,其中的本征矢应该记作 $\ket{J=j_1+j_2,M=J},\cdots,\ket{J = j_1+j_2,M=-J}$。

然后考虑 $M = j_1+j_2-1$,假设这个值可以取到且 $j_1+j_2-1\ge j_1-j_2$,则 $g_{j_1,j_2}(j_1+j_2-1)=2$。此前已经论证了 $\ms E(J=j_1+j_2)$ 的存在唯一性,其中已经包含了一个 $\ket{J=j_1+j_2-1,M=J}$;剩余简并度的存在只能归结于 $J=j_1+j_2-1$ 的存在,而恰好剩余一个简并度说明 $\ms E(k,J=j_1+j_2-1)$ 中 $k$ 也只有唯一的取值,可以丢弃,子空间记为 $\ms E(J=j_1+j_2-1)$,其中的本征矢则是 $\ket{J=j_1+j_2-1,M=J},\cdots,\ket{J=j_1+j_2-1,M=-J}$。

于是只要在 $j_1-j_2\le M\le j_1+j_2$ 的范围内,$M$ 每减少 1,$g_{j_1,j_2}(M)=j_1+j_2-M+1$ 就增加 1,若 $M\ne j_1+j_2$,则为这个简并度做贡献的 $g_{j_1,j_2}(M-1)=j_1+j_2-M$ 个本征矢已经分别位列于从 $\ms E(J=j_1+j_2),\cdots,\ms E(J=M+1)$ 等 $j_1+j_2-M$ 个子空间中——这些子空间都已经被证明可以丢弃 $k$,至少从 $\ms E(J=j_1+j_2)$ 开始如此。于是仅剩的一个简并度指向唯一的 $\ms E(J=M)$ 的存在,且 $k$ 也被这种唯一性丢弃掉。

归纳法可以说明,这种关系将产生一系列唯一的 $\ms E(J=j_1+j_2),\cdots,\ms E(J=j_1-j_2)$ 共 $2j_1+1$ 个子空间。而假如存在 $M<j_1-j_2$ 时,则 $g_{j_1,j_2}(M)=g_{j_1,j_2}(M-1)=2j_1+1$,这些简并度已经被上述恰好 $2j_1+1$ 个子空间层层盘剥完毕,未给 $J<j_1-j_2$ 留下位置。

如果解除 $j_1\ge j_2$ 的限制,在 $j_1<j_2$ 时只需在上述推理中交换 $j_1,j_2$ 的位置。因此我们得到了以下结论。

\begin{theorem}[$\bm J^2$ 的本征值及其本征子空间]
   在子空间 $\ms E(j_1,j_2)$ 中,$\bm J^2$ 的本征值 $J(J+1)\hbar^2$ 中 $J$ 的取值范围为
   \begin{equation}
       j_1+j_2,j_1+j_2-1,\cdots,|j_1-j_2|
   \end{equation}
   每个 $J$ 值对应的本征矢张成 $2J+1$ 维本征子空间 $\ms E(J)$,其中有与 $J_z$ 共享的 $2J+1$ 个唯一确定的本征矢 $\ket{J,M=J},\cdots,\ket{J,M=-J}$。
\end{theorem}

无论从 $\bm J^2$ 还是 $J_z$ 的角度看去,子空间 $\ms E(j_1,j_2)$ 的维数都是固定的。对于任意的 $j_1,j_2$,都可以验证以下恒等式:
\begin{equation}
    \sum_{M=-j_1-j_2}^{j_1+j_2}g_{j_1,j_2}(M) = \sum_{J=|j_1-j_2|}^{J=j_1+j_2}2J+1 = (2j_1+1)(2j_2+1)
\end{equation}
从分角动量表象看去,固定无关指标 $k_i$,则 $j_i$ 对应的子空间是 $2j_i+1$ 维,其张量积空间中自然得到 $(2j_1+1)(2j_2+1)$ 的结果。

\subsubsection{总角动量算符的本征矢}

依照上述记号,我们已经得到了 $\ms E(j_1,j_2)$ 的一种直和分解:
\begin{equation}
    \ms E(j_1,j_2) = \bigoplus_{J=|j_1-j_2|}^{j_1+j_2}\ms E(J)
\end{equation}

首先考虑 $J$ 最大的一项,即子空间 $\ms E(J=j_1+j_2)$。由于在 $\ms E(j_1,j_2)$ 中 $\ket{j_1,j_2;m_1=j_1,m_2=j_2}$ 是 $J_z$ 对应于 $M=j_1+j_2$ 的唯一本征矢,它一定处在 $\ms E(J=j_1+j_2)$ 中,即也是 $\bm J^2$ 的本征矢。作为起始,我们就可以适当选择常数因子,使得在满足归一化的同时,还要满足
\begin{equation}
    \ket{J=j_1+j_2,M=j_1+j_2} =\ket{j_1,j_2;m_1=j_1,m_2=j_2} 
\end{equation}
在此基础上使用 $J_-$ 就能得到全体 $\ket{J=j_1+j_2,M}$。例如
\begin{equation}
    J_-\ket{J=j_1+j_2,M=j_1+j_2} = \hbar\sqrt{2(j_1+j_2)}\ket{J=j_1+j_2,M=j_1+j_2-1}
\end{equation}
然后注意到 $J_-=J_{1-}+J_{2-}$,应用于分角动量基矢:
\begin{align}\label{eq:sprtd_anglr_mmntm_dcmps_1}
    &\ket{J=j_1+j_2,M=j_1+j_2-1} \nonumber\\=& \frac 1{\hbar\sqrt{2(j_1+j_2)}}J_-\ket{J=j_1+j_2,M=j_1+j_2}\nonumber\\
    =&  \frac 1{\hbar\sqrt{2(j_1+j_2)}}(J_{1-}+J_{2-})\ket{j_1,j_2;m_1=j_1,m_2=j_2} \nonumber\\
    = & \frac 1{\hbar\sqrt{2(j_1+j_2)}}\left[\hbar\sqrt{2j_1}\ket{j_1,j_2;m_1=j_1-1,m_2=j_2}+\hbar\sqrt{2j_2}\ket{j_1,j_2;m_1=j_1,m_2=j_2-1}\right]\nonumber\\
    = & \sqrt{\frac{j_1}{j_1+j_2}}\ket{j_1,j_2;m_1=j_1-1,m_2=j_2}+\sqrt{\frac{j_2}{j_1+j_2}}\ket{j_1,j_2;m_1=j_1,m_2=j_2-1}
\end{align}
这就将归一化的 $\ket{J=j_1+j_2,M=j_1+j_2-1}$ 表示为了分角动量基矢的线性组合。

重复以上步骤,就能得到 $\{\ket{J=j_1+j_2,M}\}$ 共 $2(j_1+j_2)+1$ 个矢量在分角动量基矢上的表示,也就求出了 $\ms E(j_1,j_2)$ 中第一个直和分量 $\ms E(J=j_1+j_2)$ 的基。

然后考虑 $J$ 次大的项,即 $\ms E(J=j_1+j_2-1)$,其中 $M$ 的极大值为 $j_1+j_2-1$。在 $\ms E(j_1,j_2)$ 中 $g_{j_1,j_2}(j_1+j_2-1)$ 为 2,而这个简并度在除去 $\ms E(J=j_1+j_2)$ 后就只剩下 1,它一定位于 $\ms E(J=j_1+j_2-1)$ 中,即等同于 $\ket{J=j_1+j_2-1,M=j_1+j_2-1}$。根据 $M$ 可能的 $m_1,m_2$ 两项和,它在分角动量表象下的展开式一定是:
\begin{equation}
    \ket{J=j_1+j_2-1,M=j_1+j_2-1} = \alpha\ket{j_1,j_2;m_1=j_1,m_2=j_2-1}+\beta\ket{j_1,j_2;m_1=j_1-1,m_2=j_2}
\end{equation}
而作为 $\ms E(j_1,j_2)$ 的正交归一基的一部分,归一化条件要求 $|\alpha|^2+|\beta|^2=0$,正交性条件主要看 $\ms E(j_1+j_2)$ 中的 $\ket{J=j_1+j_2,M=j_1+j_2-1}$,该矢量也在分角动量表象下推出了表达式 \ref{eq:sprtd_anglr_mmntm_dcmps_1},因此系数应该满足
\begin{equation}
    \alpha\sqrt{\frac{j_2}{j_1+j_2}}+\beta\sqrt{\frac{j_1}{j_1+j_2}}=0
\end{equation}
这些约束提供的自由度不足以确定 $\alpha,\beta$,但可以约定 $\alpha,\beta$ 都为实数且 $\alpha>0$,就得到
\begin{equation}
    \ket{J=j_1+j_2-1,M=j_1+j_2-1} =\sqrt{\frac{j_1}{j_1+j_2}}\ket{j_1,j_2;m_1=j_1,m_2=j_2-1}-\sqrt{\frac{j_2}{j_1+j_2}}\ket{j_1,j_2;m_1=j_1-1,m_2=j_2}
\end{equation}

在其上使用 $J_-$ 就能得到全体 $\ket{J = j_1+j_2-1,M}$。然后考虑 $J$ 的第三大项 $\ms E(J=j_1+j_2-2)$……使用上述方法,直至 $\ms E(|j_1-j_2|)$ 结束,可见与上一节讨论 $J$ 的可能取值的思路类似,经过层层盘剥就能从 $J=j_1+j_2$ 始遍历 $\ms E(j_1,j_2)$ 中的全部基矢 $\ket{J,M}$。

\begin{definition}[C-G 系数]
至此,我们说明了每个空间 $\ms E(j_1,j_2)$ 中总角动量算符 $\bm J^2,J_z$ 的共同本征矢 $\ket{J,M}$ 都是分角动量表象下的基 $\{\ket{j_1,j_2;m_1,m_2}\}$ 的线性组合,形如
\begin{equation}
    \ket{J,M} = \sum_{m_1=-j_1}^{j_1}\sum_{m_2=-j_2}^{j_2}\ket{j_1,j_2;m_1,m_2}\braket{j_1,j_2;m_1,m_2}{J,M}
\end{equation}

相反的变换可以写作
\begin{equation}
    \ket{j_1,j_2;m_1,m_2} = \sum_{J=|j_1-j_2|}^{j_1+j_2}\sum_{M=-J}^J\ket{J,M}\braket{J,M}{j_1,j_2;m_1,m_2}
\end{equation}
其中展开系数通常选作实数 $\braket{j_1,j_2;m_1,m_2}{J,M}=\braket{J,M}{j_1,j_2;m_1,m_2}$,并满足归一化条件和一些约定,称为克莱布希-高登(Clebsch–Gordan)系数,简称 C-G 系数。
\end{definition}

\subsection{刚性转子问题}
刚体是受到一定约束条件的质点的集合,其运动比质点运动更加复杂,是经典力学中的重要话题。它的复杂性主要来源于方向的产生:在三维空间中,它的坐标不仅有质心的平动自由度,还有整个刚体方向改变而产生的旋转自由度,因而即使固定了质心,也需要讨论其定点转动问题。

\subsubsection{主轴坐标系}
刚体条件意味着质点间距离保持不变,这能够推出一个刚体的所有质点具有相同的角速度 $\bm \omega$。为了简便起见,可以将刚体视作离散的质点系,其中每个质点的质量为 $m_i$,相对质心的位矢为 $\bm r_i$,速度为 $\bm v_i=\bm\omega\times\bm r_i$。因此如果选定质心作为参考点 $O$ 并建立质心系,考虑刚体的角动量-角速度关系,有
\begin{equation}
    \bm{\mc J} = \sum_i m_i\bm r_i\times \bm v_i = \sum_i m_i\bm r_i\times(\bm\omega\times \bm r_i)
\end{equation}
拆解其中的各个坐标分量就会得到以下关系式
\begin{equation}
    \mc J_x = \omega_x\sum_i m_i(y_i^2+z_i^2)-\omega_y\sum_i m_i x_iy_i-\omega_z\sum_i m_ix_iz_i
\end{equation}
同理,对 $J_y,J_z$ 作类似处理,并作以下规定
\begin{equation}
    \mc I_{\alpha} = |\epsilon_{\alpha\beta\gamma}|\sum_i m_i(\beta_i^2+\gamma_i^2),\quad \mc I_{\alpha\beta} = -|\epsilon_{\alpha\beta\gamma}|\sum_i m_i\alpha_i\beta_i,\quad\alpha,\beta,\gamma\in\{x,y,z\}
\end{equation}
就会得到以下关系
\begin{equation}
    \begin{pmatrix}
    \mc J_x\\\mc J_y\\\mc J_z
    \end{pmatrix}=\begin{pmatrix}
    \mc I_x & \mc I_{xy} & \mc I_{xz}\\ \mc I_{yx} & \mc I_y & \mc I_{yz}\\ \mc I_{zx} & \mc I_{zy} & \mc I_z
    \end{pmatrix}\begin{pmatrix}
    \omega_x\\\omega_y\\\omega_z
    \end{pmatrix}
\end{equation}
又简记为 $\bm{\mc J}=\mbf I\bm\omega$,其中 $\mbf I$ 称为惯量张量,它表现为上式中的\textbf{惯量矩阵} $(\mc I)$。从定义可以看出,矩阵 $(\mc I)$ 是实对称矩阵,甚至能证明它是正定的,从而可以正交对角化。因此,我们只需作一个旋转变换,就能找到一个新的坐标系 $OXYZ$,使得该坐标系下满足
\begin{equation}
    \begin{pmatrix}
    \mc J_X\\\mc J_Y\\\mc J_Z
    \end{pmatrix}=\begin{pmatrix}
    \mc I_X & 0 & 0\\0 & \mc I_Y & 0\\0 & 0 & \mc I_Z
    \end{pmatrix}\begin{pmatrix}
    \omega_X\\\omega_Y\\\omega_Z
    \end{pmatrix}
\end{equation}

其中对角元 $\mc I_X,\mc I_Y,\mc I_Z$ 是原惯量矩阵 $(\mc I)$ 的三个正实数本征值,称为三个\textbf{惯量主轴}方向 $OX,OY,OZ$ 上的\textbf{转动惯量},三根主轴在原坐标系下的方向向量 $\bm e_X,\bm e_Y,\bm e_Z$ 是 $(\mc I)$ 对应于这些本征值的本征向量。由此可见,沿惯量主轴方向建立\textbf{主轴坐标系},将得到对角化的惯量矩阵。值得注意的是,从刚体外部来看,这一主轴系的方向取决于刚体内质点的具体位置,也就是说,主轴系是随刚体一同转动的坐标系。

接下来考虑转动动能 $\mc T$,能够证明
\begin{equation}
    \mc T = \frac 1 2\sum_i m_iv_i^2 = \frac 1 2\bm{\omega}^{\mr T}\mbf I\bm \omega
\end{equation}
在主轴系下,这意味着
\begin{equation}\label{eq:rgd_clsscl_rttn_kntc_enrgy}
    \mc T = \frac 1 2\left(\mc I_X\omega_X^2+\mc I_Y\omega_Y^2+\mc I_Z\omega_Z^2\right) = \frac 1 2\left(\frac{\mc J_X^2}{\mc I_X}+\frac{\mc J_Y^2}{\mc I_Y}+\frac{\mc J_Z^2}{\mc I_Z}\right)
\end{equation}

假设有一个刚体的质心固定在某个固定点 $O$ 处,有外部坐标系 $O\xi\eta\zeta$。假设 $\bm{\mc L}$ 是该刚体的角动量,刚体的三个惯量主轴形成的主轴坐标系为 $OXYZ$,于是刚体的机械能就是上述动能 $\mc T$。

\subsubsection{主轴分量与外部分量}

经典力学中质点的角动量 $\mc J_i$ 量子化得到一个轨道角动量算符 $\bm L_i$;为了简便起见,假设刚体只由可数个质点组成,则总角动量 $\mc J$ 量子化得到各轨道角动量算符的耦合:
\begin{equation}
    \bm J = \sum_i\bm L_i = \sum_i\bm R_i\times\bm P_i
\end{equation}

前面讨论过两个角动量的耦合问题,而多个角动量的耦合可以用先耦合其中的两个,再与更多的角动量逐个耦合的方式来处理。因此 $\bm J$ 和以前讨论过的总角动量没有本质区别:它也是角动量,即满足角动量分量的对易关系,例如
\begin{equation}
    [J_x,J_y] = \left[\sum_i L_{xi},\sum_i L_{yi}\right] = \i\hbar L_{zi}
\end{equation}

我们接下来开始讨论坐标变换对熟知的角动量算符的影响,为此提出几个引理
\begin{lemma}[角动量在某固定方向上分量的对易关系]
   在原有的 $Oxyz$ 坐标系中任取两个常矢量 $\bm e=(e_x,e_y,e_z)$ 和 $\bm e' = (e_x',e_y',e_z')$,则有
   \begin{equation}
       [\bm J\cdot\bm e,\bm J\cdot\bm e'] = \i\hbar(\bm e\times\bm e')\cdot\bm J
   \end{equation}
\end{lemma}
\begin{proof}
    经过计算可知
    \begin{align}
        [\bm J\cdot\bm e,\bm J\cdot\bm e']& = [e_xJ_x+e_yJ_y+e_zJ_z, e_x'J_x+e_y'J_y+e_z'J_z]\nonumber\\
        & = \i\hbar(e_xe_y'J_z-e_xe_z'J_y-e_ye_x'J_z+e_ye_z'J_x+e_ze_x'J_y-e_ze_y'J_x)\nonumber\\
        & = \i\hbar\begin{vmatrix}J_x & J_y & J_z\\ e_x & e_y & e_z\\ e_x' & e_y' & e_z'
        \end{vmatrix}\nonumber\\
        & = \i\hbar(\bm e\times \bm e')\cdot\bm J
    \end{align}
\end{proof}

假使有一组新的固定右手坐标系 $O\xi\eta\zeta$,它在原 $Oxyz$ 坐标系下表示的坐标基矢是 $\bm e_\xi,\bm e_\eta,\bm e_\zeta$,则 $\bm J$ 在某个坐标上的分量,以 $\xi$ 为例就是 $J_\xi = \bm J\cdot\bm e_\xi$。根据以上引理,有形如 $[J_\xi,J_\eta] = \i\hbar J_\zeta$ 的对易关系,因为右手坐标系基矢之间有 $\bm e_\xi\times\bm e_\eta = \bm e_\zeta$。所以基矢发生了坐标变换,得到的新分量 $J_\xi,J_\eta,J_\zeta$ 也可以像原来的坐标分量 $J_x,J_y,J_z$ 一样看待。

\begin{lemma}[总角动量与单个质点位置、动量的对易关系]
    \begin{equation}
        [\bm J,\bm R_i] = 0,\quad [\bm J,\bm P_i] = 0
    \end{equation}
\end{lemma}
\begin{proof}
    首先考虑 $[\bm J,\bm R_i]$,在 $\bm J$ 的组成中编号非 $i$ 的质点的项因为与 $\bm R_i$ 延伸自不同的态空间,因此可以丢弃,只需考虑 $[\bm L_i,\bm R_i]$。对于这两个矢量算符,以 $x$ 分量为例,$[L_{ix}, X_i]$ 中 $L_{ix} = Y_iP_{iz} - Z_iP_{iy}$,没有任何含 $x$ 分量的项,因此与 $X$ 对易;$y,z$ 分量,以及 $[L_{ix},P_{ix}]$ 亦是如此,这就说明了为什么 $\bm J$ 与任意一个质点的 $\bm R_i,\bm P_i$ 都对易。
\end{proof}

\begin{lemma}[总角动量在刚体坐标系下的分量的对易关系]\label{lem:ttl_anglr_mmntm_rgd_bdy_cmmt}
令 $\bm E,\bm E'$ 是刚体上的一些质点确定的方向向量,即
\begin{equation}
    \bm E = \sum_i a_i\bm R_i, \bm E' = \sum_i a_i'\bm R_i
\end{equation}

其中 $a_i,a_i'$ 是常系数(带有长度倒数的量纲,以使 $\bm E,\bm E'$ 无量纲化),则
\begin{equation}
    [\bm J\cdot\bm E,\bm J\cdot\bm E'] = -\i\hbar(\bm E\times\bm E')\cdot\bm J
\end{equation}
\end{lemma}
\begin{proof}
    首先,考虑到
    \begin{equation}
        [\bm J,\bm J\cdot\bm R_i] = [\bm J,J_xX_i+J_yY_i+J_zZ_i]
    \end{equation}
    取其中一项进行分析,提取出同一个态空间的分量,有
    \begin{equation}
        [\bm J,J_xX_i] = [J_{xi},J_{xi}X_i] = J_{xi}[J_{xi},X_i] + [J_{xi},J_{xi}]X_i = 0
    \end{equation}
    这里使用了刚证明的 $[\bm J,\bm R_i]=0$。综合三个分量,得到 $[\bm J,\bm J\cdot \bm R_i]=0$。
    
    然后考虑
    \begin{equation}
        [\bm J\cdot\bm R_i,\bm J\cdot\bm R_j] = [J_xX_i+J_yY_i+J_zZ_i,\bm J\cdot\bm R_j]
    \end{equation}
    取其中一项,有
    \begin{equation}
        [J_xX_i,\bm J\cdot \bm R_i] = [J_x,\bm J\cdot\bm R_j]X_i+J_x[X_i,\bm J\cdot R_j] = J_x[X_i,\bm J\cdot R_j]
    \end{equation}
    其中新出现的因子
    \begin{equation}
        [X_i,\bm J\cdot\bm R_j] = [X_i,J_xX_j+J_yY_j,J_zZ_j]
        = [X_i,J_yY_j]+[X_i,J_zZ_j]
        = [X_i,J_y]Y_j+[X_i,J_z]Z_j
    \end{equation}
    将角动量分量拆开,得到
    \begin{equation}
        [X_i,J_y] = [X_i,J_{yi}] = [X_i,Z_iP_{ix}-X_iP_{iz}]  = Z_i[X_i, P_{ix}] = \i\hbar Z_i
    \end{equation}
    类似地计算 $[X_i,J_z]$ 代回得到
    \begin{equation}
        [X_i,\bm J\cdot\bm R_j] = \i\hbar Z_iY_j - \i\hbar Y_i Z_j
    \end{equation}
    类似地计算 $[Y_i,\bm J\cdot R_j],[Z_i,\bm J\cdot R_j]$ 得到
    \begin{equation}
        [\bm J\cdot\bm R_i,\bm J\cdot\bm R_j] = \i\hbar\bm J\cdot(\bm R_j\times\bm R_i)
    \end{equation}
    然后代入 $\bm E,\bm E'$ 的表达式,有
    \begin{align}
        [\bm J\cdot\bm E,\bm J\cdot\bm E'] &= \left[\bm J\cdot\sum_ia_i\bm R_i,\bm J\cdot\sum_ja_j'\bm R_j\right] = \left[\sum_ia_i\bm J\cdot\bm R_i,\sum_ja_j'\bm J\cdot\bm R_j\right]\nonumber\\
        & = \sum_{i,j}a_ia_j'[\bm J\cdot R_i,\bm J\cdot R_j] = \sum_{i,j}a_ia_j'\i\hbar\bm J\cdot(\bm R_j\times\bm R_i)\nonumber\\
        & = -\i\hbar\left[\left(\sum_ia_i\bm R_i\right)\times\left(\sum_ja_j'\bm R_j\right)\right]\cdot\bm J = -\i\hbar(\bm E\times\bm E')\cdot\bm J
    \end{align}
\end{proof}

在经典力学中得到的主轴坐标系的方向取决于惯量张量 $\mbf I$ 的性质,而其中的每个分量如 $\mc I_x$ 或 $\mc I_{xy}$ 又是各质点在外部坐标系 $Oxyz$ 中的位矢 $\bm r_i$ 确定的,这也就是说:\textbf{主轴坐标系固定在刚体上,被各质点的坐标共同确定,随着各质点的转动一同转动}。

然而刚体的方位可以被其中不共线的三点完全确定,主轴坐标系从而也被确定。因此,现在假设我们所研究的刚体存在不共线的三点,使其位矢 $\bm R_1,\bm R_2,\bm R_3$ 不共面,则适当地线性组合这些位矢就可以得到三个互相正交且长度为 1 的正交基矢 $\bm E_X,\bm E_Y,\bm E_Z$,分别代表主轴坐标系 $OX,OY,OZ$ 的方向。我们约定这是一个右手系,也就是满足
\begin{equation}
    \bm E_X\times\bm E_Y = \bm E_Z,\ \bm E_Y\times\bm E_Z =\bm E_X,\ \bm E_Z\times\bm E_X = \bm E_Y
\end{equation}
而这些基矢都有引理 \ref{lem:ttl_anglr_mmntm_rgd_bdy_cmmt} 中的形式,所以适用于其结论。设角动量在主轴系中的分量形如 $J_X = \bm J\cdot\bm E_X$,就会得到
\begin{equation}
    [J_X,J_Y] = -\i\hbar J_Z,\ [J_Y,J_Z] = -\i\hbar J_X,\ [J_Z,J_X] = -\i\hbar J_Y
\end{equation}
有趣的是,它与任意外部坐标系中的角动量分量 $J_\xi,J_\eta,J_\zeta$ 的表现相差一个符号,这意味着 $J_X,J_Y,J_Z$ 并不是一般意义上的角动量分量算符。

\subsubsection{内、外角动量标准基}

从引理 \ref{lem:ttl_anglr_mmntm_rgd_bdy_cmmt} 的证明过程也可看出:$J_X,J_Y,J_Z$ 和 $\bm J$ 的一切函数,包括 $J_z,\bm J^2$ 等都对易。这与角动量算符 $\bm J^2$ 和 $J_z$ 的关系又是类似的。我们尝试用处理 $J_z$ 的方式来处理 $J_Z$。

定义新的互为伴随的升降算符:
\begin{equation}
    N_+ = J_X+\i J_Y,\quad N_- = J_X-\i J_Y
\end{equation}
它与正常的 $J_+,J_-$ 的最大区别在于下式:
\begin{equation}
    N_\pm N_\mp = \bm J^2-J_Z^2\mp\hbar J_Z
\end{equation}

由于 $[\bm J^2,\bm J_K]=0$,可以定义两算符的共同本征矢 $\ket{k,J,K}$,其中 $k$ 用于标记其他无关算符的本征值,$J,K$ 则满足
\begin{equation}
    \bm J^2\ket{k,J,K} = J(J+1)\hbar^2\ket{k,J,K},\quad J_Z\ket{k,J,K} = K\hbar\ket{k,J,K}
\end{equation}
根据以前的结论,只能知道其中的 $J$ 是自然数,而 $K$ 的取值范围首先由下式推知:
\begin{align}
    \|N_-\ket{k,J,K}\|^2 &=\braket{k,J,K|N_+N_-|k,J,K} = [J(J+1)-K(K+1)]\hbar^2 \ge 0\\
    \|N_+\ket{k,J,K}\|^2 &=\braket{k,J,K|N_-N_+|k,J,K} = [J(J+1)-K(K-1)]\hbar^2 \ge 0
\end{align}
同样解得 $-J\le K\le J$,且与 $J_z$ 不同的是,这里反而有
\begin{equation}
    N_-\ket{k,J,K=J} = 0,\quad N_+\ket{k,J,K=-J} = 0
\end{equation}

因为 $N_\pm$ 作为 $J_X,J_Y$ 的线性组合与 $\bm J^2$ 对易,所以 $N_\pm$ 不会影响 $\bm J^2$ 的本征值,如:
\begin{equation}
    \bm J^2 N_+\ket{k,J,K} = N_+\bm J^2\ket{k,J,K} = J(J+1)\hbar^2\bm J^2\ket{k,J,K}
\end{equation}

然而 $N_\pm$ 会影响到 $\bm J_Z$ 的本征值,首先考虑到对易关系
\begin{align}
    [J_Z,N_-] & = [J_Z,J_X-\i J_Y] = [J_Z,J_X] - \i[J_Z,J_Y]\nonumber\\
    & = -\i\hbar J_Y-\i\cdot\i\hbar J_X = -\i\hbar J_Y+\hbar J_X = \hbar N_-
\end{align}
进而得到
\begin{equation}
    J_Z(N_-\ket{k,J,K}) = N_-J_Z\ket{k,J,K}+\hbar N_-\ket{k,J,K} = (k+1)\hbar(N_-\ket{k,J,K})\propto \ket{k,J,K+1}
\end{equation}
这也体现出了 $N_\pm$ 和 $J_\pm$ 的区别:此时 $N_-$ 反而会使 $J_Z$ 的本征值增加 1,同理可证 $N_+$ 的减 1 作用。接着仿照此前处理 $J_z$ 的方式,就能导出 $K$ 的取值范围:
\begin{theorem}[$J_Z$ 的本征值]
   假设 $\ket{k,J,K}$ 是 $\bm J^2,J_Z$ 的共同本征矢,其中 $\bm J^2$ 的本征值为 $J(J+1)\hbar^2$,$J_Z$ 的本征值为 $K\hbar$,则只要 $K$ 的某个值存在,$K$ 就一定可以取遍 $-J,-J+1,\cdots,J$ 的所有值。
\end{theorem}
因为 $\bm J$ 是多个质点的轨道角动量算符耦合而来,所以 $J$ 一定来自整数之间的加减,得到的仍是整数,从而使得 $K$ 也一定只能取满足 $-J\le K\le J$ 的一切整数。

又由于 $[\bm J_z,\bm J_K]=0$,可以定义两算符的共同本征矢 $\ket{k,J,K,M}$,其中 $k$ 用于标记其他无关算符的本征值,$M$ 则满足 $J_z\ket{k,J,K,M} = M\ket{k,J,K,M}$。并且还可以验证 $[J_\pm,J_Z]=0$,因此升降算符 $J_\pm$ 依然只对 $M$ 起作用而不会影响 $K$。

于是,一组 C.S.C.O. 中可以同时包含 $\bm J^2,J_z,J_Z$,使得其共同本征矢 $\{\ket{k,J,M,K}\}$ 成为态空间的一组正交归一基,它同时拥有标记角动量在外部坐标系和内部坐标系中分量的指标。

\subsubsection{球形陀螺、对称陀螺和线性陀螺}
接下来可以求解刚体运动的定态问题。首先取一个固定的外部坐标系 $O\xi\eta\zeta$,再根据刚体各质点的坐标求出一个主轴坐标系 $O'XYZ$,其中 $O'$ 为刚体的质心。对于一个自由刚体,可以将其运动分解为质心的平动和刚体的转动,因此其动能(也即哈密顿量)也写成这两部分。

量子化后,若刚体的质量为 $M$,平动部分对应于以下哈密顿算符:
\begin{equation}
    H_\text{trans} = -\frac{\hbar^2}{2M}\nabla^2
\end{equation}
与一个自由粒子无异。在某些有心力场中,力场对刚体的作用等效于对质量与刚体相等,位于刚体质心的质点的作用,则此时的 $H_\text{trans}$ 还应该加上势能项 $V(\bm R_{O'})$,其中 $\bm R_{O'}$ 是质心在 $O\xi\eta\zeta$ 中的位置算符。整个算符只对质心的波函数起作用。

而如果取角动量算符在主轴系中的分量 $\bm J=(J_X,J_Y,J_Z)$,则根据经典力学中的结论 \ref{eq:rgd_clsscl_rttn_kntc_enrgy},哈密顿算符:
\begin{equation}
    H_\text{rot} = \frac 1 2\left(\frac{J_X^2}{\mc I_X}+\frac{J_Y^2}{\mc I_Y}+\frac{J_Z^2}{\mc I_Z}\right)
\end{equation}
整个算符只与角动量相关。而刚体角动量诸分量涉及到的所谓 $\bm R$ 都是主轴坐标系中的位矢,与固定坐标系中的 $\bm R_{O'}$ 分属不同的自由度,互相之间毫无关联。这就意味着 $H_\text{rot}$ 和 $H_\text{trans}$ 是对易的。

于是面对总的哈密顿算符 $H = H_\text{rot}+H_\text{trans}$,实际已经分出与固定坐标系相关的外部态空间及其中的算符 $H_\text{rot}$ 和与主轴坐标系相关的内部态空间 $H_\text{trans}$。解决 $H$ 的本征值问题,可以分别解决两项的本征值问题再将其本征矢以类似“张量积”的形式合并。对于 $H_\text{trans}$,其本征矢的形式要么是自由粒子的平面波函数,要么是在某个具体的势场中的特征形式。而对于 $H_\text{rot}$,其本征矢的形式才与刚刚进行的讨论相关。于是我们可以单独研究 $H_\text{rot}$,这相当于固定刚体的质心或转换到质心系下讨论。对于经典力学中的刚体转子,在质心系下全部自由度包括(相对于固定系 $O\xi\eta\zeta$ 的)三个欧拉角和三个主轴方向的角动量;而在量子力学中角动量观测值被 $J_X,J_Y,J_Z$ 的本征值所代表,欧拉角观测值则可以被这三个量与 $J_\xi,J_\eta,J_\zeta$ 的本征值之间的关系所代表,因此有理由认为如果 $H_\text{rot}$ 可以表达为 $\bm J^2,J_Z$ 的函数时(实际上从本征值问题 $H_\text{rot}\ket{\psi} = E\ket{\psi}$ 来看,也是如此) $\bm J^2,J_M,J_K$ 构成了 $H_\text{rot}$ 部分,或者质心固定刚体的完整态空间的 C.S.C.O.。因此在没有额外自由度的情况下,代表其他无关算符的指标 $k$ 可以舍去,共同本征矢 $\ket{J,M,K}$ 可以代表唯一的物理状态。

首先考虑最简单的球形陀螺情况,它不意味着刚体呈现球状,而只需要 $\mc I_X = \mc I_Y=\mc I_Z = \mc I$,此时注意到 $\bm J^2 = J_X^2+J_Y^2+J_Z^2$,有
\begin{equation}
    H_\text{rot} = \frac{J^2}{2\mc I}
\end{equation}
于是本征矢 $\ket{J,M,K}$ 对应的能量本征值 $E_J$ 也就正比于 $\bm J^2$ 的本征值,表现为
\begin{equation}
    E_J = \frac{J(J+1)\hbar^2}{2\mc I}
\end{equation}
而对于指标 $M,K$,它们所代表的 $J_\zeta,J_Z$ 皆不在 $H$ 中出现,因此各贡献了 $2J+1$ 个能量简并度,总体而言 $E_J$ 能级是 $(2J+1)^2$ 重简并的。

稍复杂的情况是对称陀螺,它对应于 $\mc I_X=\mc I_Y = I_\perp, \mc I_Z = I_\parallel$,此时有
\begin{equation}
    H_\text{rot} = \frac 1 2\left(\frac{J_Z^2}{I_\parallel}+\frac{\bm J^2-J_Z^2}{\mc I_\perp}\right)
\end{equation}
于是本征矢 $\ket{J,M,K}$ 对应的能量本征值为
\begin{equation}
    E_{J,K} = \frac{J(J+1)\hbar^2}{2I_\perp} +\frac{\hbar^2K^2}2\left(\frac 1{I_\parallel}-\frac 1{I_\perp}\right)
\end{equation}
可见对 $K$ 的 $2J+1$ 重简并被部分解除,只有 $K$ 互为相反数时才能得到相同的能级,然而对 $M$ 的 $2J+1$ 重简并依然保留。

$I_X\ne I_Y\ne I_Z$ 的一般情况的能级没有显式表达式,此时容易想到 $M$ 仍与能级无关,但 $K$ 的简并完全消除。实际上定态中 $K$ 及其对应的角动量分量 $J_Z$ 没有定值,只能表示为混合的本征态,具体计算在此按下不表。

最后考察特殊的线性陀螺:它不存在三个不共线的原子,所以无法根据刚体上的质点完全确定主轴坐标系的基矢(垂直于直线的两个坐标轴可以任意旋转)。经典力学的结论表明,对于一根线性刚体杆,其角动量总是垂直于杆,因此其不再有主轴坐标系中的自由度。固定系中的 $\mc J_\zeta,\mc J_\eta,\mc J_\xi$ 就不仅刻画角动量的大小和方向,还指明了刚体的方向。在量子化后,这意味着丧失了 $K$ 的存在,完全由 $M$ 来表征物理状态,因此从固定系研究角动量,有
\begin{equation}
    H_\text{rot} = \frac{\bm J^2}{2I}
\end{equation}
其本征矢就是 $\bm J^2$ 的本征矢 $\ket{J,M}$,其中 $M$ 提供了 $2J+1$ 重简并度。

\chapter{原子结构}

\section{实际粒子体系}

\subsection{自旋}

在此前的讨论中,我们一直在处理质点的运动,或者将实际的电子看做质点,它的态空间是表征运动的\textbf{轨道态空间} $\ms E_r$ 可以被位置表象或动量表象完全概括,它的哈密顿量在任何体系下都由位置算符和动量算符构成。然而,仅对于电子而言,一些实验结果显示出了新的问题:
\begin{enumerate}
    \item \textbf{氢原子光谱的精细结构}:我们已经推算出式 \ref{eq:hydrgn_enrgy_lvl} 所述的氢原子能级结构,每种能级间距都对应着一条谱线,而更精细的光谱观察表明,每一条谱线实际都由若干条相距极近的谱线组成。谱线的分裂暗示着能级的分裂,而新的能级的创生意味着原体系的哈密顿算符中有我们尚未关注到的微弱成分。
    \item \textbf{斯特恩-盖拉赫(Stern-Gerlach)实验}:磁场中射出的银原子束在屏上的分布分裂到两个关于初始直线轨迹对称的斑点中,这可能意味着角动量的量子数 $j$ 的半整数值的存在。根据定理 \ref{thm:anglr_mmntm_spctrm},这固然是可能的情况,但我们已经证明了定理 \ref{thm:orbtl_anglr_mmntm_spctrm},即由位置、动量组合成的轨道角动量的量子数 $l$ 只有可能是整数。这意味着一种新的形式的角动量的出现。
    \item 还有\textbf{反常塞曼效应}等实验结果也表明了对原子中电子角动量(轨道角动量)的认识的局限性。
\end{enumerate}

为了解决这些问题,一些粒子(如电子)被赋予了新的自由度:它的效果相当于粒子在任何参考系下都秉持着一个固有的角动量,但它并不意味着经典力学意义下的自转。自旋的起源是一种相对论效应,但在非相对论性量子力学中只能将其作为一种设定。

\begin{postulate}[自旋角动量]
粒子的自旋角动量算符 $\bm S$ 在与轨道态空间隔离的\textbf{自旋态空间} $\ms E_s$ 上起作用。
\end{postulate}

根据定义,其三个分量必然满足角动量对易关系,例如:
\begin{equation}
    [S_x,S_y] = \i\hbar S_z
\end{equation}

\begin{postulate}[自旋态空间的结构]
在 $\ms E_s$ 中,$\bm S^2$ 和 $S_z$ 构成 C.S.C.O.,所以存在一组由两个算符的共同本征态 $\ket{s,m}$ 构成的基,满足
\begin{equation}
    \bm S^2\ket{s,m} = s(s+1)\hbar^2\ket{s,m},\quad S_z\ket{s,m} = m\hbar\ket{s,m}
\end{equation}

每种粒子的 $s$ 是唯一的,称为该粒子的\textbf{自旋}。
\end{postulate}

根据定理 \ref{thm:anglr_mmntm_spctrm},这个 $s$ 为半整数或整数,而 $m$ 遍及 $\mp s$ 及之间与 $\pm s$ 相差为整数的一切数值。给定粒子后 $\ms E_s$ 永远是 $2s+1$ 维空间,所有 $\ms E_s$ 中的态都是 $\bm S^2$ 对应于本征值 $s(s+1)\hbar^2$ 的本征矢。

\begin{postulate}[态空间的完整结构]
粒子的态空间 $\ms E = \ms E_{\bm r}\ox\ms E_s$
\end{postulate}

因此自旋算符与轨道算符总是对易的,且除了 $s=0$ 的退化情况,仅给出 $\ms E_{\bm r}$ 中的右矢不足以描述粒子的态,也即 $\bm R$ 或 $\bm P$ 或轨道态空间中的任何算符在 $\ms E$ 中都无法构成 C.S.C.O.,必须加上 $\ms E_s$ 中的 C.S.C.O. 才算圆满。

\subsubsection{电子自旋的特殊性质}

电子是化学体系最为关注的微观粒子,它的自旋如下:
\begin{postulate}[电子的自旋]
电子的自旋为 1/2。
\end{postulate}
这意味着 $\dim\ms E_s=2$,可以在其自旋态空间中选取 $\ket{+},\ket{-}$ 满足以下性质:
\begin{itemize}
    \item \textbf{是 $\bm S^2,S_z$ 的共同本征矢}
    \begin{equation}
        \bm S^2\ket{\pm} = \frac 3 4\hbar^2\ket{\pm},\quad S_z\ket{\pm} = \pm\frac 1 2\hbar\ket{\pm}
    \end{equation}
    \item \textbf{正交归一性}
    \begin{equation}
        \braket{+|-} = 0,\quad \braket{+|+}=\braket{-|-} = 1
    \end{equation}
    \item \textbf{完备性}
    \begin{equation}
        \ket{+}\bra{+}+\ket{-}\bra{-} = I
    \end{equation}
    其中 $I$ 是 $\ms E_s$ 中的恒等算符
\end{itemize}
于是 $\{\ket{+},\ket{-}\}$ 是 $\ms E_s$ 中的正交归一基。

$\bm S$ 还继承了角动量的升降算符,即 $S_\pm = S_x\pm\i S_y$,根据角动量的普遍结论,有
\begin{equation}
    S_+\ket{+} = 0,\quad S_+\ket{-} = \hbar\ket{+},\quad S_-\ket{+} = \hbar\ket{-},\quad S_-\ket{-} = 0
\end{equation}

从这些性质可以继续导出 $S_x,S_y$ 在 $\ket{+},\ket{-}$ 上的作用。于是在这两个右矢构成的表象下,可以求出 $\bm S$(的三个分量)的矩阵,具体表现为
\begin{equation}
    (\bm S) = \frac{\hbar} 2\bm\sigma:\quad \sigma_x=\begin{pmatrix}
    0 & 1 \\ 1 & 0
    \end{pmatrix},\quad \sigma_y=\begin{pmatrix}
    0 & -\i \\ \i & 0
    \end{pmatrix},\quad \sigma_z=\begin{pmatrix}
    1 & 0 \\ 0 & -1
    \end{pmatrix}
\end{equation}
这里 $\bm \sigma$ 的三个分量称为\textbf{泡利(Pauli)矩阵}。

\begin{theorem}[泡利矩阵的性质]
   通过矩阵运算,容易证明以下性质
   \begin{itemize}
       \item 酉矩阵:
       \begin{equation}
           \sigma_i^2 = 1,\quad i\in\{x,y,z\}
       \end{equation}
       \item 对易关系:
       \begin{equation}
           [\sigma_i,\sigma_j] = 2\i\epsilon_{ijk}\sigma_k,\quad i,j,k\in\{x,y,z\}
       \end{equation}
       \item 群关系:
       \begin{equation}
           \sigma_i\sigma_j = \i\epsilon_{ijk}\sigma_k,\quad i,j,k\in\{x,y,z\}
       \end{equation}
       \item 迹:
       \begin{equation}
           \text{Tr}\,\sigma_i = 0,\quad i\in\{x,y,z\}
       \end{equation}
       \item 行列式:
       \begin{equation}
           \det \sigma_i = -1,\quad i\in\{x,y,z\}
       \end{equation}
   \end{itemize}
\end{theorem}

由于 $2\times 2$ 矩阵本身只构成四维向量空间,而三个泡利矩阵与单位矩阵可以组成线性无关组,因此有
\begin{corollary}[$2\times 2$ 矩阵空间的基]
    任意 $2\times 2$ 矩阵都可以分解为单位矩阵和三个泡利矩阵的线性组合。
\end{corollary}

从表象下的矩阵性质可以导出算符的性质,即
\begin{corollary}[电子的角动量算符的特殊性质]
    \begin{itemize}
       \item 
       \begin{equation}
           S_i^2 = 1,\quad i\in\{x,y,z\}
       \end{equation}
       \item
       \begin{equation}
           S_iS_j+S_jS_i = 0,\quad i,j,k\in\{x,y,z\},i\ne j
       \end{equation}
       \item
       \begin{equation}
           \sigma_i\sigma_j = \frac\i 2\hbar\epsilon_{ijk}\sigma_k,\quad i,j,k\in\{x,y,z\}
       \end{equation}
       \item
       \begin{equation}
           S_+^2=S_-^2=0
       \end{equation}
   \end{itemize}
\end{corollary}

\subsubsection{电子的旋量描述}
接下来寻求对电子更完整的描述:将轨道态空间和自旋态空间结合起来。于是若要在 $\ms E = \ms E_{\bm r}\ox\ms E_s$ 中寻找一组 C.S.C.O.,就要将 $\ms E_{\bm r}$ 中的 C.S.C.O. 和 $\ms E_s$ 中已知的角动量表象结合起来,如以下方案:
\begin{itemize}
    \item \textbf{位置表象}:$\{X,Y,Z,\bm S^2,S_z\}$,普遍情况
    \item \textbf{动量表象}:$\{P_x,P_y,P_z,\bm S^2,S_z\}$,普遍情况
    \item \textbf{轨道角动量表象}:$\{H,\bm L^2,L_z,\bm S^2,S_z\}$,适用于中心力场问题
\end{itemize}
而且由于 $\bm S^2$ 的本征矢遍及 $\ms E_s$,本征值都为 $1/2$,因此可以在 C.S.C.O. 中忽略其存在。

取位置表象的方案,所得的表象称为 $\{\ket{\bm r,\ve}\}$,其中 $\ve=\pm$。于是任意 $\ket{\psi}\in\ms E$ 都可以在此展开
\begin{equation}
    \ket{\psi} = \sum_\ve\int\d^3 r\ket{\bm r,\ve}\braket{\bm r,\ve|\psi}
\end{equation}
也即表示为波函数 $\braket{\bm r,\ve}{\bm r} = \psi_\ve(\bm r)$。这个波函数有连续的三元组自变量 $\bm r$ 和离散指标 $\ve$,因此实际上电子的态包含两个波函数,即
\begin{equation}
    \psi_+(\bm r) = \braket{\bm r,+|\psi},\quad \psi_-(\bm r) = \braket{\bm r,-|\psi}
\end{equation}
它可以写作\textbf{二分量旋量}表达式:
\begin{equation}
    [\psi](\bm r) = \begin{pmatrix}
    \psi_+(\bm r)\\\psi_-(\bm r)
    \end{pmatrix}
\end{equation}

对于左矢,取共轭得到
\begin{equation}
    \bra{\psi} = \sum_\ve\int\d^3 r\braket{\psi|\bm r,\ve}\bra{\bm r,\ve}=\sum_\ve\int\d^3 r\psi_\ve^*(\bm r)\bra{\bm r,\ve}
\end{equation}
可以写作伴随旋量
\begin{equation}
    [\psi]^*(\bm r) = \begin{pmatrix}
    \psi_+^*(\bm r) & \psi_-^*(\bm r)
    \end{pmatrix}
\end{equation}

右矢的内积本应这样计算:
\begin{equation}
    \braket{\psi|\vphi} = \sum_\ve\int\d^3 r\braket{\psi|\bm r,\ve}\braket{\bm r,\ve|\vphi} = \int\d^3 r[\psi_+^*(\bm r)\vphi_+(\bm r)+\psi_-^*(\bm r)\vphi_-(\bm r)]
\end{equation}
也可以自然地写成旋量运算
\begin{equation}
    \braket{\psi|\vphi} = \int\d^3 r[\psi]^*(\bm r)[\vphi](\bm r)
\end{equation}

容易看出,这种旋量表示法是将轨道态空间的自由度在位置表象下缩并为波函数,而把自旋态空间的自由度仍表示为 2 维向量,它的总体结构就是以 $\bm r$ 为自变量的向量值函数空间,与 $\ms E = \ms E_{\bm r}\ox\ms E_s$ 同构。因此 $\ms E$ 上的算符在位置表象下也由两部分组成:在函数空间上的变换和 2 维向量空间上的变换,它们共同组成了一个以函数变换为矩阵元的 $2\times 2$ 矩阵。我们将这种算符分为三类:
\begin{itemize}
    \item \textbf{自旋算符}:定义在 $\ms E_s$ 中,是 $\bm S$ 的衍生物,因而仍保留泡利矩阵及其运算后得到的形式。
    
    以 $S_+$ 为例,它作用于任何 $\ket{\bm r,+}$ 都得到 0,作用于任何 $\ket{\bm r,-}$ 都得到 $\hbar\ket{\bm r,+}$,因此 $\ket{\psi'} = S_+\ket{\psi}$ 表示为
    \begin{equation}
        \ket{\psi'} = \hbar\int\d^3 r\psi_-(\bm r)\ket{\bm r,+}
    \end{equation}
    计算表象下的分量,得到
    \begin{equation}
        \braket{\bm r,+|\psi'} = \psi_+'(\bm r) = \hbar\psi_-(\bm r),\quad \braket{\bm r,-|\psi'} = \psi_-'(\bm r) = 0
    \end{equation}
    因此旋量表示为
    \begin{equation}
        [\psi'](\bm r) = \hbar\begin{pmatrix}
        \psi_-(\bm r) \\ 0
        \end{pmatrix}
    \end{equation}
    整个变换可以概括为旋量矩阵
    $
        \llbracket\hat S_+\rrbracket = \dfrac{\hbar} 2\left(\sigma_x+\i\sigma_y\right)
    $
    与 $[\psi](\bm r)$ 的乘法。
    \item \textbf{轨道算符}:定义在 $\ms E_{\bm r}$ 中,由于对 $\ms E_s$ 中的分量没有影响,所以在位置表象下,它实际上是 $\ms E_{\bm r}$ 中算符与单位矩阵的复合,例如
    \begin{equation}
        \llbracket\hat X\rrbracket =\begin{pmatrix}
        x & 0 \\ 0 & x
        \end{pmatrix},\quad
        \llbracket \hat P_x\rrbracket = \frac\hbar\i\begin{pmatrix}
        \dfrac\partial{\partial x} & 0 \\ 0 & \dfrac\partial{\partial x}
        \end{pmatrix}
    \end{equation}
    \item \textbf{混合算符}:$\ms E$ 中最普遍的算符,它表示为一般的 $2\times 2$ 矩阵。
\end{itemize}

利用这种完整描述再看氢原子,主量子数 $n$ 的能量简并并不只有轨道态空间中角动量算符的角量子数 $l$ 和磁量子数带来的 $n^2$ 个简并度,还包含电子自旋态空间中 $\ket{\pm}$ 两个简并度,以及质子自旋态空间中 $\ket{\pm}$ 两个简并度。当然,如果某些条件变化(如外加磁场)使得电子自旋或质子自旋的角动量参与到哈密顿算符中,这种自旋简并可能就会被破坏。

\subsubsection{电子自旋的耦合}

以电子为例,自旋 $1/2$ 粒子的单粒子自旋态空间已经在上文得到充分的解析。如果只关注两个粒子的自旋自由度,可以将两粒子的自旋态空间取张量积得到共同的表象
\begin{equation}
    \{\ve_1,\ve_2\} = \{\ket{+,+},\ket{+,-},\ket{-,+},\ket{-,-}\}
\end{equation}
确立这一表象的根基在于单粒子自旋态空间中 C.S.C.O. 的拼接:$\bm S_1^2,S_{1z},\bm S_2^2,S_{2z}$ 是双粒子自旋态空间的 C.S.C.O.,各自的作用为:
\begin{equation}
    \bm S_i^2\ket{\ve_1,\ve_2} = \frac 3 4\hbar^2\ket{\ve_1,\ve_2}, \quad S_{iz}\ket{\ve_1,\ve_2} = \ve_i\frac\hbar 2\ket{\ve_1,\ve_2},\quad i = 1,2
\end{equation}

体系的总自旋角动量算符定义为 $\bm S = \bm S_1+\bm S_2$,根据角动量耦合的一般理论,我们可以取双粒子自旋态空间的另一组 C.S.C.O.:$\bm S_1^2,\bm S_2^2,\bm S^2,S_z$。由于 $S_1^2,S_2^2$ 的本征值被 $1/2$ 自旋固定,所以这四个算符的共同本征矢可以简记为 $\ket{S,M}$,满足以下关系:
\begin{align}
    \bm S_1^2\ket{S,M} =\bm S_2^2\ket{S,M} = \frac 3 4\hbar^2\ket{S,M},\quad \bm S^2\ket{S,M} = S(S+1)\hbar^2\ket{S,M},\quad S_z\ket{S,M} = M\hbar\ket{S,M} 
\end{align}
其中 $S$ 的可能取值为 $\dfrac 1 2 + \dfrac 1 2 =1,\left|\dfrac 1 2 - \dfrac 1 2\right|=0$。$M$ 的可能取值为 $-1,0,1$,并且简并度为
\begin{equation}
    g_{\frac 1 2,\frac 1 2}(-1) = g_{\frac 1 2,\frac 1 2}(1) = 1,\ g_{\frac 1 2,\frac 1 2}(0) = 2
\end{equation}

我们关注的态空间是 $\ms E(j_1=1/2,j_2=1/2)$,以 $S$ 不同的子空间 $\ms E(J\equiv S)$ 进行直和分解,其中的第一项为 $\ms E(S = 1/2+1/2=1)$,$M=1$ 的唯一态一定只能在这个空间中,记作 $\ket{S = 1,M=1}$,我们可以适当选择系数,在满足归一化的同时,满足:
\begin{equation}
    \ket{S=1,M=1} = \ket{j_1=1/2,j_2=1/2;m_1=1/2,m_2=1/2} = \ket{+,+}
\end{equation}
然后利用 $J_-$ 算符,得到
\begin{align}
    \ket{S=1,M=0} &= \sqrt{\frac{1/2}{1/2+1/2}}\ket{1/2,1/2;-1/2,1/2}+\sqrt{\frac{1/2}{1/2+1/2}}\ket{1/2,1/2;1/2,-1/2} \nonumber\\&= \frac 1{\sqrt 2}\left(\ket{+,-} + \ket{-,+}\right)
\end{align}
然后利用子空间 $\ms E(S=1)$ 和 $\ms E(S=0)$ 之间的正交关系,得到
\begin{align}
    \ket{S=0,M=0} &= \sqrt{\frac{1/2}{1/2+1/2}}\ket{1/2,1/2;-1/2,1/2}-\sqrt{\frac{1/2}{1/2+1/2}}\ket{1/2,1/2;1/2,-1/2} \nonumber\\&= \frac 1{\sqrt 2}\left(\ket{+,-} - \ket{-,+}\right)
\end{align}

至此,双电子角动量态空间中总角动量表象与分角动量表象之间的关系完全确立,我们可以定义以下常见概念:
\begin{definition}[单重态与三重态]
双电子角动量态空间在总角动量表象下的本征矢中,三个矢量 $\ket{S=1,M}\ (M=\pm 1,0)$ 构成三重态,单个矢量 $\ket{S=0,M=0}$ 构成单重态。

\end{definition}

\subsection{全同粒子问题}
两个一切固有性质完全一样的粒子称为全同粒子,到现在为止,我们已经认识到的固有性质包括影响动力学性质的质量,参与电磁相互作用的电荷和相对论效应带来的自旋。粒子的全同性使得我们无法特别地区分其中的某一个,\textbf{交换体系中的两个全同粒子,体系的性质和演变规律不应该发生任何变化}。

量子力学处理全同粒子时的困难可以概括为\textbf{交换简并}问题:对于含有 $N$ 个全同粒子的体系,每个粒子都联系着一个态空间以及来自这些态空间的算符。如果给粒子编号为 $1,2,\cdots,N$,则单粒子态空间为 $\mc E(1),\cdots,\mc E(N)$,整体的态空间是它们的张量积。利用每个单粒子态空间中的 E.C.O.C. 的共同本征值,可以构成整体态空间的一组基。由于离子是全同的,这些 E.C.O.C. 中的算符也是相同的,且它们的谱也是相同的。假设每个 E.C.O.C. 的全体本征值浓缩为指标 $b_1,\cdots,b_N$,由于全同粒子不可分辨,当我们测量某个粒子的物理量得到本征值 $b_i$ 时,指定 $i$ 本身是不可能的,因为这种编号本身就不具有物理上的可能性。唯一可能做到的是对每个粒子逐个测量,无关编号。假设得到了 $N$ 个互异本征值 $b_1,\cdots,b_N$,则可以将这 $N$ 个本征值任意归属给 $N$ 个粒子,一共有 $\text{A}_N^N = N!$ 种排列方式。因为我们测量的是 E.C.O.C. 对应的物理量,因此在单粒子情况下,属于\textbf{完全测量}——能够唯一确定粒子的状态,然而在多个全同粒子的情况下,这 $N!$ 中可能使得我们无法唯一确定整体态空间中的唯一右矢。

这一问题在我们现有的五条基本公设、量子化方法、关于张量积的基本假设和自旋态空间的假设框架下无法解决:我们必须引入新的设定来消除这种不确定性。为了阐明这种设定的意义和后果,需要先引入一些新的数学工具和运算方式。

\subsubsection{置换算符与置换群}

\begin{definition}[置换]
    一个长度为 $N$ 的排列是一个有序数组 $(i,j,k,\cdots)$,其中 $i,j,k,\cdots$ 各不相同且都属于自然数的子集 $\{1,\cdots,N\}$。这一排列也可简记为 $(ijk\cdots)$,所有长度为 $N$ 的排列组成的集合记作 $\text{perm}\,n$
    一个置换 $\pi$ 可以视作发生在 $\text{perm}\,n$ 上的函数,它也可以用 $(npq\cdots)\in\text{perm}\,n$ 来标记,其含义是 $\pi(ijk\cdots)$ 的结果是将 $(ijk\cdots)$ 中的 $1$ 换成 $n$,$2$ 换成 $q$……$i$ 换成 $(npq\cdots)$ 中的第 $i$ 个数字。
\end{definition}

考虑 $N$ 个并不全同,但自旋相同的粒子,分别编号为 $1,\cdots,N$:之所以这样设定,是因为全同粒子会引发标号区分上的困难,而自旋相同会使得在当前的认知范围内粒子的态空间是同构的。然后用张量积 $\{\ket{1:u_i;\cdots;N:u_k}\}$ 描述体系的态空间的基,这里的 $\ket{u_i}$ 等都是各个同构的单粒子态空间中一组共用的基中的基矢。

\begin{definition}[置换算符]
    置换算符是一个线性算符,在 $N$ 个粒子组成的如上所述的态空间中,有 $N!$ 种置换算符,它们都可以写作 $P_{ijk\cdots}$,其中 $(ijk\cdots)\in\text{perm}\,N$。对于任意一种排列 $(np\cdots q)\in\text{perm}\,N$,置换算符在基矢上的作用形如:
    \begin{equation}
        P_{np\cdots q}\ket{1:u_i;2:u_j;\cdots;N:u_k} = \ket{n:u_i;p:u_j;\cdots q:u_k}
    \end{equation}
\end{definition}

直观上来看,固定各个单粒子上的基矢 $u_i$ 等的位置,这些粒子的编号构成了一个排列:而置换算符的下标实际代表着一个置换 $\pi$,在置换算符作用后,相当于这个排列被这个置换所转变。

我们将从置换的代数结构入手来研究置换算符:
\begin{definition}[置换群]
    长度为 $N$ 的全部置换构成了一种特殊的集合,称为置换群。之所以称为群,是因为有以下特性:
    \begin{enumerate}
        \item 这个集合中有乘法运算,且在该运算下封闭:由于置换是一个映射,两个置换 $\pi_i,\pi_j$ 的乘法就是映射的复合 $\pi_i\pi_j\equiv\pi_i\circ\pi_j$。容易验证一个长度为 $N$ 的排列进行两次置换后得到的仍是一个长度为 $N$ 的排列,因此两个长度为 $N$ 的置换的复合仍是一个长度为 $N$ 置换,它总属于这个置换群,即所谓封闭性。
        \item 存在幺元:所谓幺元即是与其进行乘法运算后不发生任何变化的元素。容易发现“不变”也是特殊的置换 $\pi_e =(12\cdots N)$,因此对于任意置换 $\pi_i$ 都有 $\pi_e\pi_i=\pi_i\pi_e=\pi_i$,所以 $(12\cdots N)$ 就是所谓幺元。
        \item 存在逆元:很容易想象每个置换都有其逆映射:假设一个置换将数 $i$ 变换为 $p_i$,则其逆映射只需要将 $p_i$ 重置为 $i$。由于置换用排列来标记,它不可能将两个不同的数 $i,j$ 同时变换为 $p_i$,因此这个逆映射是良定义的,它也是一个长度相同的置换,位于该置换群中。这个逆映射就称为置换的逆元。
    \end{enumerate}
\end{definition}

置换算符就是在基矢上对粒子的编号进行置换,因此置换算符也构成一个置换算符群,其中的乘法就是置换算符的乘法。可以定义一个映射 $\Phi$,假设 $(ijk\cdots)$ 代表一个置换,则 $\Phi[(ijk\cdots)] = P_{ijk\cdots}$。这是一个双射,且满足以下性质:若 $\pi_i,\pi_j$ 是(用排列表示的)置换,则
\begin{equation}
    \Phi(P_{\pi_i})\Phi(P_{\pi_j}) = \Phi(P_{\pi_i}P_{\pi_j})
\end{equation}
由此可以这样说:
\begin{lemma}[置换算符群与置换群的关系]
    置换算符群与置换群同构,两个群之间的同构运算即是以上定义的 $\Phi$。
\end{lemma}

$\Phi$ 是双射,所以两个群中的元素一一对应;$\Phi$ 保持群乘法,所以两个群中的结构完全相同。这使得置换群和置换算符群在数学上是不可区分的,我们只需要研究其中一个,就可以得到另一个集合作为群的全部性质。因此接下来将从置换本身出发,探讨置换算符的代数性质,然后再引申出群结构之外的,算符所在的向量空间中的特性。

\begin{definition}[对换与位调算符]
    对换是最简单的置换,在一次对调中只有两个数的位置发生交换。如果 $\pi_i$ 是一个对换,则 $P_{\pi_i}$ 称为位调算符。
\end{definition}

对换与一般的置换相比有特殊性质:
\begin{lemma}[对换的性质]
    对换的周期为 2。这也就是说,对于一个对换 $\pi_i$,有 $(\pi_i)^2 =\pi_e$
\end{lemma}
因此对于位调算符,也有 $\left(P_{\pi_i}\right)^2=1$,即它是自身的逆。

位调算符有超出对换本身的特殊性:
\begin{lemma}[位调算符的性质]
    位调算符是厄米算符
\end{lemma}
\begin{proof}
    不失一般性,只讨论对换前两个粒子的位调算符 $P_{21\cdots}$。同时为了简便起见,将态空间的基在第三个粒子及之后的粒子的分量都省略,于是这个算符的矩阵元为
    \begin{equation}
        \Braket{1:u_{i'};2:u_{j'}|P_{21}|1:u_i;2:u_j} = \Braket{1:u_{i'};2:u_{j'}|2:u_j;1:u_i} = \delta_{i'j}\delta_{j'i}
    \end{equation}
    其伴随 $P_{21}^*$ 的矩阵元为
    \begin{align}
        \Braket{1:u_{i'};2:u_{j'}|P_{21}^*|1:u_i;2:u_j} &=\left(\bra{1:u_i;2:u_j}P_{21}\ket{1:u_{i'};2:u_{j'}}\right)^*\nonumber\\
        & =\left(\Braket{1:u_i;2:u_j|1:u_{j'};2:u_{i'}}\right)^* = \delta_{ij'}\delta_{ji'}
    \end{align}
    因此 $P_{21}$ 和 $P_{21}^*$ 的每个矩阵元都相等,说明 $P_{21} = P_{21}^*$。
\end{proof}

再结合对换的性质导致的 $(P_{21})^2 = I$,我们可以概括出算符的一类新性质:
\begin{definition}[幺正算符]
    一个算符 $A$ 是幺正算符,是指 $A^*A = AA^* = I$。
\end{definition}
因此 $P_{21}^*P_{21} = (P_{21})^2 = I$,一般的位调算符都是幺正的。

一般的置换与对换的关系如下:
\begin{theorem}[置换的对换分解]
   每个给定的置换都可分解为对换的乘积,这种分解不唯一,但分解式中对换个数的奇偶性是固定的,称为置换的宇称,也称为置换表示成的排列的宇称。对于任意长度 $N\ (N>2)$,在 $\text{perm}\,n$ 中奇宇称和偶宇称的置换(排列)个数相等。
\end{theorem}
同理我们可以提置换算符分解为位调算符的乘积、置换算符的宇称,以及对于任意粒子数,奇置换算符与偶置换算符同样多。而且因为置换算符是若干幺正的位调算符的乘积,所以置换算符都是幺正的。根据位调算符的厄米性,置换算符的伴随就是其位调算符分解式颠倒过来;然而因为位调算符不总是对易的,所以置换算符一般不是厄米算符。

\subsubsection{对称化假定}

$N>2$ 的置换算符之间的不可对易性导致不可能用这些算符的共同本征矢构成一组基。但确实存在一些右矢,是所有置换算符的本征矢。

\begin{definition}[完全对称与完全反对称右矢]
    假设 $P_\alpha$ 是 $N$ 粒子体系态空间 $\ms E$ 上的置换算符,其中 $\alpha$ 是长度为 $N$ 的排列。如果对于所有这样的置换算符 $P_\alpha$,右矢 $\ket{\psi_S}$ 都满足 $P_\alpha\ket{\psi_S} = \ket{\psi_S}$,则称 $\ket{\psi_S}$ 是完全对称的。而如果右矢 $\ket{\psi_A}$ 总满足 $P_\alpha\ket{\psi_A} = \ve_\alpha\ket{\psi_A}$,其中 $\ve_\alpha$ 在 $\alpha$ 为偶排列时为 $1$,在 $\alpha$ 为奇排列时为 $-1$,则称 $\ket{\psi_A}$ 是完全反对称的。
    
    完全对称右矢的集合构成 $\ms E$ 的子空间 $\ms E_S$,完全反对称右矢的集合构成 $\ms E$ 的子空间 $\ms E_A$。
\end{definition}

构造完全对称或反对称的右矢需要这样的工具:
\begin{definition}[对称化算符与反对称化算符]
    在 $N$ 粒子体系态空间 $\ms E$ 上定义以下两个算符:
    \begin{align}
        S &= \frac{1}{N!}\sum_{\alpha\in\text{perm}\,N}P_\alpha\\
        A &= \frac{1}{N!}\sum_{\alpha\in\text{perm}\,N}\ve_\alpha P_\alpha
    \end{align}
    分别称为对称化算符和反对称化算符。
\end{definition}

为了证明这两个算符有我们需要的性质,首先提出群论中的一个基本定理:

\begin{lemma}[群的重排定理]
    给定任意群 $G = \{g_\alpha\}$,对于任意的群元 $u\in G$,诸 $ug_\alpha$ 各不相同,且 $G=\{ug_\alpha\}$。
\end{lemma}

以及投影算符的一个判据
\begin{lemma}[厄米算符与投影算符的关系]\label{lem:Hrmt_prjct}
    假设 $\ms E$ 是一个态空间,$P\in\mc L(\ms E)$ 且 $P^2=P$,则 $P$ 是到自己的像空间 $\text{range}\,P$ 的投影算符当且仅当 $P$ 是厄米算符。
\end{lemma}

最后为两个算符正名:

\begin{theorem}[对称化算符与反对称化算符的作用]\label{thm:symmtrztn_oprtr}
    以上定义的 $S$ 和 $A$ 分别是到 $\ms E_S$ 和 $\ms E_A$ 的投影算符。
\end{theorem}
\begin{proof}
    首先设 $P_{\alpha_0}$ 也是 $\ms E$ 中的置换算符,则对于任意的置换算符 $P_\alpha\in\mc L(\ms E)$,$P_\beta = P_{\alpha_0}P_{\alpha}$ 显然也是置换算符,且只要拆分成对调算符的乘积,就可以理解其奇偶性关系:$\ve_\beta = \ve_{\alpha_0}\ve_\alpha$。
    
    $P_{\alpha_0}$ 和 $P_\alpha$ 都是 $N$ 粒子置换算符群的群元,根据重排定理,给定 $\alpha_0$,取遍所有排列 $\alpha$,则 $P_\beta$ 将遍历整个群,因此可以推出
    \begin{equation}
        P_{\alpha_0}S = \frac 1{N!}\sum_{\alpha\in\text{perm}\,N}P_{\alpha_0}P_\alpha = \frac 1{N!}\sum_{\beta\in\text{perm}\,N}P_\beta = S
    \end{equation}
    以及
    \begin{equation}
        P_{\alpha_0}A = \frac 1{N!}\sum_{\alpha\in\text{perm}\,N}\ve_{\alpha}P_{\alpha_0}P_\alpha = \frac 1{N!}\ve_{\alpha_0}\sum_{\beta\in\text{perm}\,N}\ve_{\beta}P_\beta = \ve_{\alpha_0}A
    \end{equation}
    事实上同理可证 $AP_{\alpha_0}=A,SP_{\alpha_0}=S$。以上说明的问题是:经过 $S$ 作用后的算符都是完全对称算符,经过 $A$ 作用后的算符都是完全反对称算符,也即
    \begin{equation}
        \text{range}\,S\in\ms E_S,\quad\text{range}\,A\in\ms E_A
    \end{equation}
    
    假设 $\ket{\psi_S}$ 是完全对称右矢,$\ket{\psi_A}$ 是完全反对称右矢,则有
    \begin{align}
        S\ket{\psi_S} &= \frac 1{N!}\sum_{\alpha\in\text{perm}\,N}P_\alpha\ket{\psi_S} = \frac 1{N!}\sum_{\alpha\in\text{perm}\,N}\ket{\psi_S} = \ket{\psi_S}\\
        A\ket{\psi_A} &= \frac 1{N!}\sum_{\alpha\in\text{perm}\,N}\ve_\alpha P_\alpha\ket{\psi_A} = \frac 1{N!}\sum_{\alpha\in\text{perm}\,N}\ve_\alpha^2\ket{\psi_A} = \ket{\psi_A}\\
    \end{align}
    这又说明
    \begin{equation}
        \ms E_S\in \text{range}\,S,\quad\ms E_A\in\text{range}\,A
    \end{equation}
    因此
    \begin{equation}
        \ms E_S= \text{range}\,S,\quad\ms E_A=\text{range}\,A
    \end{equation}
    
    此外还有
    \begin{align}
        S^2 &= \frac 1{N!}\sum_{\alpha\in\text{perm}\,N}P_\alpha S = \frac{1}{N!}\sum_{\alpha\in\text{perm}\,N}S = S\\
        A^2 &= \frac 1{N!}\sum_{\alpha\in\text{perm}\,N}\ve_\alpha P_\alpha A = \frac{1}{N!}\sum_{\alpha\in\text{perm}\,N}\ve_\alpha^2A = A
    \end{align}
    而且 $S$ 和 $A$ 都是厄米算符,因为其求和式中的项虽然不都是厄米算符,但在取伴随之后,都会得到另一个宇称相同的置换算符。根据置换算符的幺正性,取伴随与取逆无异,而每个算符的逆都在一一对应地处在群中,相当于只是改变了求和的顺序,并没有改变最终的结果。
    
    至此 $S,A$ 已经满足了引理 \ref{lem:Hrmt_prjct} 的条件,能够推出它们是到各自像空间,也即 $\ms E_S,\ms E_A$ 的投影算符。
\end{proof}

至此可以提出量子力学公理体系中的又一条补充内容,它的目的是消除交换简并:

\begin{postulate}[对称化假定]
    体系含有多个全同粒子时,其态空间中只有一部分右矢描述状态,称为物理右矢,要么是完全对称右矢,要么是完全反对称右矢。全同粒子因此分为两类:前者对应玻色子,其全同粒子体系的态空间为 $\ms E_S$;后者对应费米子,其全同粒子体系的态空间为 $\ms E_A$。
\end{postulate}

\begin{assumption} 玻色子和费米子的经验规律 

自旋为半整数的粒子是费米子,自旋为整数的粒子是玻色子
\end{assumption}

重新审视交换简并问题,它的数学描述是:对于含有 $N$ 个全同粒子的体系,如果右矢 $\ket{u}$ 描述了它的状态,则对任意长度为 $N$ 的置换 $\alpha$,置换算符作用后 $P_\alpha\ket{u}$ 和 $\ket{u}$ 都描述相同的物理状态。记 $\ket{u}$ 和全体 $P_\alpha\ket{u}$ 张成了子空间 $\ms E_u$,其中所有右矢都描述同一个物理状态,而 $\ms E_u$ 的维数最小为 1,最大为 $N!$,这取决于体系中不同粒子作为 $\ket{u}$ 中各分量在不同本征态上的分布(假设两个粒子处于 $\ket{u}$ 中分量的同一本征态,则因为粒子本身的不可区分性,编号实际是无意义的,仅交换这两个粒子的置换算符不会产生与原有右矢线性无关的新右矢,因而不产生新的维数)。而只要这个维数大于 1,就有多个在以往意义下可区分的(线性无关的)右矢对应同一个物理状态(却可能给出不同的物理预言),因而出现了交换简并。

而对称化假定能够消除交换简并,是因为这样一个结论:

\begin{theorem}[对称化假定消除交换简并]
    以上定义的 $\ms E_u$ 中只含有 $\ms E_S$ 中的一个右矢,或 $\ms E_A$ 中的一个右矢。
\end{theorem}
\begin{proof}
    在定理 \ref{thm:symmtrztn_oprtr} 的证明中,得到了 $S=SP_\alpha,A =\ve_\alpha AP_\alpha$,因此
    \begin{equation}
        S\ket{u} = SP_\alpha\ket{u},\quad A\ket{u} = \ve_\alpha AP_\alpha\ket{u}
    \end{equation}
    考虑到 $S,A$ 是投影算符,其含义是:张成 $\ms E_u$ 的诸右矢在 $\ms E_S,\ms E_A$ 上的投影都是唯一的
\end{proof}

换言之,可以作以下定义
\begin{definition}[物理右矢]
    根据费米子或玻色子的性质,对称化假定在交换简并的全同粒子态空间 $\ms E_u$ 中挑出了唯一右矢,假设这一右矢非零,则称其为\textbf{物理右矢}。
\end{definition}

\subsubsection{泡利不相容原理}

根据以上论述,构建全同粒子体系的右矢的具体方法如下:
\begin{enumerate}
    \item 任意给粒子编号,并将各粒子的物理状态分配给这些号码,得到张量积右矢 $\ket{u}$。
    \item 根据全同粒子是玻色子或费米子,将 $S$ 或 $A$ 作用于 $\ket{u}$ 上。
    \item 归一化所得的右矢。
\end{enumerate}

对于玻色子,$S$ 的作用就是将所有置换态求平均值,而对于费米子,$A$ 在不同宇称的置换项中正负号的差异与行列式有异曲同工之妙。假设 $\vphi_1,\cdots,\vphi_N$ 各自是一些归一化的单粒子右矢,设
\begin{equation}
    \ket{u} = \ket{1:\vphi_1;\cdots;N:\vphi_N}
\end{equation}
则有 \textbf{斯莱特(Slater)行列式}:
\begin{equation}
    A\ket{u} = \frac{1}{N!}\begin{vmatrix}\ket{1:\vphi_1} & \cdots & \ket{1:\vphi_N}\\ \vdots & & \vdots \\ \ket{N:\vphi_1} & \cdots & \ket{N:\vphi_N}
    \end{vmatrix}
\end{equation}
这将导致一个重要的结果:假如单粒子态 $\ket{\vphi_1},\cdots,\ket{\vphi_N}$
中有任意两个相同,则行列式中就会有两列相同,使得 $A\ket{u} =0$,不再能成为物理右矢。这是对称化假设所不容的情况,我们将其总结为以下规律

\begin{corollary}[泡利不相容原理]
    同一个量子态不可能同时被多个全同费米子所占据。
\end{corollary}

泡利不相容原理在全同费米子体系中带来了极为深刻的影响。首先,物理右矢所在的物理态空间 $\ms E_S$ 和 $\ms E_A$ 的基的构成就不一样。从单粒子态空间的基 $\{u_i\}$ 可以构筑 $\ms E$ 中的基 $\{1:u_i;2:u_j;\cdots;N:u_k\}$,将算符 $S$ 或 $A$ 作用于这组基上,就得到了张成 $\ms E_S$ 或 $\ms E_A$ 的右矢组,但这个右矢组是线性相关的,因为在原来的基中先进行任意的诸粒子置换 $P_\alpha$,再经 $S$ 或 $A$ 作用,由已证的 $SP_\alpha = S$ 和 $AP_\alpha = \ve_\alpha A$ 可知,总会得到同一个右矢。

为了提出物理态空间中的线性无关组,需要引入以下概念:
\begin{definition}[占有数]
    对于右矢 $\ket{1:u_i;2:u_j;\cdots;N:u_k}$,单粒子态 $\ket{u_p}$ 的占有数 $n_p$ 是其在序列 $\ket{u_i},\ket{u_j},\cdots,\ket{u_k}$ 中出现的次数,即处于 $\ket{u_k}$ 态的粒子数。
\end{definition}
显然 $\sum\limits_k n_k=N$。假如两个互异右矢中各态的占有数分别都相等,则可以通过置换算符在两个右矢之间互化,因此通过 $S$ 或 $A$ 作用于两个右矢将得到同一个右矢,即唯一的态

\begin{definition}[福克态]
    对于玻色子,这个唯一的归一化态这样表示:
    \begin{equation}
    \ket{n_1,n_2,\cdots,n_k,\cdots} = cS|\underbrace{1:u_1;\cdots;n_1:u_1}_{\text{有} n_1 \text{个粒子处在态} \ket{u_1}};\underbrace{n_1+1:u_2;\cdots;n_1+n_2:u_2}_{\text{有}\,n_2 \text{个粒子处在态}\,\ket{u_2}};\cdots\rangle
    \end{equation}
    我们可以约定所有的 $n_j>0$,这样这个表示中提及的都是占有粒子的态 $\ket{u_j}$;但也可以从 $\ket{u_i}$ 的角度考虑:无论每个态的占有数是否为零,都用 $n_i$ 在态矢中如实记录,这样的处理方式是比较方便的。
    
    对于费米子,这个唯一的归一化态这样表示:
    \begin{equation}
        \ket{u_i,u_j,\cdots,u_l,\cdots} = \begin{cases}
        cA\ket{1:u_i;2:u_j;\cdots;q:u_l;\cdots} & \text{所有}\,u_i\,\text{不同}\\
        0 & \text{存在两个}\,u_i\,\text{相同}
    \end{cases}
    \end{equation}
    这个态实际上在玻色子写法的基础上增加了对粒子顺序的强调,它的粒子数意义即
    \begin{equation}
        \ket{u_i,u_j,\cdots,u_l,\cdots} \in \left\{\ket{0,\cdots,0,n_i=1,n_j=1,0,\cdots,0,n_l=1,0,\cdots}\right\}
    \end{equation}
    
    其中的 $c$ 都是归一化系数。按照以上方式定义的物理右矢称为体系的福克(Fock)态。
\end{definition}

考虑两个福克态右矢的内积。因为它们实际都包含 $N!$ 个诸粒子置换的项,所以其内积有 $(N!)^2$ 个项。假如两个右矢在相同,则这个内积就是福克态右矢的范数。对于费米子而言,算符 $A$ 中各种置换产生的 $N!$ 项各不相同、互相正交,因而在内积的 $(N!)^2$ 项中,有 $N!$ 项涉及相同的归一化右矢的内积,为总和贡献 $N!$,而其他项都是正交右矢之间的内积,为总和贡献 0;再考虑到 $A$ 前自带的 $1/N!$ 因子,所以能够计算出对于 $N$ 费米子福克态,
\begin{equation}
    \left\|\ket{u_i,u_j,\cdots,u_l,\cdots}\right\|^2 = c^2\cdot\left(\frac 1{N!}\right)^2\cdot N! = \frac{c^2}{N!}
\end{equation}
因此费米子福克态的归一化系数为 $c=\sqrt{N!}$。

对于玻色子而言,算符 $S$ 各种置换可能产生相同的项。事实上,如果一个态 $\ket{u_1}$ 
的占有数为 $n_1$,则所有仅交换这 $n_1$ 个粒子的置换都得到相同的态。因此如果考虑到所有态的占有数,任取一个福克态右矢 $\ket{n_1,\cdots,n_k}$,在经历的所有的 $N!$ 种置换中,能找到 $n_1!\cdots n_k!$
个相同的结果。所以 $N!$ 个项可以分成 $\frac{N!}{n_1!\cdots n_k!}$
个等价类,每一类中都有实际相同的 $n_1!\cdots n_k!$
个态,这些态与自身和与同类态的内积都是 1,在总和中贡献了 $(n_1!\cdots n_k!)^2$ 
而考虑到类的数量,以及 $S$ 算符自带的因子 $1/N!$,所以能够计算出对于 $N$ 玻色子福克态:
\begin{equation}
    \left\|\ket{n_1,\cdots,n_k}\right\|^2 = c^2\cdot\left(\frac 1{N!}\right)^2\cdot\frac{N!}{n_1!\cdots n_k!}\cdot (n_1!\cdots n_k!)^2 = \frac{c^2 n_1!\cdots n_k!}{N!}
\end{equation}
因此可以求出一个归一化因子用占有数表示的通式,结合前面对费米子的讨论,它对两种粒子都适用:
\begin{equation}
    c = \sqrt{\frac{N!}{n_1!\cdots n_k!}}
\end{equation}

前述内积中的项都是张量积矢量的内积,也就是各分量内积的乘积。而只要两个右矢不同,则说明在某个态的占有数不同,就不可能存在一项,使得各分量恰好都处于同一个态,从而使这一项的内积乘积不为 0(我们不妨假设 $\{u_i\}$ 是正交基)。因此,从这个定义可以推出用福克态右矢之间的正交关系,只要参与以下内积的右矢不为零,就有
\begin{equation}
            \Braket{n_1,n_2,\cdots,n_k,\cdots|n_1',n_2',\cdots,n_k',\cdots} = \delta_{n_1n_1'}\cdots\delta_{n_kn_k'}\cdots
\end{equation}

此时就可以揭示玻色子和费米子的物理态空间中基的区别。

\begin{theorem}[福克态右矢作为正交归一基]
    \begin{itemize}
        \item 对于全同玻色子体系,各占有数 $n_k$ 取任意自然数,并满足 $\sum\limits_k n_k=N$ 的诸右矢 $\ket{n_1,n_2,\cdots,n_k,\cdots}$ 构成其物理态空间 $\ms E_S$ 中的正交归一基。
        \item 对于全同费米子体系,各占有数 $n_k$ 取 0 或 1,并满足 $\sum\limits_k n_k=N$ 的诸右矢 $\ket{n_1,n_2,\cdots,n_k,\cdots}$ 构成其物理态空间 $\ms E_A$ 中的正交归一基。
    \end{itemize}
\end{theorem}
\begin{proof}
    考察 $\{\ket{n_1,n_2,\cdots,n_k}\}$ 的来历:它是投影算符 $S$ 或 $A$ 作用在 $\ms E$ 的基上的结果,因此张成了投影像空间 $\ms E_S$ 或 $\ms E_A$。对于玻色子,根据 $S$ 的定义,它是多个 $\ms E$ 中相互正交且非零的右矢之加权和,因而也是非零的;对于费米子,根据 $A$ 的定义和斯莱特行列式,只要某个占有数大于 1,则所得矢量为 0;而若所有占有数都是 1 或 0,则每个单粒子态 $\ket{u_i}$ 或有粒子,或没有粒子,$A$ 的定义式中所有的置换算符都会使至少两个粒子所处的单粒子态发生变化,因而得到互相正交的非零右矢,其加权和也非零。再根据 $\{\ket{n_1,n_2,\cdots,n_k}\}$ 内部在非零前提下的正交关系可知,它确实是物理态空间 $\ms E_S$ 或 $\ms E_A$ 中的基。
\end{proof}

玻色子和费米子的巨大差异还体现在全同粒子系的基态上。假设体系中所有粒子彼此独立,可以视为没有相互作用,于是对应的哈密顿算符是单粒子哈密顿算符的和,形如
\begin{equation}
    H(1,\cdots,N) = h(1)+\cdots+h(N)
\end{equation}
粒子的全同性使得 $H$ 中对编号 $1,\cdots,N$ 是对称的,因此分解后各 $h(i)$ 项的形式都相同。只需在单粒子态空间中求解单粒子哈密顿算符的本征值问题 $h(i)\ket{\psi_n} = e_n\ket{\psi_n}$,假定 $h(i)$ 的谱离散无简并。则对于全同玻色子体系,物理本征矢为
\begin{equation}
    \Ket{\Psi_{n_1,\cdots,n_N}^{(S)}} = c\sum_{\alpha\in\text{perm}\,N} P_\alpha\ket{1:\psi_{n_1};\cdots;N:\psi_{n_N}}
\end{equation}
对应的能量为
\begin{equation}
    E_{n_1,\cdots,n_N} = e_{n_1}+\cdots+e_{n_N}
\end{equation}
如果 $e_1$ 是最小的本征值,对应于本征态 $\ket{\psi_1}$,则如果所有玻色子都处于这一个态,就是系统能量最低的\textbf{基态}。此时能量和本征矢为
\begin{equation}
    E_{1,\cdots,1} = Ne_1,\quad\Ket{\Psi_{1,\cdots,1}^{(S)}} = \ket{1:\psi_1;\cdots;N:\psi_1}
\end{equation}

而对于费米子体系,根据泡利不相容原理,$N$ 个粒子共处一态不成立。假设单粒子能量本征值单调递增(这是 S-L 本征值问题的特点),形如
\begin{equation}
    e_1<e_2<\cdots<e_{n}<\cdots
\end{equation}
则基态能量为
\begin{equation}
    E_{1,\cdots,N} = e_1+\cdots+e_N
\end{equation}
本征矢为
\begin{equation}
    \Ket{\Psi_{1,\cdots,N}^{(A)}} = \frac 1{\sqrt{N!}}\begin{vmatrix}\ket{1:\psi_1} & \cdots & \ket{1:\psi_N}\\\vdots & & \vdots \\ \ket{N:\psi_1} & \cdots & \ket{N:\psi_N}\end{vmatrix}
\end{equation}
可以看出,给定一个体系,向其中填充 $N$ 个全同费米子,所形成的基态中往往最大单粒子能量 $e_N\ne e_1$,称为体系的\textbf{费米能}。

\subsection{粒子的二次量子化}

所谓二次量子化又称为第二类量子化,它能够用于处理粒子数很多的全同粒子问题,并解除了量子力学中粒子数守恒的桎梏。

\subsubsection{福克空间及其中的算符}
\begin{definition}[福克空间]
    $N$ 个玻色子/费米子构成的对称/反对称福克态右矢构成的空间记为 $\ms E_S(N)$ 和 $\ms E_A(N)$。于是定义福克空间为
    \begin{align}
        \ms E_\text{Fock}^S &= \ms E_S(0)\oplus\ms E_S(1)\oplus\cdots\oplus\ms E_S(N)\oplus\cdots\\
        \ms E_\text{Fock}^A &= \ms E_S(0)\oplus\ms E_S(1)\oplus\cdots\oplus\ms E_S(N)\oplus\cdots
    \end{align}
    并且赋予其内积结构:既然所有粒子数 $N$ 的福克态都混入一个空间中,我们就要求所有 $N$ 不同的右矢都正交。
\end{definition}

由于 $N$ 粒子福克态就是 $\ms E_{S,A}$ 的一组基,所以自然的想到,福克空间的基中的矢量也形如 $\ket{n_1,\cdots,n_l\cdots}$,只是没有了 $\sum n_i=N$ 的约束。

这种不固定粒子数的空间允许了一种奇异的、混合了不同粒子数的叠加态,我们暂不追究其物理意义,反而提出一组算符以在新的空间中对粒子数这一自由度进行充分地操作。

\begin{definition}[产生算符]
    在玻色子的福克空间 $\ms E_\text{Fock}^S$ 中,对于任意形如 $\ket{n_1,\cdots,n_i\cdots}$ 的基矢,如果其中 $n_i$ 指的是该态在单粒子态 $\ket{u_i}$ 上的占有数,则可以定义产生算符 $a_i^*$,使得
    \begin{equation}
        a_{u_i}^*\ket{n_1,\cdots,n_i,\cdots} = \sqrt{n_i+1}\ket{n_1,\cdots,n_i+1,\cdots}
    \end{equation}
    
    在费米子的福克空间 $\ms E_\text{Fock}^A$ 中,对于任意形如 $\ket{u_j,\cdots,u_k\cdots,u_l}$ 的基矢,还要补充定义
    \begin{equation}
        a_{u_i}^*\ket{u_j,\cdots,u_k\cdots,u_l} = \ket{u_i,u_j,\cdots,u_k\cdots,u_l}
    \end{equation}
    总是将 $u_i$ 插入到最左边的位置。
\end{definition}

在

\subsubsection{交换与反交换关系}

\subsubsection{单粒子对称算符}

\subsubsection{多粒子算符}

\section{量子力学中的近似方法}

\subsection{定态微扰理论}

\subsection{变分法}

\subsection{含时微扰理论}

\subsection{HF近似}

\section{多电子原子}

\appendix

\chapter{一些集合论记号}
\section{集合的笛卡尔积}
从熟知的有序对(ordered pair)出发,可以定义更长的有序组合:
\begin{definition}[元组]
   长度为 $n\ (n>1)$ 的元组(tuple)又称为 $n$ 元组,形如 $(a_1,a_2,\cdots,a_n)$,它等价于二元组的嵌套 $((\cdots((a_1,a_2),a_3),\cdots),a_n)$。各 $a_i$ 称为该元组的坐标(coordinate)或分量。
\end{definition}
于是有序对即是 2 元组。由于有序对相等要求两个分量上的元素分别相等,因此容易推知以下结论:
\begin{theorem}[元组相等的条件]
    $n$ 元组 $(a_1,\cdots,a_n)\ (n>1)$ 和 $(b_1,\cdots,b_n)$ 相等,等价于 $\forall i\in\{1,\cdots,n\}\,(a_i=b_i)$。
\end{theorem}

元组所属的集合与其各分量所属的集合的关系描述如下:

\begin{definition}[笛卡尔积]\label{def:Crtsn_prdct}
    集合 $A_1,A_2,\cdots,A_n$ 的笛卡尔积(Cartesian product)是一个由这些集合中元素结合成的元组组成的集合 $A_1\times A_2\times\cdots\times A_n=\{(a_1,\cdots,a_n)|a_1\in A_1,\cdots,a_n\in A_n\}$
\end{definition}
特别地,全体有序实数对组成的集合为 $\R\times\R$,简写为 $\R^2$,类似地可以推知 $\R^n$ 以及任意集合 $A$ 带来的 $A^n$ 的含义。
\begin{instance}
如果建立了一个空间直角坐标系,则三维空间中的点都可以表示为坐标 $(x,y,z)$ 的形式,它正是此处定义的三元组,由于每个坐标分量都能遍及全部实数,所以所有这样的三维坐标的集合构成 $\R^3$。这也是常常把三维空间简记为 $\R^3$ 的原因。
\end{instance}
\section{映射}
利用元组的概念可以规范化映射的表示:
\begin{definition}[映射]
   从 $X$ 到 $Y$ 的映射(mapping)记作 $f:X\ra Y$。它是 $X\times Y$ 的一个子集,满足只要给定一个 $x\in X$,就只存在唯一的 $y\in Y$ 使得 $(x,y)\in f$,又记作 $y=f(x)$。这时候 $X$ 称为 $f$ 的定义域(domain),记作 $X=\text{dom} f$,$x\in X$ 是 $f$ 的自变量;而 $Y$ 称为 $f$ 的陪域(codomain),$y\in Y$ 是 $f$ 的因变量。
\end{definition}
映射根据不同的应用场景,又有函数(function)等别名。
\begin{definition}[多元映射]
   如果映射可以表示为 $f:X_1\times\cdots\times X_n\ra Y$,则可称为 $n$ 元映射。若 $x_i\in X_i,\,i=1,\cdots,n$,则 $f(x_1,\cdots,x_n)=f((x_1,\cdots,x_n))$。
\end{definition}
\begin{instance}
可以定义一个二元映射 $\text{plus}:\C^2\ra \C$ 满足 $\text{plus}\,(a,b)=a+b$。于是 $\text{plus}$ 的含义与加法符号 $+$ 没有本质区别:只是通常把 $\text{plus}\,(a,b)=c$ 简记为 $a+b=c$。因此\textbf{运算}(operation)符也可以用多元映射来定义。
\end{instance}
\begin{instance}
为了进一步说明这一抽象,可以定义 $\text{oplus}:\C^2\ra \C$ 满足 $\text{oplus}\,(a,b)=a+b+ab$,并且把 $\text{oplus}\,(a,b)=c$ 简记为 $a\oplus b=c$。尽管这是初等数学练习中的常见小把戏,但恰如其分地说明了这一思想。
\end{instance}


\chapter{线性代数}
运算是一类特殊的映射,代数的研究对象常常是带有运算的集合。对线性代数的关注正是被附带线性运算的集合所吸引:
\section{复向量空间}
\begin{definition}[(复)向量空间]\label{def:cmplx_vctr_spc}
   (复)向量空间 $V$ 起源于一个集合 $V$,其中的元素称为向量。它的要素可以表示为四元组 $(V,\C,+,\cdot)$:
   \begin{itemize}
       \item $\C$ 是复数集,其中的元素又称为标量。
       \item 向量加 $\text{vadd}:V^2\ra V$ 是一个二元映射,又称为一个二元运算,对于 $\vphi,\psi\in V$,$\text{vadd}(\vphi,\psi)$ 常记作 $\vphi+\psi$。
       \item 标量乘 $\text{smul}:\C\times V\ra V$ 也是一个二元运算,对于 $\vphi,\psi\in V$,$\text{smul}(\vphi,\psi)$ 常记作 $\vphi\cdot\psi$ 并简记为 $\vphi\psi$。
   \end{itemize}
   $+$ 和 $\cdot$ 正是线性运算的符号,具体的运算规则可以按需指定,其前提是满足以下性质:
   \begin{enumerate}
        \item \textbf{封闭性}:若 $\psi,\vphi\in V,\, a\in\C$,则 $\psi+\vphi\in V, a\psi\in V$;
        \item \textbf{交换律}:若 $\psi,\vphi\in V$,则 $\psi+\vphi=\vphi+\psi$;
        \item \textbf{结合律}:设 $\psi,\vphi,\chi\in V,\ a,b\in\C$,则 $(\psi+\vphi)+\chi=\psi+(\vphi+\chi)$ 且 $a(b\psi)=(ab)\psi$;
        \item \textbf{加法单位元}:存在元素 $0\in V$ 使得对所有 $\psi\in V$ 都有 $\psi + 0 =\psi$,它又称为零向量;
        \item \textbf{加法逆元}:对于每个 $\psi\in V$ 都存在 $\vphi\in V$ 使得 $\psi+\vphi=0$;
        \item \textbf{乘法单位元}:对于所有 $\psi\in V$ 都有 $1\psi=\psi$,其中 $1$ 就是复数 $1$;
        \item \textbf{分配律}:设 $\psi,\vphi\in V,\ a,b\in\C$,则 $a(\psi+\vphi)=a\psi+a\vphi,\ (a+b)\psi=a\psi+b\psi$。
    \end{enumerate}
\end{definition}

\begin{instance}
复数集 $\C$ 本身自然地引出一个向量空间 $(\C,\C,+,\cdot)$,其中的 $+,\cdot$ 遵循常规的复数运算法则。此后提及复数集 $\C$ 时也可以同时代指这个向量空间。
\end{instance}

\begin{instance}
复数 $n$ 元组的集合 $\C^n$ 也自然地构成一个向量空间 $(\C^n,\C,+,\cdot)$:两个元组相加,就是把每个分量相加;一个标量(复数)乘元组,就是把每个分量与这个数相乘。正因为如此,当 $n>1$ 时 $\C^n$ 中的元素又被简称为(复)向量,而 $\C^n$ 也同时代指这个向量空间。
\end{instance}

\begin{definition}[线性映射]\label{def:linearmap}
从某个向量空间 $\ms E$ 到某个向量空间 $\ms F$ 的映射 $A$ 被称作线性映射,需要满足以下性质

    \begin{enumerate}
        \item \textbf{加性}:若 $\psi,\vphi\in\ms E$,则 $A(\psi+\vphi)=A\psi+A\vphi$
        \item \textbf{齐性}:若 $\psi\in\ms E,\ a\in\C$,则 $A(a\psi)=a(A\psi)$
    \end{enumerate}
    
此类线性映射组成集合 $\mc L(\ms E,\ms F)$。
\end{definition}

可见特地定义此类映射的意义是,它可以把定义域作为向量空间的运算关系原封不动地转移到陪域上,这是一种“保持结构的映射”或称为\textbf{同态}。

\begin{instance}
由于 $\C$ 也是一个向量空间,从任意向量空间 $\ms E$ 到 $\C$ 的线性映射称为 $\ms E$ 上的\textbf{线性泛函}。这种映射的特殊性在于它的陪域 $\C$ 不仅是向量空间,也是向量空间所附带的标量集合。
\end{instance}

\begin{instance}
态空间映向自身的线性映射称为\textbf{算符},这一定义蕴含的特殊性无需多言。$\ms E$ 上的算符的集合又记作 $\mc L(\ms E,\ms E):=\mc L(\ms E)$
\end{instance}

\section{复内积空间}

\chapter{多元函数微积分}

\chapter{常微分方程}

\chapter{傅里叶变换}
\fi

\end{document}
